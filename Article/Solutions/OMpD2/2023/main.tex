\documentclass[12pt]{article}
\usepackage{../../sty/articlept}

\olympiad{OMpD 2022 N2 Fase 2}

\title{\olympiad}
\author{Samuel de Araújo Brandão}
\date{\today}

\begin{document}
\maketitle
\preface

\tableofcontents

\clearpage

\section{Problemas}
\begin{enumerate}[label={\textbf{\arabic*.}}]
	\item Alguns amigos internautas formaram $6$ equipes de futebol, e decidiram realizar
	      um torneio onde cada equipe enfrenta outra exatamente uma vez em uma partida.
	      Em cada partida, quem vence ganha $3$ pontos, quem perde não ganha ponto,
	      e se as duas equipes empatarem, cada uma leva $1$ ponto.

	      Ao final do torneio, verificou-se que as pontuações das equipes foram
	      $10, 9, 6, 6, 4$ e $2$ pontos. A respeito deste torneio, responda os
	      seguintes itens, justificando sua resposta em cada um deles.

	      \begin{itemize}
		      \item[(a)] Quantas partidas terminaram empatadas no torneio?
		      \item[(b)] Determine, para cada um das $6$ equipes, o número de vitórias, empates e derrotas.
		      \item[(c)] Se considermos apenas as partidas disputadas entre a equipe que fez $9$ pontos contra as duas equipes que fizeram $6$ pontos, e a disputada entre as duas equipes que fizeram $6$ pontos, explique porque dentre essas três partidas, há pelo menos $2$ empates.
	      \end{itemize}
	\item Encontre todos os pares $a,b$ de números reais tais que o número
	      $\lfloor an + b \rfloor$ é um quadrado perfeito, para todo inteiro positivo
	      $n$.

	      Observação: $\lfloor x \rfloor$ é o maior inteiro menor ou igual a $x$.
	      Por exemplo, $\lfloor 4 \rfloor = 4$, $\lfloor -5 \rfloor
		      = -5$, $\lfloor \pi \rfloor = 3$, $\lfloor \frac{1+\sqrt{5}}{2} \rfloor
		      = 1$ e $\lfloor -2,7 \rfloor = -3$.
	\item Para cada inteiro positivo $x$, seja $\varphi(x)$ o número de inteiros
	      $1 \leq k \leq x$ que não possuem fatores primos em comum com $x$. Por
	      exemplo, $\varphi(12) = 4$, pois $1, 5, 7$ e $11$ são todos os
	      números naturais menores ou iguais a $12$ que não possuem fatores
	      primos em comum com $12$. Determine todos os inteiros positivos
	      $n$ tais que existem inteiros positivos distintos $a_1,a_2,\ldots
		      ,a_n$ de modo que o conjunto:
	      \[
		      S = \{a_1, a_2, \ldots, a_n, \varphi(a_1), \varphi(a_2), \ldots, \varphi(a_n)\}
	      \]
	      possui exatamente $2n$ inteiros consecutivos (em alguma ordem).
	\item Seja $ABC$ um triângulo acutângulo escaleno, e seja $O$ seu
	      circuncentro. Seja $K$ um ponto sobre o lado $BC$. A reta $OK$ intersecta
	      o circuncírculo do triângulo $BOC$ novamente em $M$. Seja $L$ o simétrico
	      de $K$ relativo a $AC$. Demonstre que os circuncírculos dos triângulos
	      $LCM$ e $ABC$ são tangentes se, e somente se, $AK \perp BC$.
\end{enumerate}

\clearpage

\section{Soluções}
\subsection{Problema 1.}
\begin{problem}
Alguns amigos internautas formaram $6$ equipes de futebol, e decidiram realizar
um torneio onde cada equipe enfrenta outra exatamente uma vez em uma partida.
Em cada partida, quem vence ganha $3$ pontos, quem perde não ganha ponto,
e se as duas equipes empatarem, cada uma leva $1$ ponto.

Ao final do torneio, verificou-se que as pontuações das equipes foram
$10, 9, 6, 6, 4$ e $2$ pontos. A respeito deste torneio, responda os
seguintes itens, justificando sua resposta em cada um deles.

\begin{itemize}
	\item[(a)] Quantas partidas terminaram empatadas no torneio?
	\item[(b)] Determine, para cada um das $6$ equipes, o número de vitórias, empates e derrotas.
	\item[(c)] Se considermos apenas as partidas disputadas entre a equipe que fez $9$ pontos contra as duas equipes que fizeram $6$ pontos, e a disputada entre as duas equipes que fizeram $6$ pontos, explique porque dentre essas três partidas, há pelo menos $2$ empates.
\end{itemize}
\end{problem}

\begin{enumerate}[label=(\textbf{\alph*})]
	\item 8 partidas terminaram em empate.

	      É trivial perceber isso ao notar que, já que existem $6$ times e cada
	      time jogou contra exatos $5$ times, houveram $5+4+3+2+1=15$ partidas no total.
	      Portanto, caso não houvessem empates na partida, o total de pontos seria
	      $15\cdot3=45$ pontos. Mas pode-se ver que foram distribuídos apenas $10+
		      9+6+6+4+2=37$ pontos no total. Ou seja, $45-37=8$ pontos ``sumiram''.
	      Já que apenas $2$ pontos são distribuídos a cada empate, houveram $(3-2)\cdot8=8$
	      empates.
	\item Sejam os times $A$, $B$, $C$, $D$, $E$ e $F$ aqueles que pontuaram, respectivamente,
	      $10$, $9$, $6$, $6$, $4$, $2$. Abaixo encontra-se a única possível configuração,
	      onde $V$ = vitória, $E$ = empate e $D$ = derrota.

	      \begin{itemize}
		      \item Time $A$: $3V$, $1E$, $1D$
		      \item Time $B$: $2V$, $3E$, $0D$
		      \item Time $C$: $1V$, $3E$, $1D$
		      \item Time $D$: $1V$, $3E$, $1D$
		      \item Time $E$: $0V$, $4E$, $1D$
		      \item Time $F$: $0V$, $2E$, $3D$
	      \end{itemize}

	      Nenhuma outra configuração é válida por não cumprir todas as seguintes
	      obrigações indispensáveis:

	      \begin{itemize}
		      \item todos os times devem jogar $5$ partidas,
		      \item devem haver $8$ empates entre $15$ partidas totais,
		      \item o time A deve ganhar $3$ partidas, empatar $1$ e perder $1$,
		      \item o time F deve empatar $2$ partidas e perder $3$,
		      \item o time E deve empatar $4$ partidas e perder $1$, já que, caso
		            ganhe $1$, empate $1$ e perca $3$, os times $B$, $C$ e $D$
		            teriam que ganhar, no total, $12$ pontos provenientes de empates,
		            uma tarefa claramente impossível, já que cada um desses times deve
		            ganhar, no máximo, $3$ pontos culminados de empate,
		      \item $B$, $C$ e $D$, devem ganhar, cada um, $3$ pontos provenientes de
		            empate.
	      \end{itemize}

	\item Sabe-se que o tanto de empates do grupo $1$: $A$, $E$ e $F$ é $1+4+2=7$, e o
	      tanto de empates do grupo $2$: $B$, $C$ e $D$ é $3+3+3=9$ empates.
	      É fato que a partida entre $E$ e $F$ terminou em empate. Portanto,
	      $A$ empatou,no máximo, com $1$ time do grupo $2$, $E$ empatou, no máximo,
	      com $3$ times do grupo $2$ e $F$ empatou, no máximo, com $1$ time
	      do grupo $2$. Logo, houverão, no máximo, $1+3+2=5$ empates entre os grupos
	      $1$ e $2$. Logo entre os times do grupo $2$, houveram, no mínimo, $(9-5)
		      \cdot \frac{1}{2}=2$ empates, já que cada partida empatada resulta em
	      dois times participantes de um empate.
\end{enumerate}

\clearpage

\subsection{Problema 2.}
\begin{problem}
Encontre todos os pares $a,b$ de números reais tais que o número
$\lfloor an + b \rfloor$ é um quadrado perfeito, para todo inteiro positivo
$n$.

Observação: $\lfloor x \rfloor$ é o maior inteiro menor ou igual a $x$.
Por exemplo, $\lfloor 4 \rfloor = 4$, $\lfloor -5 \rfloor
	= -5$, $\lfloor \pi \rfloor = 3$, $\lfloor \frac{1+\sqrt{5}}{2} \rfloor
	= 1$ e $\lfloor -2,7 \rfloor = -3$.

\end{problem}

Apenas os pares $(a,b)=(0,b)$, onde $b\in[m^2,m^2+1)$ e $m\in\ZZ$

Perceba, primeiramente, que para todo $n$, existe $m \in \ZZ$ tal que
$\lfloor an + b \rfloor = m^2$. Para isso, é trivial perceber que basta sempre
anular $n$ e fazer com que $b\in[m^2,m^2+1)$. Ou seja, $a=0$.

Perceba que essa é a única configuração possível, já que:
\begin{itemize}
	\item $a<0$: é impossível, pois, basta que $n > \frac{b}{\mid a\mid}$, resultando em
	      $\lfloor an + b \rfloor < 0$. Portanto, nunca será um quadrado perfeito.
	\item $a>0$: também é impossível. Visando contradição, seja $x^2_n \coloneqq \lfloor an + b \rfloor$ com $x_n \in \ZZ$.
	      Deve-se perceber que $x_{n+1} \ge x_n+1$ para infinitos $n$, já que,
	      caso contrário, a sequência ficaria constante a partir de algum momento.
	      Isso é impossível porque $a>0$. Podemos ver que
	      a diferença entre termos consecutivos nos mostra algo bem interessante:
	      \[
		      x^2_{n+1}-x^2_n = \lfloor an + a + b \rfloor - \lfloor an + b \rfloor.
	      \]
	      Como vimos anteriormente, $x_{n+1} \ge x_n+1$ para infinitos $n$.
	      Portanto,
	      \[
		      x^2_{n+1}-x^2_n \ge (x_n + 1)^2-x^2_n = 2x_n + 1.
	      \]
	      Agora, para finalizar, perceba que $\lfloor an + a + b \rfloor \le
		      an + a + b$ e $\lfloor an + b \rfloor > an + b - 1$. Portanto,
	      \[
		      \lfloor an + a + b \rfloor - \lfloor an + b \rfloor < an + a + b -
		      an - b + 1 = a + 1 \iff 2x_n+1 < a + 1.
	      \]
	      Contradição! É impossível que $2x_n+1$ seja cotado por $a+1$,
	      já que $x_n \to \infty$.
\end{itemize}
Portanto, nos resta apenas os pares $(a,b)=(0,b)$, onde $b\in[m^2,m^2+1)$ e $m\in\ZZ$

\begin{observation}
	O mais complicado foi realmente mostrar que $a>0$ é impossível. Para tanto,
	foi de extrema importância conhecer o seguinte padrão:
	geralmente, quando é dito ``válido para todos os $n$'' é bem
	importante comparar termos consecutivos. Para tanto, foi
	relativamente trivial perceber que $x_{n+1} \ge x_n+1$ para
	infinitos $n$.
\end{observation}

\begin{observation}
	Após comparar os termos consecutivos, existe outro padrão muito
	recorrente que deveríamos conhecer:
	\[
		\lfloor r + s \rfloor - \lfloor r \rfloor \le \lfloor s \rfloor + 1 =
		\lceil s \rceil.
	\]
	Após isso, ficou relativamente fácil, já que é trivial perceber que
	\[
		x^2_{n+1} - x^2_n < a +1,
	\]
	e relembrar que $x_{n+1} \ge x_n+1$, culminando em
	$x^2_{n+1}-x^2_n \ge (x_n + 1)^2-x^2_n = 2x_n + 1$.

\end{observation}


\clearpage

\subsection{Problema 3.}
\begin{problem}
Para cada inteiro positivo $x$, seja $\varphi(x)$ o número de inteiros
$1 \leq k \leq x$ que não possuem fatores primos em comum com $x$. Por
exemplo, $\varphi(12) = 4$, pois $1, 5, 7$ e $11$ são todos os
números naturais menores ou iguais a $12$ que não possuem fatores
primos em comum com $12$. Determine todos os inteiros positivos
$n$ tais que existem inteiros positivos distintos $a_1,a_2,\ldots
	,a_n$ de modo que o conjunto:
\[
	S = \{a_1, a_2, \ldots, a_n, \varphi(a_1), \varphi(a_2), \ldots, \varphi(a_n)\}
\]
possui exatamente $2n$ inteiros consecutivos (em alguma ordem).
\end{problem}

\clearpage

\subsection{Problema 4.}
\begin{problem}
Seja $ABC$ um triângulo acutângulo escaleno, e seja $O$ seu
circuncentro. Seja $K$ um ponto sobre o lado $BC$. A reta $OK$ intersecta
o circuncírculo do triângulo $BOC$ novamente em $M$. Seja $L$ o simétrico
de $K$ relativo a $AC$. Demonstre que os circuncírculos dos triângulos
$LCM$ e $ABC$ são tangentes se, e somente se, $AK \perp BC$.
\end{problem}
\clearpage

\section{Referências}\label{sec:references}
\end{document}
