\documentclass[12pt]{article}
\usepackage{../sty/articlept}

\olympiad{OMpD 2022 N2 Fase 2}

\title{\olympiad}
\author{Samuel de Araújo Brandão}
\date{\today}

\begin{document}
  \maketitle
  \preface

  \tableofcontents

  \clearpage

  \section{Problemas}
  \begin{enumerate}[label={\textbf{\arabic*.}}]
    \item Considere um tabuleiro quadriculado $6 \times 6$, formado de $36$ casinhas unitárias. Desejamos colocar 
      $6$ torres neste tabuleiro, uma torre em cada casinha, de modo que não haja duas torres numa mesma linha,
      nem duas torres numa mesma coluna. Note que, uma vez colocadas as torres dessa maneira, temos que, para toda 
      casinha onde não foi colocada uma torre, há uma torre na mesma linha que ela e uma torre na mesma coluna que ela.
      Diremos que tais torres são alinhadas com essa casinha. 

      Para cada uma dessas $30$ casas sem torres, pinte-a de verde se as duas torres alinhadas com essa mesma casinha 
      estão a uma mesma distância dela, e pinte-a de amarelo caso contrário. Por exemplo, ao colocarmos as $6$ torres 
      (T) como abaixo, temos:

      \begin{figure}[h]
        \centering
        \includegraphics[width=0.3\textwidth]{first.png}  
      \end{figure}

      \begin{enumerate}[label=({\alph*})]
        \item É possível colocarmos as torres de modo que haja $30$ casas verdes?
        \item É possível colocarmos as torres de modo que haja $30$ casas amarelas?
        \item É possível colocarmos as torres de modo que haja $15$ casas verdes e $15$ amarelas?
      \end{enumerate}
    \item Seja $ABCD$ um retângulo. O ponto $E$ está sobre o lado $\overline{AB}$ e o ponto $F$ está sobre o lado 
      $\overline{AD}$, de modo que $\angle FEC = \angle CEB$ e $\angle DFC = \angle CFE$. 

      Determine a medida do ângulo $\angle FCE$ e da razão $AD/AB$.
    \item Seja $N$ um inteiro positivo. Inicialmente, um número inteiro positivo $A$ está escrito no quadro.
      A cada passo, podemos realizar uma das duas operações seguintes com o número escrito no quadro:

      \begin{enumerate}[label=(\roman*)]
        \item Somar $N$ ao número escrito no quadro e trocar tal número pela soma obtida; 
        \item Se o número no quadro for maior do que $1$ e tiver pelo menos um algarismo $1$, então 
          podemos remover o algarismo $1$ desse número, e trocar o número escrito inicialmente por este 
          (com remoção de possíveis zeros à esquerda). 
      \end{enumerate}

      Por exemplo, se $N = 63$ e $A = 25$, podemos fazer a seguinte sequência de operações: 
      \[
        25 \to 88 \to 151 \to 51 \to 5
      \]

      E se $N = 143$ e $A = 2$, podemos fazer a seguinte sequência de operações: 
      \[
        2 \to 145 \to 288 \to 431 \to 574 \to 717 \to 860 \to 1003 \to 3
      \]

      Para quais valores de $N$ sempre é possível, não importando o valor inicial de $A$ na lousa, obter 
      o número $1$ na lousa, mediante um número finito de operações?

    \item Dizemos que uma sêxtupla de números reais positivos $(a_1, a_2, a_3, b_1, b_2, b_3)$ é \textit{phika}
      se $a_1 + a_2 + a_3 = b_1 + b_2 + b_3 = 1$.
      \begin{enumerate}[label=({\alph*})]
    \item Prove que existe uma sêxtupla phika $(a_1, a_2, a_3, b_1, b_2, b_3)$ tal que:
      \[
        a_1(\sqrt{b_1} + a_2) + a_2(\sqrt{b_2} + a_3) + a_3(\sqrt{b_3} + a_1) > 1 - \frac{1}{20222022}.
      \]

    \item Prove que para toda sêxtupla phika $(a_1, a_2, a_3, b_1, b_2, b_3)$, temos:
      \[
        a_1(\sqrt{b_1} + a_2) + a_2(\sqrt{b_2} + a_3) + a_3(\sqrt{b_3} + a_1) < 1.
      \]
      \end{enumerate}
  \end{enumerate}

  \clearpage

  \section{Soluções}
    \subsection{Problema 1.}

    \begin{problem}{}{}
      Considere um tabuleiro quadriculado $6 \times 6$, formado de $36$ casinhas unitárias. Desejamos colocar 
      $6$ torres neste tabuleiro, uma torre em cada casinha, de modo que não haja duas torres numa mesma linha,
      nem duas torres numa mesma coluna. Note que, uma vez colocadas as torres dessa maneira, temos que, para toda 
      casinha onde não foi colocada uma torre, há uma torre na mesma linha que ela e uma torre na mesma coluna que ela.
      Diremos que tais torres são alinhadas com essa casinha. 

      Para cada uma dessas $30$ casas sem torres, pinte-a de verde se as duas torres alinhadas com essa mesma casinha 
      estão a uma mesma distância dela, e pinte-a de amarelo caso contrário. Por exemplo, ao colocarmos as $6$ torres 
      (T) como abaixo, temos:

      \begin{center}
        \includegraphics[width=0.3\textwidth]{first.png}
      \end{center}

      \begin{enumerate}[label=({\alph*})]
        \item É possível colocarmos as torres de modo que haja $30$ casas verdes?
        \item É possível colocarmos as torres de modo que haja $30$ casas amarelas?
        \item É possível colocarmos as torres de modo que haja $15$ casas verdes e $15$ amarelas?
      \end{enumerate}
    \end{problem}

  \clearpage

  \section{Referências}\label{sec:references}
\end{document}
