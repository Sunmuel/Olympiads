\documentclass[11pt]{article}

% Specify how big is going to be the paper margins.
\usepackage[
  paperwidth=6in, paperheight=9in,
  top=0.5in,
  bottom=0.5in,
  inner=0.5in,
  outer=0.5in
]{geometry}

% amsmath: Add useful commands like aligh and gather.
% amsfonts: Add useful fonts like \mathbb{R}.
% amssymb: Add useful symbles like \therefore (needs amsfonts to work).
% amsthm: Add useful environments like \begin{proof}
\usepackage{amsmath, amsfonts, amssymb, amsthm}

% Add Latin Modern Fonts like Sans-serif and Roman.
\usepackage{lmodern}

% Enables enumeration of items.
\usepackage{enumitem}

% Enables adding images.
\usepackage{graphicx}

\begin{document}
\section{\textbf{\textsf{OMpD 2022}}}

\textbf{Problema 1.} Considere um tabuleiro quadriculado $6 \times 6$, formado de $36$ casinhas unitárias. Desejamos colocar $6$ torres neste tabuleiro, uma torre em cada casinha, de modo que não haja duas torres numa mesma linha, nem duas torres numa mesma coluna. Note que, uma vez colocadas as torres dessa maneira, temos que, para toda casinha onde não foi colocada uma torre, há uma torre na mesma linha que ela e uma torre na mesma coluna que ela. Diremos que tais torres são alinhadas com essa casinha. 

Para cada uma dessas $30$ casas sem torres, pinte-a de verde se as duas torres alinhadas com essa mesma casinha estão a uma mesma distância dela, e pinte-a de amarelo caso contrário. Por exemplo, ao colocarmos as $6$ torres (T) como abaixo, temos:

\begin{figure}[h]
  \centering
  \includegraphics[width=0.3\textwidth]{first.png}  
\end{figure}

\begin{itemize}
    \item[(a)] É possível colocarmos as torres de modo que haja $30$ casas verdes?
    \item[(b)] É possível colocarmos as torres de modo que haja $30$ casas amarelas?
    \item[(c)] É possível colocarmos as torres de modo que haja $15$ casas verdes e $15$ amarelas?
\end{itemize}

\textbf{Problema 2.} Seja $ABCD$ um retângulo. O ponto $E$ está sobre o lado $\overline{AB}$ e o ponto $F$ está sobre o lado $\overline{AD}$, de modo que $\angle FEC = \angle CEB$ e $\angle DFC = \angle CFE$. 

Determine a medida do ângulo $\angle FCE$ e da razão $AD/AB$.

\textbf{Problema 3.} Seja $N$ um inteiro positivo. Inicialmente, um número inteiro positivo $A$ está escrito no quadro. A cada passo, podemos realizar uma das duas operações seguintes com o número escrito no quadro: 

\begin{itemize}
    \item[(i)] Somar $N$ ao número escrito no quadro e trocar tal número pela soma obtida; 
    \item[(ii)] Se o número no quadro for maior do que $1$ e tiver pelo menos um algarismo $1$, então podemos remover o algarismo $1$ desse número, e trocar o número escrito inicialmente por este (com remoção de possíveis zeros à esquerda). 
\end{itemize}

Por exemplo, se $N = 63$ e $A = 25$, podemos fazer a seguinte sequência de operações: 
\[
25 \to 88 \to 151 \to 51 \to 5
\]

E se $N = 143$ e $A = 2$, podemos fazer a seguinte sequência de operações: 
\[
2 \to 145 \to 288 \to 431 \to 574 \to 717 \to 860 \to 1003 \to 3
\]

Para quais valores de $N$ sempre é possível, não importando o valor inicial de $A$ na lousa, obter o número $1$ na lousa, mediante um número finito de operações?

\textbf{Problema 4.} Dizemos que uma sêxtupla de números reais positivos $(a_1, a_2, a_3, b_1, b_2, b_3)$ é \textit{phika} se $a_1 + a_2 + a_3 = b_1 + b_2 + b_3 = 1$. 

\begin{itemize}
    \item[(a)] Prove que existe uma sêxtupla phika $(a_1, a_2, a_3, b_1, b_2, b_3)$ tal que:
    \[
    a_1(\sqrt{b_1} + a_2) + a_2(\sqrt{b_2} + a_3) + a_3(\sqrt{b_3} + a_1) > 1 - \frac{1}{20222022}.
    \]

    \item[(b)] Prove que para toda sêxtupla phika $(a_1, a_2, a_3, b_1, b_2, b_3)$, temos:
    \[
    a_1(\sqrt{b_1} + a_2) + a_2(\sqrt{b_2} + a_3) + a_3(\sqrt{b_3} + a_1) < 1.
    \]
\end{itemize}


\section{\textbf{\textsf{OMpD 2023}}}

\textbf{Problema 1.} Alguns amigos internautas formaram $6$ equipes de futebol, e decidiram realizar um torneio onde cada equipe enfrenta outra exatamente uma vez em uma partida. Em cada partida, quem vence ganha $3$ pontos, quem perde não ganha ponto, e se as duas equipes empatarem, cada uma leva $1$ ponto. 

Ao final do torneio, verificou-se que as pontuações das equipes foram $10, 9, 6, 6, 4$ e $2$ pontos. A respeito deste torneio, responda os seguintes itens, justificando sua resposta em cada um deles. 

\begin{itemize}
    \item[(a)] Quantas partidas terminaram empatadas no torneio? 
    \item[(b)] Determine, para cada um das $6$ equipes, o número de vitórias, empates e derrotas. 
    \item[(c)] Se considermos apenas as partidas disputadas entre a equipe que fez $9$ pontos contra as duas equipes que fizeram $6$ pontos, e a disputada entre as duas equipes que fizeram $6$ pontos, explique porque dentre essas três partidas, há pelo menos $2$ empates.
\end{itemize}

\textbf{Problema 2.} Encontre todos os pares $a,b$ de números reais tais que o número $\lfloor an + b \rfloor$ é um quadrado perfeito, para todo inteiro positivo $n$. 

Observação: $\lfloor x \rfloor$ é o maior inteiro menor ou igual a $x$. Por exemplo, $\lfloor 4 \rfloor = 4$, $\lfloor -5 \rfloor = -5$, $\lfloor \pi \rfloor = 3$, $\lfloor 1+ \frac{5}{2} \rfloor = 1$ e $\lfloor -2,7 \rfloor = -3$.

\textbf{Problema 3.} Para cada inteiro positivo $x$, seja $\varphi(x)$ o número de inteiros $1 \leq k \leq x$ que não possuem fatores primos em comum com $x$. Por exemplo, $\varphi(12) = 4$, pois $1, 5, 7$ e $11$ são todos os números naturais menores ou iguais a $12$ que não possuem fatores primos em comum com $12$. Determine todos os inteiros positivos $n$ tais que existem inteiros positivos distintos $a_1,a_2,\ldots,a_n$ de modo que o conjunto: 
\[
S = \{a_1, a_2, \ldots, a_n, \varphi(a_1), \varphi(a_2), \ldots, \varphi(a_n)\}
\]
possui exatamente $2n$ inteiros consecutivos (em alguma ordem). 

\textbf{Problema 4.} Seja $ABC$ um triângulo acutângulo escaleno, e seja $O$ seu circuncentro. Seja $K$ um ponto sobre o lado $BC$. A reta $OK$ intersecta o circuncírculo do triângulo $BOC$ novamente em $M$. Seja $L$ o simétrico de $K$ relativo a $AC$. Demonstre que os circuncírculos dos triângulos $LCM$ e $ABC$ são tangentes se, e somente se, $AK \perp BC$.


\section{\textbf{\textsf{OMpD 2024}}}

\textbf{Problema 1.} Sejam $O, M, P, D$ algarismos distintos entre si, e distintos de zero, tais que $O < M < P < D$ e a seguinte equação é verdadeira: 
\[
OMP D \times (OM - D) = MDDMP - OM.
\]

\begin{itemize}
    \item[(a)] Usando estimativas, explique por que é impossível que o valor de $O$ seja maior do que ou igual a $3$. 
    \item[(b)] Explique por que $O$ não pode ser igual a $1$. 
    \item[(c)] É possível termos $M$ maior do que ou igual a $5$? Justifique. 
    \item[(d)] Determine os valores de $M, P$ e $D$. 
\end{itemize}

Observação: $X_1X_2 \ldots X_n$ é o número obtido da justaposição dos algarismos $X_1, X_2, \ldots, X_n$, nessa ordem. Por exemplo, se $n = 4, X_1 = 2, X_2 = 0, X_3 = 2$ e $X_4 = 2$, então $X_1X_2 \ldots X_n = 2024$.

\textbf{Problema 2.} Sejam $ABCD$ um quadrilátero convexo, $M, N$ e $P$ os pontos médios das diagonais $AC$ e $BD$ e do lado $AD$, respectivamente. Suponha também que $\angle ABC + \angle DCB = 90^\circ$ e que $AB = 6, CD = 8$. Calcule o perímetro do triângulo $MNP$.

\textbf{Problema 3.} Uma barata tonta está inicialmente no vértice $A$ de um cubão $ABCDEFGH$ de aresta medindo $1$ metro, conforme a figura ao lado. A cada segundo, a barata anda $1$ metro, sempre escolhendo ir para um dos três vértices adjacentes ao vértice que ela está. Por exemplo, após $1$ segundo a barata pode parar no vértice $B, D$ ou $E$. 

\begin{figure}[h]
  \centering
  \includegraphics[width=0.25\textwidth]{second.png}
\end{figure}

\begin{itemize}
    \item[(a)] De quantas maneiras a barata pode parar no vértice $G$ após $3$ segundos? 
    \item[(b)] É possível a barata parar no vértice $A$ após exatamente $2023$ segundos? 
    \item[(c)] De quantas maneiras a barata pode parar em $A$ após exatamente $2024$ segundos? 
\end{itemize}

Observação: uma maneira de a barata parar em um vértice após uma quantidade de segundos difere da outra maneira se, em algum segundo, a barata estaria em vértices distintos na trajetória. Dessa forma, há $2$ maneiras de a barata parar em $C$ após $2$ segundos: uma delas passa por $A, B, C$ e a outra passa por $A, D, C$.

\textbf{Problema 4.} Seja $a_0, a_1, a_2, \ldots$ uma sequência infinita de inteiros positivos com as seguintes propriedades: 

\begin{itemize}
    \item $a_0$ é um inteiro positivo dado; 
    \item Para cada $n \geq 1$ inteiro, $a_n$ é o menor inteiro maior do que $a_{n-1}$ de tal modo que $a_n + a_{n-1}$ é um quadrado perfeito. 
\end{itemize}

Por exemplo, se $a_0 = 3$, então $a_1 = 6, a_2 = 10, a_3 = 15$ e assim por diante. 

\begin{itemize}
    \item[(a)] Seja $T$ o conjunto dos números da forma $a_k - a_\ell$, com $k \geq \ell \geq 0$ inteiros. Demonstre que, independentemente do valor de $a_0$, o número de inteiros positivos que não estão em $T$ é finito. 
    \item[(b)] Calcule, em função de $a_0$, a quantidade de inteiros positivos que não estão em $T$. 
\end{itemize}

\section{\textbf{\textsf{OBM 2017}}}

\textbf{Problema 1.} Os pontos $X$, $Y$ e $Z$ estão marcados nos lados $AB$, $BC$ e $AC$ do triângulo $ABC$, respectivamente. Os pontos $A'$, $B'$ e $C'$ estão nos lados $XZ$, $XY$ e $YZ$ do triângulo $XYZ$, respectivamente, de modo que
\[
\frac{AB}{A'B'}=\frac{AC}{A'C'}=\frac{BC}{B'C'}=2
\]
e $ABB'A'$, $BCC'B'$ e $ACC'A'$ são trapézios em que os lados do triângulo $ABC$ são bases.
\begin{itemize}
  \item[(a)] Determine a razão entre a área do trapézio $ABB'A'$ e a área do \\ triângulo $A'B'X$.
  \item[(b)] Determine a razão entre a área do triângulo $XYZ$ e a área do triângulo $ABC$.
\end{itemize}

\textbf{Problema 2.} Sabemos que o número real $C$ e números reais não-nulos $x$, $y$ e $z$, dois a dois distintos, satisfazem:
\[
x+\frac{y}{z}+\frac{z}{y}=y+\frac{z}{x}+\frac{x}{z}=z+\frac{x}{y}+\frac{y}{x}=C.
\]
\begin{itemize}
\item[(a)] Mostre que $C=-1$;
\item[(b)] Exiba pelo menos uma solução $(x,y,z)$ para a equação dada.
\end{itemize}

\textbf{Problema 3.} Seja $n>1$ um inteiro e considere um tabuleiro $n\times n$, em que algumas das $n^2$ casas foram pintadas de preto, e as restantes foram pintadas de branco. Prove que é possível escolhermos uma das $n^2$ casas do tabuleiro, de modo que, ao removermos completamente a linha e a coluna que a contém, haja um número diferente de casas pretas e de casas brancas, dentre as $(n-1)^2$ casas restantes.

\textbf{Problema 4.} Na Terra dos Impas, somente os algarismos ímpares são utilizados para contar e escrever números. Assim, em vez dos números $1,2,3,4,5,6,7,8,9,10,11,12,\ldots$ os Impas têm os números correspondentes $1,3,5,7,9,11,13,15,17,19,31,33,\ldots$ (note que os números dos Impas têm somente algarismos ímpares). Por exemplo, se uma criança tem $11$ anos, os Impas diriam que ela tem $31$ anos.
\begin{itemize}
\item[(a)] Como os Impas escrevem o nosso número $20$?
\item[(b)] Numa escola desse lugar, a professora escreveu no quadro-negro a continha de multiplicar abaixo. Se você fosse um aluno Impa, o que escreveria como resultado?
\[
13\times 5
\]
\item[(c)] Escreva, na linguagem dos Impas, o número que na nossa representação decimal é escrito como $2017$.
\end{itemize}

\textbf{Problema 5.} No triângulo $ABC$, com $AB\ne AC$, seja $I$ seu incentro. Os pontos $P$ e $Q$ são definidos como os pontos onde o circuncírculo do triângulo $BCI$ intersecta novamente as retas $AB$ e $AC$, respectivamente. Seja $D$ o ponto de interseção de $AI$ e $BC$.
\begin{itemize}
\item[(a)] Prove que $P$, $Q$ e $D$ são colineares;
\item[(b)] Sendo $T$, diferente de $P$, o ponto de encontro dos circuncírculos dos triângulos $PDB$ e $QDC$, prove que $T$ está no circuncírculo do triângulo $ABC$.
\end{itemize}
Observação: O Incentro de um triângulo é o ponto de interseção de suas bissetrizes internas e o Circuncírculo de um triângulo é a circunferência que passa pelos seus três vértices.

\textbf{Problema 6.} Demonstre que, para todo $n$ inteiro positivo, existem inteiros positivos $a$ e $b$, sem fatores primos em comum, de modo que $a^2+2017b^2$ possui mais de $n$ fatores primos distintos.

\section{\textbf{\textsf{OBM 2018}}}

\textbf{Problema 1.} Um cubo $n\times n\times n$ é formado por $n^3$ cubinhos unitários e tem, inicialmente, um cubinho vermelho em somente um de seus vértices. Numeramos esse cubinho com o número $1$. A cada dia a partir do dia $2$, os cubinhos vizinhos (cubinhos com faces comuns) a cubinhos vermelhos também ficam vermelhos e são numerados com o número do dia.

\begin{figure}[h]
  \centering
  \includegraphics[width=0.4\textwidth]{third.png}
\end{figure}

Por exemplo, o cubo $2\times 2\times 2$ acima, no primeiro dia, tem um cubinho vermelho com o número $1$, no segundo dia tem quatro cubinhos vermelhos, um com o número $1$ e três com o número $2$, no terceiro dia tem sete, um cubinho com o número $1$, três com o número $2$ e três com o número $3$, e somente no quarto dia terá todos os seus cubinhos na cor vermelha. Para representar a numeração final podemos usar $n$ tabuleiros representando cada uma $n$ das camadas do cubo vistas de frente. Por exemplo, para o cubo $2\times 2\times 2$ acima temos as seguintes camadas:

\clearpage

\begin{figure}[h]
  \centering
  \includegraphics[width=0.25\textwidth]{fourth.png}
\end{figure}

\begin{itemize}
\item[(a)] Na figura a seguir temos as quatro camadas do cubo $4\times 4\times 4$ e os cubos numerados com $1$ e $2$. Copie esses $4$ tabuleiros no caderno de respostas e preencha os números de cada cubinho.

\begin{figure}[h]
  \centering
  \includegraphics[width=0.7\textwidth]{fifth.png}
\end{figure}

\item[(b)] Em um cubo $10\times 10\times 10$, quantos cubinhos são numerados com $7$? E quantos são numerados com $13$?
\item[(c)] Em um cubo $2018\times 2018\times 2018$, qual o número que aparece mais vezes na numeração dos cubinhos? (Se houver mais de um número que aparece o maior número de vezes liste todos.)
\end{itemize}

\textbf{Problema 2.} Uma quádrupla $(A,B,C,D)$ é dita dobarulho quando $A$, $B$ e $C$ são algarismos não nulos e $D$ é um inteiro positivo tais que:
\begin{enumerate}
\item $A\le 8$.
\item $D>1$.
\item $D$ divide os seis números de três algarismos $ABC$, $BCA$, $CAB$, $(A+1)CB$, $CB(A+1)$ e $B(A+1)C$.
\end{enumerate}
Determine todas as quádruplas dobarulho.

Observação: Estamos usando uma barra para distinguir a representação decimal do número de três algarismos $ABC$ do produto $A\cdot B\cdot C$. Por exemplo, se $ABC=126$, então $A=1$, $B=2$ e $C=6$.

\textbf{Problema 3.} Seja $ABC$ um triângulo acutângulo de circuncentro $O$ e ortocentro $H$. A circunferência de centro $X_A$ passa pelos pontos $A$ e $H$ e tangencia o circuncírculo do triângulo $ABC$. Defina de maneira análoga os pontos $X_B$ e $X_C$. Sejam $O_A$, $O_B$ e $O_C$ os simétricos de $O$ em relação aos lados $BC$, $CA$ e $AB$, respectivamente. Prove que as retas $O_AX_A$, $O_BX_B$ e $O_CX_C$ são concorrentes.

\textbf{Problema 4.}
\begin{itemize}
\item[(a)] Num triângulo $XYZ$, o incírculo tangencia os lados $XY$ e $XZ$ nos pontos $T$ e $W$, respectivamente. Prove que
\[
XT = XW = \frac{XY+XZ-YZ}{2}.
\]
Seja $ABC$ um triângulo e $D$ o pé da altura relativa ao lado $A$. Sejam $I$ e $J$ os incentros dos triângulos $ABD$ e $ACD$, respectivamente. Os incírculos de $ABD$ e $ACD$ tangenciam $AD$ nos pontos $M$ e $N$, respectivamente. Seja $P$ o ponto de tangência do incírculo inscrito de $ABC$ com o lado $AB$. O círculo de centro $A$ e raio $AP$ intersecta a altura $AD$ em $K$.
\item[(b)] Mostre que os triângulos $IMK$ e $KNJ$ são congruentes.
\item[(c)] Mostre que o quadrilátero $IDJK$ é inscritível.
\end{itemize}

\textbf{Problema 5.} Numa lousa estão escritos inicialmente os números \\ $1,2,\ldots,10$. Para quaisquer dois números $a$ e $b$ na lousa chamamos de $S_{a,b}$ a soma de todos os números na lousa com exceção de $a$ e $b$. Uma operação permitida é escolher dois números $a$ e $b$ na lousa, apagá-los e escrever o número $a+b+\frac{ab}{S_{a,b}}$. Após realizar essa operação algumas vezes restam na lousa apenas dois números $x$ e $y$, com $x\ge y$.
\begin{itemize}
\item[(a)] Quantas operações foram realizadas?
\item[(b)] Determine o maior valor possível para $x$.
\end{itemize}

\textbf{Problema 6.} Para todo inteiro positivo $n$ definimos $s(n)$ como a soma dos dígitos de $n$. Determine todos os pares $(a,b)$ de inteiros positivos para os quais
\[
\frac{s(an+b)-s(n)}{\,}
\]
assume um número finito de valores ao variar $n$ nos inteiros positivos.

\section{\textbf{\textsf{OBM 2019}}}

\textbf{Problema 1.} Um número de oito dígitos é dito robusto se cumprir ambas condições a seguir:
\begin{itemize}
\item[(i)] Nenhum dos seus algarismos é $0$.
\item[(ii)] A diferença entre dois algarismos consecutivos é $4$ ou $5$.
\end{itemize}
Responda às perguntas a seguir:
\begin{itemize}
\item[(a)] Quantos são os números robustos?
\item[(b)] Um número robusto é dito super-robusto se todos os seus algarismos são distintos. Calcule a soma de todos os números super-robustos.
\end{itemize}

\textbf{Problema 2.} Sejam $a$, $b$ e $k$ inteiros positivos com $k>1$ tais que
\[
\mathrm{mmc}(a,b)+\mathrm{mdc}(a,b)=k(a+b).
\]
Prove que $a+b\ge 4k$.

\textbf{Problema 3.} Seja $ABC$ um triângulo acutângulo inscrito em um círculo $\Gamma$ de centro $O$. Seja $D$ o pé da altura relativa ao vértice $A$. Sejam $E$ e $F$ pontos sobre $\Gamma$ tais que $AE=AD=AF$. Sejam $P$ e $Q$ os pontos de interseção da reta $EF$ com os lados $AB$ e $AC$, respectivamente. Seja $X$ o segundo ponto de interseção de $\Gamma$ com o círculo circunscrito ao triângulo $APQ$. Mostre que as retas $XD$ e $AO$ encontram-se em um ponto que está sobre $\Gamma$.

\textbf{Problema 4.} Seja $ABC$ um triângulo acutângulo e $D$ um ponto qualquer sobre o lado $BC$. Seja $E$ o simétrico de $D$ em relação a $AC$ e seja $F$ o simétrico de $D$ em relação a $AB$. A reta $ED$ intersecta a reta $AB$ em $G$, enquanto a reta $FD$ intersecta a reta $AC$ em $H$. Prove que os pontos $A$, $E$, $F$, $G$ e $H$ estão sobre uma mesma circunferência.

\textbf{Problema 5.} Na figura abaixo, um quadradinho branco é cercado por quatro quadradinhos pretos e três quadradinhos brancos são cercados por sete quadradinhos pretos.

\begin{figure}[h]
  \centering
  \includegraphics[width=0.2\textwidth]{sixth.png}
\end{figure}

Qual o número máximo de quadradinhos brancos que podem ser cercados por $n$ quadradinhos pretos?

\textbf{Problema 6.} No plano cartesiano, todos os pontos com ambas coordenadas inteiras são pintados de azul. Dois pontos azuis são ditos mutuamente visíveis se o segmento de reta que os conecta não possui outros pontos azuis. Prove que existe um conjunto de $2019$ pontos azuis que são mutuamente visíveis dois a dois.

\section{\textbf{\textsf{OBM 2020}}}

\textbf{Problema 1.} Seja $ABC$ um triângulo acutângulo, e $D$ um ponto sobre $BC$ tal que $AD$ é perpendicular a $BC$. A bissetriz 
do ângulo $\angle DAC$ intesecta o segmento $DC$ em $E$. Seja $F$ o ponto sobre a reta $AE$ tal que $BF$ é perpendicular 
a $AE$. Se $\angle BAE = 45^\circ$, calcule a medida do ângulo $\angle BFC$.

\textbf{Problema 2.} Em uma lousa encontra-se o seguinte texto:

\[
  \text{A equação } x^2 - 824x + \_\_\_143 = 0 \text{ possui duas soluções inteiras.}
\]


Onde \_\_\_ representa alguma quantidade de algarismos de um número que está borrada na lousa.
Quais são as possíveis equações originalmente na lousa?

\textbf{Problema 3.} Consideremos uma sequência infinita $x_1, x_2, \ldots$ de \\ números inteiros positivos tais que, para todo inteiro $n \ge 1$:
\begin{itemize}
\item Se $x_n$ é par, então $x_{n+1} = \frac{x_n}{2}$;
\item Se $x_n$ é ímpar, então $x_{n+1} = \frac{x_n-1}{2}+2^{k-1}$, onde $k$ é o inteiro tal que $2^{k-1}\le x_n<2^k$.
\end{itemize}
Determine o menor valor possível de $x_1$ para o qual a sequência contenha algum termo igual a $2020$.

\textbf{Problema 4.} Um número inteiro positivo é dito cilíndrico se o primeiro algarismo e o último algarismo de sua representação decimal são iguais. Por exemplo, $4$ e $4104$ são cilíndricos, pois os seus primeiros e últimos algarismos são $4$, mas $10$ não é cilíndrico, pois o seu primeiro algarismo é $1$, enquanto o seu último algarismo é $0$.
Um número cilíndrico é dito supercilíndrico se pode ser escrito como a soma de dois números cilíndricos. Por exemplo $101 = 99 + 2$ e $22 = 11 + 11$ são supercilíndricos, mas $561 = 484 + 77$ não é supercilíndrico, pois não é cilíndrico.
Quantos números de $4$ algarismos são supercilíndricos?

\textbf{Problema 5.} Seja $ABC$ um triângulo acutângulo de circuncentro $O$. Seja $M$ o ponto médio de $AB$ e $K , C$ o segundo ponto de interseção dos circuncírculos dos triângulos $ABC$ e $CMO$. As retas $CK$ e $OM$ encontram-se em $P$. Prove que $\angle KAP = \angle MCB$.

\textbf{Problema 6.} Seja $k$ um número inteiro positivo. Arnaldo e Bernaldo jogam um jogo em um tabuleiro $2020\times 2020$. Inicialmente todas as casas do tabuleiro estão vazias. Uma jogada consiste em escolher uma casa vazia e colocar nesta uma ficha azul ou uma ficha vermelha.
Arnaldo vence o jogo se em algum momento existirem $k$ casas consecutivas em uma mesma linha ou coluna preenchidas com fichas de uma mesma cor. Bernaldo vence se todo o tabuleiro é preenchido sem que Arnaldo vença. Arnaldo é o primeiro a jogar e, a partir de então, cada jogador joga alternadamente.
Quais são os valores de $k$ para os quais Arnaldo tem uma estratégia vencedora?



\section{\textbf{\textsf{OBM 2021}}}

\textbf{Problema 1.} Um matemático está se divertindo com números naturais e seus divisores positivos. Para cada inteiro positivo $n\ge 3$, ele faz a seguinte sequência de operações: calcula a quantidade de divisores positivos de $n$, depois ele toma o número obtido e calcula a quantidade de divisores positivos e assim sucessivamente, até obter o número $2$, quando ele finalmente para. A \textit{lonjura} de $n$ é definida como a quantidade de operações necessárias para se obter o número $2$. Note que sempre se chega em $2$. Por exemplo, a lonjura de $12$ é $4$, pois $12$ tem $6$ divisores positivos $\{1,2,3,4,6,12\}$, o $6$ tem $4$ divisores positivos $\{1,2,3,6\}$, o $4$ tem $3$ divisores positivos $\{1,2,4\}$ e o $3$ tem $2$ divisores positivos $\{1,3\}$. Note também que $6$ tem lonjura $3$, $4$ tem lonjura $2$, e $3$ tem lonjura $1$.
\begin{itemize}
\item[(a)] Quantos números de $3$ a $1000$ possuem lonjura $2$?
\item[(b)] Qual é a maior lonjura possível dentre os números de $3$ a $1000$?
\end{itemize}

\textbf{Problema 2.} Seja $ABC$ um triângulo acutângulo. Defina $A_1$ como o ponto médio do maior arco $BC$ do circuncírculo de $ABC$. Sejam $A_2$ e $A_3$ os pés das perpendiculares de $A_1$ até as retas $AB$ e $AC$, respectivamente. Defina $B_2,B_3,C_2$ e $C_3$ de modo análogo.
\begin{itemize}
\item[(a)] Prove $A_2A_3$ intersecta $BC$ em seu ponto médio.
\item[(b)] Mostre que as retas $A_2A_3$, $B_2B_3$ e $C_2C_3$ são concorrentes.
\end{itemize}

\textbf{Problema 3.} Em um campeonato de futebol com $2021$ times, todos jogam contra todos exatamente uma vez. Ao final de cada partida, em caso de empate, cada time ganha $1$ ponto e, caso contrário, o vencedor da partida ganha $3$ pontos e o perdedor não ganha e nem perde ponto. Ao final do campeonato, os dois times com as maiores pontuações disputam uma final. O OBM Futebol Clube venceu a sua primeira partida e sabe-se que, por terem vencido o campeonato anterior, eles levam vantagem em qualquer caso de empate na soma de pontos final. Qual é a pontuação final mínima para que o OBM Futebol Clube tenha alguma chance de ir para a final?

\textbf{Problema 4.} Na figura a seguir temos um triângulo $ABC$, retângulo em $B$, e $BDEF$ é um quadrado inscrito nesse triângulo. As circunferências inscritas nos triângulos $CFE$ e $EDA$ possuem raios $c$ e $b$, respectivamente. No interior do quadrado desenham-se duas circunferências de raio $b$ e duas circunferências de raio $a$, cada uma tangente a um dos lados do quadrado, como mostra a figura. Sabe-se que as circunferências de raio $b$ no interior do quadrado são tangentes entre si e tangentes às duas de raio $a$. Lembre-se que quando temos duas circunferências tangentes, o centro é colinear com o ponto de tangência. Determine a razão $\frac{c}{a}$.

\begin{figure}[h]
  \centering
  \includegraphics[width=0.55\textwidth]{seventh.png}
\end{figure}

\textbf{Problema 5.} Uma tripla de inteiros positivos $(a,b,c)$ é chamada miranha se
\begin{itemize}
\item $a$ divide $bc+1$;
\item $b$ divide $ca+1$;
\item $c$ divide $ab+1$.
\end{itemize}
Determine todas as triplas miranhas.

\textbf{Problema 6.} Seja $\alpha\ge 1$ um número real. Considere o conjunto
\[
A(\alpha)=\{\lfloor n\alpha\rfloor\mid n\ \text{inteiro positivo}\}=\{\lfloor\alpha\rfloor,\lfloor 2\alpha\rfloor,\lfloor 3\alpha\rfloor,\lfloor 4\alpha\rfloor,\ldots\}.
\]
Suponha que todos os inteiros positivos que não pertencem ao conjunto $A(\alpha)$ são exatamente os inteiros positivos que deixam um determinado resto $r$ na divisão por $2021$, com $0\le r<2021$. Determine todos os possíveis valores de $\alpha$.

\textit{Observação:} O símbolo $\lfloor x\rfloor$ é o maior inteiro menor ou igual a $x$. Por exemplo, se $\alpha=\sqrt{3}$, temos $\lfloor\sqrt{3}\rfloor=\lfloor 1,73\ldots\rfloor=1$, $\lfloor 2\sqrt{3}\rfloor=\lfloor 2\cdot 1,73\ldots\rfloor=3$, $\lfloor 3\sqrt{3}\rfloor=\lfloor 3\cdot 1,73\ldots\rfloor=5$, $\lfloor 4\sqrt{3}\rfloor=\lfloor 4\cdot 1,73\ldots\rfloor=6$ e assim por diante. Nesse caso temos $A(\alpha)=\{1,3,5,6,\ldots\}$.




\section{\textbf{\textsf{OBM 2022}}}

\textbf{Problema 1.} Um jogo para uma pessoa tem as seguintes regras: inicialmente há dez pilhas de pedras, com $1,2,3,\ldots,10$ pedras, respectivamente. Uma jogada consiste em fazer uma das duas seguintes operações:
\begin{itemize}
\item[(i)] escolher duas pilhas, cada uma com pelo menos duas pedras, juntá-las e depois adicionar mais duas pedras a nova pilha;
\item[(ii)] escolher uma pilha com pelo menos $4$ pedras, tirar duas pedras dela e separá-la em duas pilhas de quantidades de pedras positivas escolhidas pelo jogador.
\end{itemize}
O jogo continua até não ser mais possível fazer uma operação.
\begin{itemize}
\item[(a)] Dê um exemplo de uma sequência de jogadas que conduzem ao término do jogo.
\item[(b)] Faça uma tabela informando qual é o número total de pedras e o número de pilhas no início e após cada uma das $5$ primeiras operações no seu exemplo acima.
\item[(c)] Mostre que o número de pilhas com apenas uma pedra ao final do jogo é sempre o mesmo, independentemente de como se realizam as jogadas.
\end{itemize}

\textbf{Problema 2.} Os números reais $a,b,c$ são diferentes de zero e cumprem o seguinte sistema de equações:

\begin{align*}
& a+ab=c\\
& b+bc=a\\
& c+ca=b
\end{align*}

Determine os possíveis valores de $abc$.

\textbf{Problema 3.} Seja $ABC$ um triângulo com incentro $I$ e seja $\Gamma$ sua circunferência circunscrita. Sejam $M$ o ponto médio de $BC$, $K$ o ponto médio do arco $BC$ que não contém $A$, $L$ o ponto médio do arco $BC$ que contém $A$ e $J$ a reflexão de $I$ pela reta $KL$. A reta $LJ$ intersecta $\Gamma$ novamente no ponto $T\,( \ne L)$. A reta $TM$ intersecta $\Gamma$ novamente no ponto $S\,(\ne T)$. Prove que $S,I,M$ e $K$ estão sobre uma mesma circunferência.

\textbf{Problema 4.} Na figura, $PQ$ e $BC$ são paralelos, $AB=BC$ e $MB=MC$. Além disso, $\angle CQM=\angle MQP$ e $PQ$ é tangente à circunferência inscrita ao triângulo $ABC$.
Dado que $AQ=1$, calcule o perímetro do triângulo $ABC$.

\begin{figure}[h]
  \centering
  \includegraphics[width=0.31\textwidth]{eighth.png}
\end{figure}

\textbf{Problema 5.} Inicialmente um número está escrito no quadro. Então, a cada minuto, Esmeralda escolhe um divisor $d>1$ do número $n$ escrito na lousa, apaga $n$ e escreve $n+d$. Se o número inicial é $2022$, qual é o maior número composto que Esmeralda nunca poderá escrever no quadro?

\textbf{Problema 6.} Determine o maior inteiro positivo $k$ para o qual a seguinte afirmação é verdadeira: dados $k$ subconjuntos distintos do conjunto $\{1,2,3,\ldots,2023\}$, cada um com $1011$ elementos, é possível particionar os subconjuntos em duas coleções de forma que quaisquer dois subconjuntos em uma mesma coleção possuem algum elemento em comum.

\section{\textbf{\textsf{OBM 2023}}}

\textbf{Problema 1.} Um número inteiro positivo é dito vaivém quando, considerando a sua representação na base dez, o primeiro algarismo da esquerda para a direita é maior que o segundo, o segundo é menor que o terceiro, o terceiro é maior que o quarto e assim por diante alternando maior e menor até o último algarismo. Por exemplo, $2021$ é vaivém, pois $2>0$ e $0<2$ e $2>1$. O número $2023$ não é vaivém, pois $2>0$ e $0<2$, mas $2$ não é maior do que $3$.
\begin{itemize}
\item[(a)] Existem quantos inteiros positivos vaivéns de $2000$ até $2100$?
\item[(b)] Qual é o maior número vaivém sem algarismos repetidos?
\item[(c)] Quantos números de $7$ algarismos distintos formados pelos dígitos $1,2,3,4,5,6$ e $7$ são vaivéns? Por exemplo, $4253617$ é um destes números. Mas $5372146$ não é ($2$ é maior do que $1$ e $1$ é menor do que $4$) e $2163457$ também não é ($4$ é menor do que $5$).
\end{itemize}

\textbf{Problema 2.} Considere um triângulo $ABC$ com $AB<AC$ e sejam $H$ e $O$ seu ortocentro e circuncentro, respectivamente. Uma reta partindo de $B$ corta as retas $AO$ e $AH$ em $M$ e $M'$ de modo que $M'$ é ponto médio de $BM$. Outra reta partindo de $C$ corta as retas $AH$ e $AO$ em $N$ e $N'$ de modo que $N'$ é ponto médio de $CN$. Prove que $M, M', N, N'$ estão sobre uma mesma circunferência.

\textbf{Problema 3.} Seja $n$ um inteiro positivo. Mostre que existem inteiros $x_1,x_2,\ldots,x_n$, não todos iguais, satisfazendo
\[
\begin{cases}
x_1^2+x_2+x_3+\cdots+x_n=0\\
x_1+x_2^2+x_3+\cdots+x_n=0\\
x_1+x_2+x_3^2+\cdots+x_n=0\\
\quad\vdots\\
x_1+x_2+x_3+\cdots+x_n^2=0
\end{cases}
\]
se, e somente se, $2^n-1$ é composto.
\medskip

\textbf{Problema 4.} Determine o menor inteiro $k$ para o qual existem inteiros positivos $a,b$ e $c$, dois a dois distintos, tais que 

\[
  a^2=bc \text{\quad e \quad} k=2b+3c-a.
\]

\textbf{Problema 5.} Um inteiro $n\ge 3$ é fabuloso quando existe um inteiro $a$ com $2\le a\le n-1$ para o qual $a^n-a$ é divisível por $n$. Encontre todos os inteiros fabulosos.

\textbf{Problema 6.} Seja $m$ um inteiro positivo com $m\le 2024$. Ana e Banana jogam um jogo alternadamente em em um tabuleiro $1\times 2024$, com quadrados inicialmente pintados de branco. Ana começa o jogo. Cada jogada de Ana consiste em escolher $k\le m$ quadrados brancos quaisquer do tabuleiro e pintá-los todos de verde. Cada jogada de Banana consiste em escolher qualquer sequência de quadrados verdes consecutivos e pintá-los todos de branco. Qual é o menor valor de $m$ para o qual Ana consegue garantir que, após alguma de suas jogadas, o tabuleiro inteiro estará pintado de verde?

\section{\textbf{\textsf{OBM 2024}}}

\textbf{Problema 1.} Considere uma sequência cujo primeiro termo é um inteiro positivo dado $N>1$. Considere a fatoração de $N$ em primos. Se $N$ é uma potência de $2$, a sequência é formada por um único termo: $N$. Caso contrário, o segundo termo da sequência é obtido trocando o maior fator primo $p$ de $N$ por $p+1$ na fatoração em primos. Se o novo número não é uma potência de $2$, repetimos o mesmo procedimento com ele, lembrando de fatorá-lo novamente em primos. Caso contrário, a sequência numérica termina. E assim sucessivamente.

Por exemplo, se o primeiro termo da sequência é $N=300=2^2\cdot 3\cdot 5^2$, como o seu maior fator primo é $p=5$, o segundo termo é $2^2\cdot 3\cdot(5+1)^2=2^4\cdot 3^3$. Repetindo o procedimento, o maior fator primo do segundo termo é $p=3$ e então o terceiro termo é $2^4\cdot(3+1)^3=2^{10}$. Como obtivemos uma potência de $2$, a sequência tem $3$ termos: $2^2\cdot 3\cdot 5^2$, $2^4\cdot 3^3$ e $2^{10}$.

\begin{itemize}
\item[(a)] Quantos termos tem a sequência cujo primeiro termo é $N=2\cdot 3\cdot 5\cdot 7\cdot 11\cdot 13\cdot 17\cdot 19\cdot 23$?
\item[(b)] Mostre que se um fator primo $p$ deixa resto $1$ na divisão por $3$, então $\frac{p+1}{2}$ é um número inteiro que também deixa resto $1$ na divisão por $3$.
\item[(c)] Apresente um termo inicial $N$ menor do que $1.000.000$ (um milhão) tal que a sequência iniciada por $N$ tem exatamente $11$ termos.
\end{itemize}

\textbf{Problema 2.} Seja $ABC$ um triângulo escaleno. Sejam $E$ e $F$ os pontos médios dos lados $AC$ e $AB$, respectivamente, e seja $D$ um ponto qualquer no segmento $BC$. As circunferências circunscritas aos triângulos $BDF$ e $CDE$ intersectam a reta $EF$ em $K, F$ e $L, E$, respectivamente, e intersectam-se em $X, D$. O ponto $Y$ está sobre a reta $DX$ de modo que $AY$ é paralelo a $BC$. Prove que os pontos $K, L, X$ e $Y$ estão sobre uma mesma circunferência.

\textbf{Problema 3.} Os números de $1$ a $100$ são colocados sem repetição em cada casinha de um tabuleiro $10\times 10$. Um caminho crescente de tamanho $k$ nesse tabuleiro é uma sequência de casinhas $c_1,c_2,\ldots,c_k$ tal que, para cada $i=2,3,\ldots,k$, as seguintes propriedades são satisfeitas:
\begin{itemize}
\item as casinhas $c_i$ e $c_{i-1}$ compartilham um lado ou um vértice;
\item o número em $c_i$ é maior que o número em $c_{i-1}$.
\end{itemize}
Qual é o maior inteiro positivo $k$ para o qual sempre podemos encontrar um caminho crescente de tamanho $k$, independentemente de como os números de $1$ a $100$ estão dispostos no tabuleiro?

\textbf{Problema 4.} Um número é chamado trilegal se seus algarismos pertencem ao conjunto $\{1,2,3\}$ e se ele é divisível por $99$. Quantos são os números trilegais de $10$ algarismos?

\textbf{Problema 5.} Esmeralda escolhe dois inteiros positivos distintos $a$ e $b$, com $b>a$, e escreve a equação $x^2-ax+b=0$ no quadro. Se a equação possui raízes inteiras positivas distintas $c$ e $d$, com $d>c$, ela escreve a equação $x^2-cx+d=0$ no quadro. Ela repete o procedimento enquanto obtiver raízes inteiras positivas distintas. Caso ela escreva uma equação na qual isso não ocorre, ela para.
\begin{itemize}
\item[(a)] Mostre que Esmeralda pode escolher $a$ e $b$ de modo que ela vá escrever exatamente $2024$ equações no quadro.
\item[(b)] Qual é a maior quantidade de equações que ela pode escrever sabendo que um dos números escolhidos inicialmente é $2024$?
\end{itemize}

\textbf{Problema 6.} Seja $ABC$ um triângulo isósceles com $AB=BC$. Seja $D$ um ponto sobre o segmento $AB$, $E$ um ponto sobre o segmento $BC$ e $P$ um ponto sobre o segmento $DE$, tal que $AD=DP$ e $CE=PE$. Seja $M$ o ponto médio de $DE$. A reta paralela a $AB$ por $M$ intersecta $AC$ em $X$ e a reta paralela a $BC$ por $M$ intersecta $AC$ em $Y$. As retas $DX$ e $EY$ intersectam-se em $F$. Prove que $FP$ é perpendicular a $DE$.


\end{document}
