\documentclass[12pt]{article}
\usepackage{../../sty/otis}

\unit{\textsf{\textbf{Elem Geo}}}

\title{\unit}
\author{Samuel de Araújo Brandão}
\date{\today}

\begin{document}
\maketitle
\preface

\tableofcontents

\clearpage

\section{\textsf{Practice Problems}}
\begin{itemize}
  \item [\texttt{\textcolor{color7}{ZABF24E3}}] \textcolor{color3}{\textbf{(Monge's Theorem)}} Let $\omega_1$, $\omega_2$, $\omega_3$ be three pairwise incongruent circles, and let $X_{12}$, $X_{23}$, $X_{31}$ be the pairwise exsimilicenters. Show that $X_{12}$, $X_{23}$, $X_{31}$ are collinear.
	\item [\texttt{\textcolor{color7}{21USEMO4}}] \href{https://artofproblemsolving.com/community/c6h2707599p23524100}{\textbf{(USEMO 2021 P4)}} \quad Let $ABC$ be a triangle with circumcircle $\omega$, and let $X$ be the reflection of $A$ in $B$. Line $CX$ meets $\omega$ again at $D$. Lines $BD$ and $AC$ meet at $E$, and lines $AD$ and $BC$ meet at $F$. Let $M$ and $N$ denote the midpoints of $AB$ and $ AC$.
	      Can line $EF$ share a point with the circumcircle of triangle $AMN?$
	\item [\texttt{\textcolor{color7}{18JMO3}}] \href{https://artofproblemsolving.com/community/c5h1629606p10226149}{\textbf{(USAJMO 2018 P3)}} \quad Let $ABCD$ be a quadrilateral inscribed in circle $\omega$ with $\overline{AC} \perp \overline{BD}$. Let $E$ and $F$ be the reflections of $D$ over lines $BA$ and $BC$, respectively, and let $P$ be the intersection of lines $BD$ and $EF$. Suppose that the circumcircle of $\triangle EPD$ meets $\omega$ at $D$ and $Q$, and the circumcircle of $\triangle FPD$ meets $\omega$ at $D$ and $R$. Show that $EQ = FR$.
	\item [\texttt{\textcolor{color7}{17TSTST5}}] \href{https://artofproblemsolving.com/community/c6h1470025p8526136}{\textbf{(USA TSTST 2017 P5)}} \quad Let $ABC$ be a triangle with incenter $I$. Let $D$ be a point on side $BC$ and let $\omega_B$ and $\omega_C$ be the incircles of $\triangle ABD$ and $\triangle ACD$, respectively. Suppose that $\omega_B$ and $\omega_C$ are tangent to segment $BC$ at points $E$ and $F$, respectively. Let $P$ be the intersection of segment $AD$ with the line joining the centers of $\omega_B$ and $\omega_C$. Let $X$ be the intersection point of lines $BI$ and $CP$ and let $Y$ be the intersection point of lines $CI$ and $BP$. Prove that lines $EX$ and $FY$ meet on the incircle of $\triangle ABC$.
	\item [\texttt{\textcolor{color7}{16TSTST2}}] \href{https://artofproblemsolving.com/community/c6h1264174p6575204}{\textbf{(USA TSTST 2016 P2)}} \quad Let $ABC$ be a scalene triangle with orthocenter $H$ and circumcenter $O$. Denote by $M$, $N$ the midpoints of $\overline{AH}$, $\overline{BC}$. Suppose the circle $\gamma$ with diameter $\overline{AH}$ meets the circumcircle of $ABC$ at $G \neq A$, and meets line $AN$ at a point $Q \neq A$. The tangent to $\gamma$ at $G$ meets line $OM$ at $P$. Show that the circumcircles of $\triangle GNQ$ and $\triangle MBC$ intersect at a point $T$ on $\overline{PN}$.
  \item [\texttt{\textcolor{color6}{22USEMO4}}] \textcolor{color6}{[9\clubsuit]} \href{https://artofproblemsolving.com/community/c6h2946605p26379724}{\textbf{(USEMO 2022 P4)}} \quad Let $ABCD$ be a cyclic quadrilateral whose opposite sides are not parallel. Suppose points $P, Q, R, S$ lie in the interiors of segments $AB, BC, CD, DA,$ respectively, such that$$\angle PDA = \angle PCB, \text{ } \angle QAB = \angle QDC, \text{ } \angle RBC = \angle RAD, \text{ and } \angle SCD = \angle SBA.$$Let $AQ$ intersect $BS$ at $X$, and $DQ$ intersect $CS$ at $Y$. Prove that lines $PR$ and $XY$ are either parallel or coincide.
	\item [\texttt{\textcolor{color6}{23TSTST1}}] \textcolor{color6}{[5\clubsuit]} \href{https://artofproblemsolving.com/community/c6h2946605p26379724}{\textbf{(USA TSTST 2023 P1)}} \quad Let $ABC$ be a triangle with centroid $G$. Points $R$ and $S$ are chosen on rays $GB$ and $GC$, respectively, such that
	      \[ \angle ABS=\angle ACR=180^\circ-\angle BGC.\]Prove that $\angle RAS+\angle BAC=\angle BGC$.
	\item [\texttt{\textcolor{color6}{16EGMO4}}] \textcolor{color6}{[5\clubsuit]} \href{https://artofproblemsolving.com/community/c6h1227238p6177803}{\textbf{(EGMO 2016 P4)}} \quad Two circles $\omega_1$ and $\omega_2$, of equal radius intersect at different points $X_1$ and $X_2$. Consider a circle $\omega$ externally tangent to $\omega_1$ at $T_1$ and internally tangent to $\omega_2$ at point $T_2$. Prove that lines $X_1T_1$ and $X_2T_2$ intersect at a point lying on $\omega$.
	\item [\texttt{\textcolor{color6}{11IRNTST1}}] \textcolor{color6}{[5\clubsuit]} \href{https://artofproblemsolving.com/community/c6h405937p2266382}{\textbf{(Iran TST 2011 P1)}} \quad In acute triangle $ABC$ angle $B$ is greater than$C$. Let $M$ is midpoint of $BC$. $D$ and $E$ are the feet of the altitude from $C$ and $B$ respectively. $K$ and $L$ are midpoint of $ME$ and $MD$ respectively. If $KL$ intersect the line through $A$ parallel to $BC$ in $T$, prove that $TA=TM$.
	\item [\texttt{\textcolor{color6}{06SLG2}}] \textcolor{color6}{[3\clubsuit]} \href{https://artofproblemsolving.com/community/p10632285}{\textbf{(Shortlist 2006 G2)}} \quad Let $ ABCD$ be a trapezoid with parallel sides $ AB > CD$. Points $ K$ and $ L$ lie on the line segments $ AB$ and $ CD$, respectively, so that $AK/KB=DL/LC$. Suppose that there are points $ P$ and $ Q$ on the line segment $ KL$ satisfying\[\angle{APB} = \angle{BCD}\qquad\text{and}\qquad \angle{CQD} = \angle{ABC}.\]Prove that the points $ P$, $ Q$, $ B$ and $ C$ are concyclic.
	\item [\texttt{\textcolor{color3}{17SLG3}}] \textcolor{color3}{[9\clubsuit]} \href{https://artofproblemsolving.com/community/c6h1671271p10632285}{\textbf{(Shortlist 2017 G3)}} \quad Let $O$ be the circumcenter of an acute triangle $ABC$. Line $OA$ intersects the altitudes of $ABC$ through $B$ and $C$ at $P$ and $Q$, respectively. The altitudes meet at $H$. Prove that the circumcenter of triangle $PQH$ lies on a median of triangle $ABC$.
	\item [\texttt{\textcolor{color3}{20ELMO4}}] \textcolor{color3}{[5\clubsuit]} \href{https://artofproblemsolving.com/community/c6h2210480p16724057}{\textbf{(ELMO 2020 P4)}} \quad Let acute scalene triangle $ABC$ have orthocenter $H$ and altitude $AD$ with $D$ on side $BC$. Let $M$ be the midpoint of side $BC$, and let $D'$ be the reflection of $D$ over $M$. Let $P$ be a point on line $D'H$ such that lines $AP$ and $BC$ are parallel, and let the circumcircles of $\triangle AHP$ and $\triangle BHC$ meet again at $G \neq H$. Prove that $\angle MHG = 90^\circ$.
	\item [\texttt{\textcolor{color3}{04IMO1}}] \textcolor{color3}{[3\clubsuit]} \href{https://artofproblemsolving.com/community/p99445}{\textbf{(IMO 2004 P1)}} \quad Let $ABC$ be an acute-angled triangle with $AB\neq AC$. The circle with diameter $BC$ intersects the sides $AB$ and $AC$ at $M$ and $N$ respectively. Denote by $O$ the midpoint of the side $BC$. The bisectors of the angles $\angle BAC$ and $\angle MON$ intersect at $R$. Prove that the circumcircles of the triangles $BMR$ and $CNR$ have a common point lying on the side $BC$.
	\item [\texttt{\textcolor{color3}{23AMO1}}] \textcolor{color3}{[3\clubsuit]} \href{https://artofproblemsolving.com/community/c5h3038296p27349297}{\textbf{(USAMO 2023 P1)}} \quad In an acute triangle $ABC$, let $M$ be the midpoint of $\overline{BC}$. Let $P$ be the foot of the perpendicular from $C$ to $AM$. Suppose that the circumcircle of triangle $ABP$ intersects line $BC$ at two distinct points $B$ and $Q$. Let $N$ be the midpoint of $\overline{AQ}$. Prove that $NB=NC$.
	\item [\texttt{\textcolor{color3}{99AMO6}}] \textcolor{color3}{[3\clubsuit]} \href{https://artofproblemsolving.com/community/c6h54506p340041}{\textbf{(USAMO 1999 P6)}} \quad Let $ABCD$ be an isosceles trapezoid with $AB \parallel CD$. The inscribed circle $\omega$ of triangle $BCD$ meets $CD$ at $E$. Let $F$ be a point on the (internal) angle bisector of $\angle DAC$ such that $EF \perp CD$. Let the circumscribed circle of triangle $ACF$ meet line $CD$ at $C$ and $G$. Prove that the triangle $AFG$ is isosceles.

\end{itemize}

\clearpage

\section{\textsf{Solutions}}

\subsection{Lecture Notes}

\subsubsection{USEMO 2021 P4}
\begin{problem}
Let $ABC$ be a triangle with circumcircle $\omega$, and let $X$ be the reflection of $A$ in $B$. Line $CX$ meets $\omega$ again at $D$. Lines $BD$ and $AC$ meet at $E$, and lines $AD$ and $BC$ meet at $F$. Let $M$ and $N$ denote the midpoints of $AB$ and $ AC$.
Can line $EF$ share a point with the circumcircle of triangle $AMN?$
\end{problem}

\clearpage

\subsubsection{USAJMO 2018 P3}
\begin{problem}
Let $ABCD$ be a quadrilateral inscribed in circle $\omega$ with $\overline{AC} \perp \overline{BD}$. Let $E$ and $F$ be the reflections of $D$ over lines $BA$ and $BC$, respectively, and let $P$ be the intersection of lines $BD$ and $EF$. Suppose that the circumcircle of $\triangle EPD$ meets $\omega$ at $D$ and $Q$, and the circumcircle of $\triangle FPD$ meets $\omega$ at $D$ and $R$. Show that $EQ = FR$.
\end{problem}

\clearpage

\subsubsection{USA TSTST 2017 P5}
\begin{problem}
Let $ABC$ be a triangle with incenter $I$. Let $D$ be a point on side $BC$ and let $\omega_B$ and $\omega_C$ be the incircles of $\triangle ABD$ and $\triangle ACD$, respectively. Suppose that $\omega_B$ and $\omega_C$ are tangent to segment $BC$ at points $E$ and $F$, respectively. Let $P$ be the intersection of segment $AD$ with the line joining the centers of $\omega_B$ and $\omega_C$. Let $X$ be the intersection point of lines $BI$ and $CP$ and let $Y$ be the intersection point of lines $CI$ and $BP$. Prove that lines $EX$ and $FY$ meet on the incircle of $\triangle ABC$.
\end{problem}

\clearpage

\subsubsection{USA TSTST 2016 P2}
\begin{problem}
Let $ABC$ be a scalene triangle with orthocenter $H$ and circumcenter $O$. Denote by $M$, $N$ the midpoints of $\overline{AH}$, $\overline{BC}$. Suppose the circle $\gamma$ with diameter $\overline{AH}$ meets the circumcircle of $ABC$ at $G \neq A$, and meets line $AN$ at a point $Q \neq A$. The tangent to $\gamma$ at $G$ meets line $OM$ at $P$. Show that the circumcircles of $\triangle GNQ$ and $\triangle MBC$ intersect at a point $T$ on $\overline{PN}$.
\end{problem}

\clearpage

\clearpage

\subsection{Mandatory}

\subsubsection{USEMO 2022 P4}
\begin{problem}
Let $ABCD$ be a cyclic quadrilateral whose opposite sides are not parallel. Suppose points $P, Q, R, S$ lie in the interiors of segments $AB, BC, CD, DA,$ respectively, such that$$\angle PDA = \angle PCB, \text{ } \angle QAB = \angle QDC, \text{ } \angle RBC = \angle RAD, \text{ and } \angle SCD = \angle SBA.$$Let $AQ$ intersect $BS$ at $X$, and $DQ$ intersect $CS$ at $Y$. Prove that lines $PR$ and $XY$ are either parallel or coincide.
\end{problem}

\clearpage

\subsubsection{USA TSTST 2023 P1}
\begin{problem}
Let $ABC$ be a triangle with centroid $G$. Points $R$ and $S$ are chosen on rays $GB$ and $GC$, respectively, such that
\[ \angle ABS=\angle ACR=180^\circ-\angle BGC.\]Prove that $\angle RAS+\angle BAC=\angle BGC$.
\end{problem}

\clearpage

\subsubsection{EGMO 2016 P4}
\begin{problem}
Two circles $\omega_1$ and $\omega_2$, of equal radius intersect at different points $X_1$ and $X_2$. Consider a circle $\omega$ externally tangent to $\omega_1$ at $T_1$ and internally tangent to $\omega_2$ at point $T_2$. Prove that lines $X_1T_1$ and $X_2T_2$ intersect at a point lying on $\omega$.
\end{problem}

\clearpage

\subsubsection{Iran TST 2011 P1}
\begin{problem}
In acute triangle $ABC$ angle $B$ is greater than$C$. Let $M$ is midpoint of $BC$. $D$ and $E$ are the feet of the altitude from $C$ and $B$ respectively. $K$ and $L$ are midpoint of $ME$ and $MD$ respectively. If $KL$ intersect the line through $A$ parallel to $BC$ in $T$, prove that $TA=TM$.
\end{problem}

\clearpage

\subsubsection{Shortlist 2006 G2}
\begin{problem}
Let $ ABCD$ be a trapezoid with parallel sides $ AB > CD$. Points $ K$ and $ L$ lie on the line segments $ AB$ and $ CD$, respectively, so that $AK/KB=DL/LC$. Suppose that there are points $ P$ and $ Q$ on the line segment $ KL$ satisfying\[\angle{APB} = \angle{BCD}\qquad\text{and}\qquad \angle{CQD} = \angle{ABC}.\]Prove that the points $ P$, $ Q$, $ B$ and $ C$ are concyclic.
\end{problem}

\clearpage

\subsection{Not mandatory}

\subsubsection{Shortlist 2017 G3}
\begin{problem}
Let $O$ be the circumcenter of an acute triangle $ABC$. Line $OA$ intersects the altitudes of $ABC$ through $B$ and $C$ at $P$ and $Q$, respectively. The altitudes meet at $H$. Prove that the circumcenter of triangle $PQH$ lies on a median of triangle $ABC$.
\end{problem}

\clearpage

\subsubsection{ELMO 2020 P4}
\begin{problem}
Let acute scalene triangle $ABC$ have orthocenter $H$ and altitude $AD$ with $D$ on side $BC$. Let $M$ be the midpoint of side $BC$, and let $D'$ be the reflection of $D$ over $M$. Let $P$ be a point on line $D'H$ such that lines $AP$ and $BC$ are parallel, and let the circumcircles of $\triangle AHP$ and $\triangle BHC$ meet again at $G \neq H$. Prove that $\angle MHG = 90^\circ$.
\end{problem}

\clearpage

\subsubsection{IMO 2004 P1}
\begin{problem}
Let $ABC$ be an acute-angled triangle with $AB\neq AC$. The circle with diameter $BC$ intersects the sides $AB$ and $AC$ at $M$ and $N$ respectively. Denote by $O$ the midpoint of the side $BC$. The bisectors of the angles $\angle BAC$ and $\angle MON$ intersect at $R$. Prove that the circumcircles of the triangles $BMR$ and $CNR$ have a common point lying on the side $BC$.
\end{problem}

\clearpage

\subsubsection{USAMO 2023 P1}
\begin{problem}
In an acute triangle $ABC$, let $M$ be the midpoint of $\overline{BC}$. Let $P$ be the foot of the perpendicular from $C$ to $AM$. Suppose that the circumcircle of triangle $ABP$ intersects line $BC$ at two distinct points $B$ and $Q$. Let $N$ be the midpoint of $\overline{AQ}$. Prove that $NB=NC$.
\end{problem}

\clearpage

\subsubsection{IMO 1997 P2}
\begin{problem}
It is known that $ \angle BAC$ is the smallest angle in the triangle $ ABC$. The points $ B$ and $ C$ divide the circumcircle of the triangle into two arcs. Let $ U$ be an interior point of the arc between $ B$ and $ C$ which does not contain $ A$. The perpendicular bisectors of $ AB$ and $ AC$ meet the line $ AU$ at $ V$ and $ W$, respectively. The lines $ BV$ and $ CW$ meet at $ T$.
Show that $ AU = TB + TC$.
\end{problem}

\clearpage

\section{\textsf{References}}\label{sec:references}
\end{document}
