\begin{lemma}{Three Tangents Lemma}{}
	Let $ABC$ be an acute triangle. Let $BE$ and $CF$ be altitudes of
	$\triangle{ABC}$, and denote by $M$ the midpoint of $BC$. Prove that $ME$, $MF$,
	and the line through $A$ parallel to $BC$ are all tangents to $(AEF)$.
\end{lemma}
\begin{centering}
	\definecolor{qqwuqq}{rgb}{0,0.39215686274509803,0}
\definecolor{xdxdff}{rgb}{0.49019607843137253,0.49019607843137253,1}
\definecolor{ffqqqq}{rgb}{1,0,0}
\definecolor{qqzzqq}{rgb}{0,0.6,0}
\definecolor{qqqqff}{rgb}{0,0,1}
\begin{tikzpicture}[scale=0.7,line cap=round,line join=round,>=triangle 45,x=1cm,y=1cm]
	\clip(-13.479734933638587,-0.27847742737593095) rectangle (6.128228152397923,10.49436817401107);
	\draw [shift={(-6,10)},linewidth=0.6pt/g,color=ffqqqq,fill=ffqqqq,fill opacity=0.1] (0,0) -- (-48.01278750418334:0.3460228779888796) arc (-48.01278750418334:0:0.3460228779888796) -- cycle;
	\draw [shift={(-6,3.6)},linewidth=0.6pt/g,color=ffqqqq,fill=ffqqqq,fill opacity=0.1] (0,0) -- (41.98721249581666:0.3460228779888796) arc (41.98721249581666:90:0.3460228779888796) -- cycle;
	\draw [shift={(-8.206896551724139,4.482758620689655)},linewidth=0.6pt/g,color=ffqqqq,fill=ffqqqq,fill opacity=0.1] (0,0) -- (20.18580300946485:0.3460228779888796) arc (20.18580300946485:68.19859051364818:0.3460228779888796) -- cycle;
	\draw [shift={(3,0)},linewidth=0.6pt/g,color=ffqqqq,fill=ffqqqq,fill opacity=0.1] (0,0) -- (131.98721249581666:0.3460228779888796) arc (131.98721249581666:180:0.3460228779888796) -- cycle;
	\draw [shift={(-2.81767955801105,6.464088397790055)},linewidth=0.6pt/g,color=qqwuqq,fill=qqwuqq,fill opacity=0.10000000149011612] (0,0) -- (-159.81419699053518:0.3460228779888796) arc (-159.81419699053518:-138.01278750418336:0.3460228779888796) -- cycle;
	\draw [shift={(-6,10)},linewidth=0.6pt/g,color=qqwuqq,fill=qqwuqq,fill opacity=0.10000000149011612] (0,0) -- (-111.80140948635183:0.3460228779888796) arc (-111.80140948635183:-90:0.3460228779888796) -- cycle;
	\draw [shift={(-2.81767955801105,6.464088397790055)},linewidth=0.6pt/g,color=qqqqff,fill=qqqqff,fill opacity=0.1] (0,0) -- (-138.01278750418336:0.3460228779888796) arc (-138.01278750418336:-96.02557500836669:0.3460228779888796) -- cycle;
	\draw [shift={(-10,0)},linewidth=0.6pt/g,color=qqqqff,fill=qqqqff,fill opacity=0.1] (0,0) -- (0:0.3460228779888796) arc (0:41.98721249581666:0.3460228779888796) -- cycle;
	\draw [shift={(-8.206896551724139,4.482758620689655)},linewidth=0.6pt/g,color=qqqqff,fill=qqqqff,fill opacity=0.1] (0,0) -- (-21.801409486351808:0.3460228779888796) arc (-21.801409486351808:20.185803009464852:0.3460228779888796) -- cycle;
	\draw [shift={(-6,10)},linewidth=0.6pt/g,color=qqqqff,fill=qqqqff,fill opacity=0.1] (0,0) -- (-90:0.3460228779888796) arc (-90:-48.012787504183336:0.3460228779888796) -- cycle;
	\draw [linewidth=0.6pt/g,color=qqqqff] (-6,10)-- (-10,0);
	\draw [linewidth=0.6pt/g,color=qqqqff] (-10,0)-- (3,0);
	\draw [linewidth=0.6pt/g,color=qqqqff] (3,0)-- (-6,10);
	\draw [linewidth=0.6pt/g,color=qqzzqq] (3,0)-- (-8.206896551724139,4.482758620689655);
	\draw [linewidth=0.6pt/g,color=qqzzqq] (-10,0)-- (-2.81767955801105,6.464088397790055);
	\draw [linewidth=0.6pt/g,color=ffqqqq] (-6,6.8) circle (3.2cm);
	\draw [linewidth=0.6pt/g,color=qqzzqq] (-3.5,0)-- (-8.206896551724139,4.482758620689655);
	\draw [linewidth=0.6pt/g,color=qqzzqq] (-3.5,0)-- (-2.81767955801105,6.464088397790055);
	\draw [linewidth=0.6pt/g,color=qqzzqq] (-3.2306889125105824,3.210632838489085) -- (-3.093044452758099,3.196103701070769);
	\draw [linewidth=0.6pt/g,color=qqzzqq] (-3.2246351052529496,3.2679846967192865) -- (-3.086990645500466,3.25345555930097);
	\draw [shift={(-6,10)},linewidth=0.6pt/g,color=ffqqqq] (-48.01278750418334:0.3460228779888796) arc (-48.01278750418334:0:0.3460228779888796);
	\draw[linewidth=0.6pt/g,color=ffqqqq] (-5.723419569145305,9.876821474219552) -- (-5.644396588901106,9.841627609710853);
	\draw [linewidth=0.6pt/g,color=ffqqqq] (-6,10)-- (-6,3.6);
	\draw [linewidth=0.6pt/g,color=ffqqqq] (-8.206896551724139,4.482758620689655)-- (-2.81767955801105,6.464088397790055);
	\draw [shift={(-6,3.6)},linewidth=0.6pt/g,color=ffqqqq] (41.98721249581666:0.3460228779888796) arc (41.98721249581666:90:0.3460228779888796);
	\draw[linewidth=0.6pt/g,color=ffqqqq] (-5.876821474219548,3.8765804308547) -- (-5.841627609710848,3.9556034110988993);
	\draw [shift={(-8.206896551724139,4.482758620689655)},linewidth=0.6pt/g,color=ffqqqq] (20.18580300946485:0.3460228779888796) arc (20.18580300946485:68.19859051364818:0.3460228779888796);
	\draw[linewidth=0.6pt/g,color=ffqqqq] (-7.98980876843795,4.69380974789134) -- (-7.927783687499039,4.754110069948964);
	\draw [shift={(3,0)},linewidth=0.6pt/g,color=ffqqqq] (131.98721249581666:0.3460228779888796) arc (131.98721249581666:180:0.3460228779888796);
	\draw[linewidth=0.6pt/g,color=ffqqqq] (2.723419569145305,0.12317852578044718) -- (2.644396588901106,0.15837239028914674);
	\draw [linewidth=0.6pt/g,color=qqqqff] (-3.5,0)-- (-10,0);
	\draw [linewidth=0.6pt/g,color=qqqqff] (-6.721164760167593,-0.06920457559777517) -- (-6.721164760167593,0.06920457559777517);
	\draw [linewidth=0.6pt/g,color=qqqqff] (-6.778835239832405,-0.06920457559777517) -- (-6.778835239832405,0.06920457559777517);
	\draw [linewidth=0.6pt/g,color=qqqqff] (-3.5,0)-- (3,0);
	\draw [linewidth=0.6pt/g,color=qqqqff] (-0.2788352398324071,0.06920457559777517) -- (-0.2788352398324071,-0.06920457559777517);
	\draw [linewidth=0.6pt/g,color=qqqqff] (-0.22116476016759382,0.06920457559777517) -- (-0.22116476016759382,-0.06920457559777517);
	\draw [shift={(-2.81767955801105,6.464088397790055)},linewidth=0.6pt/g,color=qqqqff] (-138.01278750418336:0.3460228779888796) arc (-138.01278750418336:-96.02557500836669:0.3460228779888796);
	\draw [shift={(-2.81767955801105,6.464088397790055)},linewidth=0.6pt/g,color=qqqqff] (-138.01278750418336:0.27105125442462236) arc (-138.01278750418336:-96.02557500836669:0.27105125442462236);
	\draw [shift={(-10,0)},linewidth=0.6pt/g,color=qqqqff] (0:0.3460228779888796) arc (0:41.98721249581666:0.3460228779888796);
	\draw [shift={(-10,0)},linewidth=0.6pt/g,color=qqqqff] (0:0.27105125442462236) arc (0:41.98721249581666:0.27105125442462236);
	\draw [shift={(-8.206896551724139,4.482758620689655)},linewidth=0.6pt/g,color=qqqqff] (-21.801409486351808:0.3460228779888796) arc (-21.801409486351808:20.185803009464852:0.3460228779888796);
	\draw [shift={(-8.206896551724139,4.482758620689655)},linewidth=0.6pt/g,color=qqqqff] (-21.801409486351808:0.27105125442462236) arc (-21.801409486351808:20.185803009464852:0.27105125442462236);
	\draw [shift={(-6,10)},linewidth=0.6pt/g,color=qqqqff] (-90:0.3460228779888796) arc (-90:-48.012787504183336:0.3460228779888796);
	\draw [shift={(-6,10)},linewidth=0.6pt/g,color=qqqqff] (-90:0.27105125442462236) arc (-90:-48.012787504183336:0.27105125442462236);
	\draw [linewidth=0.6pt/g] (-10,10)-- (3,10);
	\begin{scriptsize}
		\draw [fill=black] (-10,0) circle (2.5pt);
		\draw[color=black] (-10.250188072409044,0.20018755384201395) node {$B$};
		\draw [fill=black] (3,0) circle (2.5pt);
		\draw[color=black] (3.094760922028745,0.24632393757386406) node {$C$};
		\draw [fill=black] (-6,10) circle (2.5pt);
		\draw[color=black] (-5.913368001615087,10.246385111452375) node {$A$};
		\draw [fill=black] (-2.81767955801105,6.464088397790055) circle (2pt);
		\draw[color=black] (-2.6030824688548058,6.61314489256918) node {$E$};
		\draw [fill=black] (-8.206896551724139,4.482758620689655) circle (2pt);
		\draw[color=black] (-8.5,4.2) node {$F$};
		\draw [fill=black] (-3.5,0) circle (2pt);
		\draw[color=black] (-3.2374577451677515,0.2693921294397891) node {$M$};
		\draw [fill=xdxdff] (21.743909671462525,10) circle (2.5pt);
		\draw[color=xdxdff] (-13.433598549906737,10.627010277240139) node {$D$};
		\draw [fill=black] (-6,3.6) circle (2pt);
		\draw[color=black] (-5.9710384812799004,3.395132127272634) node {$O$};
	\end{scriptsize}
\end{tikzpicture}

\end{centering}

\clearpage

\subsubsection{USAJMO 2011 P5}
\begin{problem}
Points $A,B,C,D,E$ lie on a circle $\omega$ and point $P$ lies outside the circle. The given points are such that
\begin{enumerate}[label={(\roman*)}]
	\item lines $PB$ and $PD$ are tangent to $\omega$,
	\item$P, A, C$ are collinear, and
	\item$DE \parallel AC$.
\end{enumerate}
Prove that $BE$ bisects $AC$.
\end{problem}

\begin{figure}[h]
	\centering
	\definecolor{qqwuqq}{rgb}{0,0.39215686274509803,0}
\definecolor{qqqqff}{rgb}{0,0,1}
\definecolor{qqzzqq}{rgb}{0,0.6,0}
\definecolor{ffqqqq}{rgb}{1,0,0}
\begin{tikzpicture}[scale=0.3, line cap=round,line join=round,>=triangle 45,x=1cm,y=1cm]
	\clip(-25.667248042741424,-15.147668166464284) rectangle (35.26106183980339,24.20846010296341);
	\draw [shift={(4.2754025541365905,-2.931704608550804)},line width=2pt,color=qqwuqq,fill=qqwuqq,fill opacity=0.10000000149011612] (0,0) -- (-2.411229194988626:1.2319955546925359) arc (-2.411229194988626:55.56101069119638:1.2319955546925359) -- cycle;
	\draw [shift={(-14.84851292802123,-2.1264204255947226)},line width=2pt,color=qqwuqq,fill=qqwuqq,fill opacity=0.10000000149011612] (0,0) -- (-2.411229194988624:1.2319955546925359) arc (-2.411229194988624:55.56101069119638:1.2319955546925359) -- cycle;
	\draw [shift={(14.617184106211349,-3.367184106211349)},line width=2pt,color=qqwuqq,fill=qqwuqq,fill opacity=0.10000000149011612] (0,0) -- (77.02776011381499:1.2319955546925359) arc (77.02776011381499:135:1.2319955546925359) -- cycle;
	\draw [shift={(0,0)},line width=2pt,color=qqwuqq,fill=qqwuqq,fill opacity=0.10000000149011612] (0,0) -- (-12.972239886185005:1.2319955546925359) arc (-12.972239886185005:45:1.2319955546925359) -- cycle;
	\draw [line width=2pt,color=ffqqqq] (0,0) circle (15cm);
	\draw [line width=2pt,color=qqzzqq] (20,20)-- (-3.6848191951669653,-14.54036132628516);
	\draw [line width=2pt,color=qqzzqq] (7.809226192370489,2.2217881972069593) -- (7.678589613603229,2.684822293059809);
	\draw [line width=2pt,color=qqzzqq] (7.809226192370489,2.2217881972069593) -- (8.288226981183783,2.266785241004571);
	\draw [line width=2pt,color=qqzzqq] (8.157590402416519,2.7298193368574206) -- (8.02695382364926,3.1928534327102702);
	\draw [line width=2pt,color=qqzzqq] (8.157590402416519,2.7298193368574206) -- (8.636591191229812,2.7748163806550323);
	\draw [line width=2pt,color=qqqqff] (20,20)-- (-3.3671841062113494,14.617184106211349);
	\draw [line width=2pt,color=qqqqff] (20,20)-- (14.617184106211349,-3.367184106211349);
	\draw [line width=2pt,color=qqzzqq] (-3.3671841062113494,14.617184106211349)-- (-14.84851292802123,-2.1264204255947226);
	\draw [line width=2pt,color=qqzzqq] (-9.45621272716232,5.737350700657852) -- (-9.586849305929583,6.2003847965107015);
	\draw [line width=2pt,color=qqzzqq] (-9.45621272716232,5.737350700657852) -- (-8.977211938349027,5.782347744455464);
	\draw [line width=2pt,color=qqzzqq] (-9.107848517116288,6.245381840308314) -- (-9.23848509588355,6.708415936161163);
	\draw [line width=2pt,color=qqzzqq] (-9.107848517116288,6.245381840308314) -- (-8.628847728302995,6.290378884105926);
	\draw [line width=2pt,color=ffqqqq] (10,10) circle (14.142135623730951cm);
	\draw [line width=2pt,color=qqqqff] (0,0)-- (14.617184106211349,-3.367184106211349);
	\draw [line width=2pt,color=qqqqff] (0,0)-- (-3.3671841062113494,14.617184106211349);
	\draw [line width=2pt,color=qqqqff] (0,0)-- (4.2754025541365905,-2.931704608550804);
	\draw [line width=2pt,color=qqzzqq] (4.2754025541365905,-2.931704608550804)-- (-14.84851292802123,-2.1264204255947226);
	\draw [line width=2pt,color=qqqqff] (14.617184106211349,-3.367184106211349)-- (4.2754025541365905,-2.931704608550804);
	\draw [line width=2pt,color=qqqqff] (0,0)-- (20,20);
	\draw [line width=2pt,color=qqzzqq] (-3.3671841062113494,14.617184106211349)-- (14.617184106211349,-3.367184106211349);
	\draw [shift={(4.2754025541365905,-2.931704608550804)},line width=2pt,color=qqwuqq] (-2.411229194988626:1.2319955546925359) arc (-2.411229194988626:55.56101069119638:1.2319955546925359);
	\draw[line width=2pt,color=qqwuqq] (5.239508781318625,-2.4494445162236227) -- (5.514967703370632,-2.3116559184158523);
	\draw [shift={(-14.84851292802123,-2.1264204255947226)},line width=2pt,color=qqwuqq] (-2.411229194988624:1.2319955546925359) arc (-2.411229194988624:55.56101069119638:1.2319955546925359);
	\draw[line width=2pt,color=qqwuqq] (-13.884406700839195,-1.6441603332675432) -- (-13.608947778787185,-1.506371735459773);
	\draw [shift={(14.617184106211349,-3.367184106211349)},line width=2pt,color=qqwuqq] (77.02776011381499:1.2319955546925359) arc (77.02776011381499:135:1.2319955546925359);
	\draw[line width=2pt,color=qqwuqq] (14.319797086441744,-2.3310197493571168) -- (14.234829366507569,-2.0349727902559085);
	\draw [shift={(0,0)},line width=2pt,color=qqwuqq] (-12.972239886185005:1.2319955546925359) arc (-12.972239886185005:45:1.2319955546925359);
	\draw[line width=2pt,color=qqwuqq] (1.0361643568542362,0.29738701976960746) -- (1.3322113159554438,0.38235473970378236);
	\begin{scriptsize}
		\draw [fill=black] (0,0) circle (2.5pt);
		\draw[color=black] (-0.6721418387289163,-0.17892217694994317) node {$O$};
		\draw [fill=black] (20,20) circle (2.5pt);
		\draw[color=black] (20.641381257451958,20.806068771312958) node {$P$};
		\draw [fill=black] (12.235624303440147,8.676952109183548) circle (2pt);
		\draw[color=black] (11.278215041788684,8.93784492777484) node {$A$};
		\draw [fill=black] (-3.6848191951669653,-14.54036132628516) circle (2pt);
		\draw[color=black] (-4.491328058275778,-14.346871055914134) node {$C$};
		\draw [fill=black] (-3.3671841062113494,14.617184106211349) circle (2pt);
		\draw[color=black] (-3.9574633179090126,15.549554404624796) node {$D$};
		\draw [fill=black] (14.617184106211349,-3.367184106211349) circle (2pt);
		\draw[color=black] (15.138467779825296,-3.6285097300890503) node {$B$};
		\draw [fill=black] (-14.84851292802123,-2.1264204255947226) circle (2pt);
		\draw[color=black] (-15.620354568998353,-1.2055851391937251) node {$E$};
		\draw [fill=black] (4.2754025541365905,-2.931704608550804) circle (2pt);
		\draw[color=black] (4.5022394909797345,-3.5463766931095475) node {$M$};
	\end{scriptsize}
\end{tikzpicture}

\end{figure}

\clearpage

\subsubsection{Shortlist 2010 G1}

\begin{problem}
Let $ABC$ be an acute triangle with $D, E, F$ the feet of the altitudes
lying on $BC, CA, AB$ respectively. One of the intersection points of the
line $EF$ and the circumcircle is $P.$ The lines $BP$ and $DF$ meet at point
$Q.$ Prove that $AP = AQ$.
\end{problem}

\begin{figure}[h]
	\centering
	\definecolor{ffwwqq}{rgb}{1,0.4,0}
\definecolor{wwqqcc}{rgb}{0.4,0,0.8}
\definecolor{wwccqq}{rgb}{0.4,0.8,0}
\definecolor{qqwuqq}{rgb}{0,0.39215686274509803,0}
\definecolor{ffqqtt}{rgb}{1,0,0.2}
\definecolor{qqzzff}{rgb}{0,0.6,1}
\begin{tikzpicture}[scale=0.7,line cap=round,line join=round,>=triangle 45,x=1cm,y=1cm]
	\clip(-14,-5) rectangle (3.85066872014997,15.100995344430798);
	% \clip(-24.751826875499052,-5.393061030219024) rectangle (16.85066872014997,15.100995344430798);
	\fill[line width=1.2pt,color=qqzzff,fill=qqzzff,fill opacity=0.1] (-6,10) -- (-10,0) -- (3,0) -- cycle;
	\draw [line width=1.2pt,color=qqzzff] (-6,10)-- (-10,0);
	\draw [line width=1.2pt,color=qqzzff] (-10,0)-- (3,0);
	\draw [line width=1.2pt,color=qqzzff] (3,0)-- (-6,10);
	\draw [line width=1.2pt,color=ffqqtt] (-3.5,3.2) circle (7.244998274671981cm);
	\draw [line width=1.2pt,color=qqwuqq] (1.7846032699052694,8.156104143347527)-- (-10.736403612827731,3.5527927894015696);
	\draw [line width=1.2pt,color=wwccqq] (-8.206896551724139,4.482758620689655)-- (-6,0);
	\draw [line width=1.2pt,color=wwqqcc] (-10,0)-- (1.7846032699052694,8.156104143347527);
	\draw [line width=1.2pt,color=ffwwqq] (-10,0)-- (-10.736403612827731,3.5527927894015696);
	\draw [line width=1.2pt,color=ffwwqq] (-10,0)-- (-12.908778820726198,14.03345697960008);
	\draw [line width=1.2pt,color=qqwuqq] (-6,0)-- (-12.908778820726198,14.03345697960008);
	\draw [line width=1.2pt,color=ffwwqq] (-12.908778820726198,14.03345697960008)-- (-6,10);
	\draw [line width=1.2pt,color=ffwwqq] (-9.436973501856762,12.133600585973628) -- (-9.547602375952922,11.944107943264914);
	\draw [line width=1.2pt,color=ffwwqq] (-9.361176444773278,12.089349036335163) -- (-9.471805318869437,11.899856393626452);
	\draw [line width=1.2pt,color=ffwwqq] (-10.736403612827731,3.5527927894015696)-- (-6,10);
	\draw [line width=1.2pt,color=ffwwqq] (-8.48260001110793,6.805984424305418) -- (-8.305767264811019,6.676075266577386);
	\draw [line width=1.2pt,color=ffwwqq] (-8.430636348016716,6.876717522824183) -- (-8.253803601719802,6.746808365096151);
	\draw [line width=1.2pt,color=wwqqcc] (-7.016540287675389,2.064847459340631)-- (-6,10);
	\draw [line width=1.2pt,color=wwqqcc] (-6.617092054644521,6.046364464388411) -- (-6.399448233030872,6.018482994952221);
	\draw [line width=1.2pt,color=wwqqcc] (1.7846032699052694,8.156104143347527)-- (-6,10);
	\draw [line width=1.2pt,color=wwqqcc] (-2.13298537363624,8.971294778935729) -- (-2.0824113564584885,9.184809364411802);
	\begin{scriptsize}
		\draw [fill=black] (-10,0) circle (2.5pt);
		\draw[color=black] (-9.831100607017337,0.4764893768675147) node {$B$};
		\draw [fill=black] (3,0) circle (2.5pt);
		\draw[color=black] (3.1806503888792195,0.4764893768675147) node {$C$};
		\draw [fill=black] (-6,10) circle (2.5pt);
		\draw[color=black] (-5.815669861234757,10.482152874555222) node {$A$};
		\draw [fill=black] (-6,0) circle (2pt);
		\draw[color=black] (-5.815669861234757,0.4326048878425687) node {$D$};
		\draw [fill=black] (-2.81767955801105,6.464088397790055) circle (2pt);
		\draw[color=black] (-2.6340444069261553,6.883624774509643) node {$E$};
		\draw [fill=black] (-8.206896551724139,4.482758620689655) circle (2pt);
		\draw[color=black] (-8.031836556994541,4.908822768387069) node {$F$};
		\draw [fill=black] (1.7846032699052694,8.156104143347527) circle (2pt);
		\draw[color=black] (2.0177114297181444,8.639004335507487) node {$P_1$};
		\draw [fill=black] (-10.736403612827731,3.5527927894015696) circle (2pt);
		\draw[color=black] (-10.48936794239153,4.05307523240062) node {$P_2$};
		\draw [fill=black] (-7.016540287675389,2.064847459340631) circle (2pt);
		\draw[color=black] (-6.781128619783574,2.5610026055524537) node {$Q_1$};
		\draw [fill=black] (-12.908778820726198,14.03345697960008) circle (2pt);
		\draw[color=black] (-12.661650149126368,14.519525864850262) node {$Q_2$};
	\end{scriptsize}
\end{tikzpicture}

\end{figure}

Let $\measuredangle$ denote directed angles mod $180^{\circ}$. The line EF meets
the circumcircle at two points. Directed angles allow us to treat both
at once, so we fix one of them.

Our goal is to show $\measuredangle PQA = \measuredangle APQ$, i.e., that $\triangle
	APQ$ is isosceles. We will first prove that $A$, $F$, $P$, $Q$ are concyclic.

Since $\triangle DEF$ is an orthic triangle, $\measuredangle CFA = \measuredangle
	ADC = 90^{\circ}$. Therefore, $FACD$ is cyclic, culminating in $\measuredangle
	ACD = \measuredangle AFD = \measuredangle AFQ$. However, $APCB$ is
cyclic, so $\measuredangle APB = \measuredangle ACB =
	\measuredangle ACD$. Thus, $\measuredangle APQ = \measuredangle AFQ$. Which
means that $AFPQ$ is a cyclic quadrilateral.

$FECB$ is a cyclic quadrilateral either, by the same $FACD$'s reason.
Therefore, putting together everything we've seen so far:

\[
	\measuredangle PQA = \measuredangle PFA = \measuredangle EFB = \measuredangle ECB
	= \measuredangle ACB = \measuredangle APB.
\]
Hence, $AP=AQ$.
