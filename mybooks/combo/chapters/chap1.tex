\subsubsection{USAJMO 2016 P4}
\begin{problem}
Find, with proof, the least integer $N$ such that if any $2016$ elements are
removed from the set ${1, 2,...,N}$, one can still find $2016$ distinct numbers
among the remaining elements with sum $N$.
\end{problem}

The least integer $N$ that satisfies the statement is
\[
	\sum^{4032}_{i=2017} i = 6049 \cdot 1008 = 6097392.
\]

Notice that if we form pairs of numbers from the set with equal sum, there will be
at least $3024$ such pairs. Even if at most $2016$ pairs are destroyed, there will
still remain at least $1008$ pairs with equal sum, i.e., $2016$ numbers. Each pair
consists of the $x$th number from the left and the $x$th number from the right.
This way, each pair has sum $6048+1$ as in $(1,6048), (2,6047), \dots, (3024,3025)$.
$6049 \cdot 1008 = 6097392$.

$N$ can't be less than $6,097,392$ because the least possible sum is
\[
	\sum^{2016}_{i=1} i=2033136.
\]
However, the first $2016$ numbers of the set can be removed, changing the least
possible sum to
\[
	\sum^{4032}_{i=2017} i = 6097392.
\]

\begin{insight}
	The most important idea that didn't came to my mind was forming pairs with equal
	sum, since I had already thought about lower bound $N \ge \sum^{2016}_{i=1}i$
	(not entirely correct, but almost there). It feels like the biggest problem
	was focusing more on doing something instead of thinking about what to do.
	Therefore, from now on it is really important to think better about what I am
	going to do before just doing.
\end{insight}

\clearpage

\subsubsection{USAJMO 2019 P1}

\begin{problem}
There are $a+b$ bowls arranged in a row, numbered $1$ through $a+b$, where $a$
and $b$ are given positive integers. Initially, each of the first $a$
bowls contains an apple, and each of the last $b$ bowls contains a
pear. A legal move consists of moving an apple from bowl $i$ to bowl $i+1$
and a pear from bowl $j$ to bowl $j-1$, provided that the difference $i-j$ is
even. We permit multiple fruits in the same bowl at the same time. The goal is
to end up with the first $b$ bowls each containing a pear and the last $a$
bowls each containing an apple. Show that this is possible if and only if the
product $ab$ is even.
\end{problem}

We claim that, in order to reach the desired configuration, $a$ and $b$ cannot
both be odd; equivalently, $ab$ must be even. We will first show that an even
product $ab$ always allows us to reach the goal. Then we will prove that the
desired configuration is impossible when $ab$ is odd.

Let $i'$ be the apple originally in bowl $i$ and $j'$ be the pear
in $j$.

\begin{claim}
	If $i-j$ is even, then $i'$ can always be swapped with $j'$.
\end{claim}

\begin{proof}
	WLOG assume $i<j$. After $n$ legal moves, $i'$ and $j'$ are in bowls $i+n$ and
	$j-n$, respectively. Since $i-j$ is even, $n=\frac{j-i}{2}$ is an integer.
	Therefore, $i'$ and $j'$ meet at
	\[
		i+n=j-n=\frac{i+j}{2}.
	\]
	After that, $i'$ can move to $\frac{i+j}{2}+n$ to reach $j$ and $j'$ to
	$\frac{i+j}{2}-n$ to reach $i$.
\end{proof}

Now suppose $ab$ is even.

\begin{itemize}
	\item If $\min(a,b)=1$, say $a=1$, then $b$ must be even for $ab$ to be even.
	      Therefore the bowls $1$ and $a+b$ can be swapped, because $(a+b)-1$ is
	      even.
	\item If $\min(a,b)\ge 2$
	      \begin{itemize}
		      \item and $a+b$ is odd, $(a+b)-1$ is even, so the first and the last
		            bowls can be swapped. Therefore, $ab$ can be reduced to $(a-1)
			            (b-1)$ since the bowls $1$ and $a+b$ already have the desired
		            fruits, which is still even. Thus, by complete induction, $a+b$
		            odd works, since $(a-1)+(b-1)$ is smaller than $a+b$.
		      \item and $a+b$ is even, $(a+b)-2$ and $(a+b-1)-1$ are even. $a$ and $b$
		            are both even numbers, because if they were both odd numbers, $ab$
		            wouldn't be even. So the
		            second and the last bowls can be swapped, as the first and
		            penultimate bowls. Therefore, we can reduce $ab$ to $(a-1)(b-1)$,
		            which is still even, since the bowls $1$, $2$, $a+b-1$ and $a+b$
		            already have the desired fruits. Thus, by complete induction,
		            $a+b$ even works, since $(a-2)+(b-2)$ is smaller than $a+b$.
	      \end{itemize}
\end{itemize}
Hence, when $ab$ is even, we can reach the desired goal.

Now, by the sake of contradiction, say $ab$ is odd. Let $A$ be the amount of
apples in the first odd-numbered bowls and $B$ of pears in the last odd-numbered
$b$ bowls. We already know that $i$ and $j$ must have parity in order to $i-j$ be
even. Therefore $A$ and $B$ are invariants, because after a legal move, $i'$ and
$j'$ must keep having parity, so $A$ and $B$ are always going to each decrease by
$1$ or increase by $1$.

Since $A-B$ is invariant, the remainder must always be equal, regardless of the fruits'
order (apples first, then pears, or pears first, then apples). Since $a$ is odd,
but $a+b$ is even, $A=\frac{a+1}{2}$ and $B=\frac{b-1}{2}$ originally, but after
the swaps, $A=\frac{a-1}{2}$ and $B=\frac{b+1}{2}$. Consequently,
\[
	\frac{a+1}{2}-\frac{b-1}{2}=\frac{a-1}{2}-\frac{b+1}{2}\iff\frac{a-b+2}{2}=
	\frac{a-b-2}{2}\iff 2=-2
\]
Contradiction! Hence, we can reach the desired goal if and only if $ab$ is even.

\begin{insight}
	I am proud of myself because this is the first serious combinatorics problem I
	solved, even though there were a lot of mistakes in my original solution.

	\begin{itemize}
		\item Lack of rigorosity. I understand now that writing more to make the
		      solution more comprehensible is good. However, time matters a lot in
		      an olympiad context. Therefore, solving all the problems before actually
		      writing the solutions seems to be the best thing to do. This way, I can
		      keep track of my time better, understand if I can or not be that rigorous.
		      Another thing about writing solutions is being more "mathematician".
		      What I mean is that writing solutions mathematically is often faster
		      and more understandable.
		\item Learned about induction, an awesome tool to use when an iteration or
		      anything related must be demonstrated. In this problem, we needed to show
		      that an specific configuration could only be obtained when $ab$ were even.
		      However, notice how many iterations and steps there are here!
		\item Learned about invariants either. Can be used in a similar way. Usually
		      when something changes a lot, there is another thing that actually never
		      changes, which can be useful to find contradictions.
	\end{itemize}
\end{insight}

\begin{problem}
A domino is a $ 1 \times 2 $ or $ 2 \times 1 $ tile. Let $n \ge 3 $ be an integer.
Dominoes are placed on an $n \times n$ board in such a way that each domino covers
exactly two cells of the board, and dominoes do not overlap. The value of a row
or column is the number of dominoes that cover at least one cell of this row or
column. The configuration is called balanced if there exists some $k \ge 1 $ such
that each row and each column has a value of $k$. Prove that a balanced
configuration exists for every $n \ge 3 $, and find the minimum number of dominoes
needed in such a configuration.
\end{problem}
