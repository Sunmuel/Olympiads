\documentclass[12pt]{article}

% Specify how big is going to be the paper margins.
\usepackage[a4paper, margin=1in]{geometry}

% amsmath: Add useful commans like aligh and gather.
% amsfonts: Add useful fonts like \mathbb{R}.
% amssymb: Add useful symbles like \therefore (needs amsfonts to work).
\usepackage{amsmath, amsfonts, amssymb}

% Makes the use of colors possible.
\usepackage{xcolor}

\definecolor{color1}{HTML}{084a8c}
\definecolor{color2}{HTML}{487fb7}
\definecolor{color3}{HTML}{95b7da}
\definecolor{color4}{HTML}{c9daea}

% Add Latin Modern Fonts like Sans-serif and Roman.
\usepackage{lmodern}

% Makes header and footer configurable.
\usepackage{fancyhdr}

% Makes the use of colored and configured tables possible
\usepackage{tcolorbox}

% Add commands to specify theorems like \newtheorem{x}{y}.
\usepackage{amsthm}

% Enables enumeration of items.
\usepackage{enumitem}

% Enables adding images.
\usepackage{graphicx}

% Enables cool hyper references.
\usepackage[colorlinks=true, linkcolor=color2, urlcolor=color2, citecolor=color2]{hyperref}

\title{\sffamily\bfseries{Soluções Jacob Palis 2022 N2}}
\author{Samuel de Araújo Brandão}
\date{4 de Setembro de 2025}

\pagestyle{fancy}
\fancyhf{}

\fancyhead[L]{\sffamily\bfseries{Soluções Jacob Palis 2022 N2}}
\fancyhead[R]{\textcolor{color2}{Samuel Brandão}, 4 de Setembro de 2025}
\fancyfoot[C]{\thepage}
\setlength{\headheight}{14.5pt}

\tcbset{
  statementbox/.style = {
    enhanced,
    width=\textwidth,

    title={Enunciado},
    titlefilled,
    fonttitle=\sffamily\bfseries,
    coltitle=color4
    colbacktitle=color1,

    colback=white,
    colframe=color1,
    boxrule=1pt,
    arc=2mm,
    boxsep=2pt,
  }
}

\tcbset{
  theorembox/.style = {
    enhanced,
    width=\textwidth,

    colback=white,
    colframe=color1,
    boxrule=1pt,
    arc=2mm,
    boxsep=2pt
  }
}

\tcbset{
  lemmabox/.style = {
    enhanced,
    width=\textwidth,

    colback=white,
    colframe=color2,
    boxrule=1pt,
    arc=2mm,
    boxsep=2pt
  }
}

\renewcommand*\contentsname{\textsf{Conteúdos}}
\newcommand{\kb}[1]{\left\lfloor #1 \right\rfloor}

\begin{document}
  \maketitle
  Uma coleção de soluções para a \textbf{Jacob Palis 2022 Nível 2}, inspirada no estilo de Evan Chen.
  Pode-se encontrar todos os problemas \textbf{\href{https://www.obm.org.br/content/uploads/2022/10/Prova_JacobPalis_2022.pdf}
  {aqui}} e as respostas oficiais \textbf{\href{https://www.obm.org.br/content/uploads/2022/10/gabarito_jacob_palis_2022.pdf}{aqui}}.

  Todas as soluções foram inteiramente escritas por mim, enquanto me preparava para a
  International Mathematical Olympiad (IMO).

  Caso encontre algum erro ou tiver sugestões ou comentários, sinta-se a vontade 
  para entrar em contato!

  \tableofcontents

  \clearpage

  \section{\textsf{Problemas}}
    \subsection{Testes}
      \begin{enumerate}[label=\textbf{\arabic*.}]
        \item Sônico, o tatu bola veloz, está caminhando pelas estradas do planeta Mobius coletando anéis. Abaixo está um mapa com a 
          quantidade de anéis em cada estrada do reino. Sabendo que Sônico partiu da cidade $A$ até a cidade $B$, a diferença entre a maior 
          e a menor quantidade de anéis que ele pode ter coletado é:
          \begin{figure}[h]
            \centering
            \includegraphics[width=0.25\textwidth]{first.png}
          \end{figure}
        \item Dentro de cada uma de duas caixas há uma bola vermelha ou uma bola verde. Na primeira caixa, do lado de fora, há um aviso:
          ``Dentro de pelo menos uma das duas caixas há uma bola verde'' e na segunda caixa há o aviso: ``Há uma bola vermelha dentro da outra 
          caixa''. Sabe-se que os dois avisos são ambos falsos ou ambos são verdadeiros. O que pode ser afirmado com certeza?
        \item Qual é a soma dos primeiros $2022$ dígitos após a vírgula na representação decimal da fração $\dfrac{2021}{148}$?
        \item Uma data é especial quando ocorre no dia $k$ do mês $k$ e cai no $k$-ésimo dia da semana (em que o primeiro dia da semana é 
          domingo) para algum $k$, $1 \le k \le 7$. Por exemplo, 5 de maio de 2022 é especial, pois caiu na quinta-feira (nesse exemplo, $k=5$).
          Qual é a maior quantidade de dias especiais que pode ocorrer em um ano? \\
          \textit{Dados:} Janeiro, Março e Maio têm 31 dias; Fevereiro tem 28 ou 29 dias; Abril e Junho têm 30 dias.
        \item A soma dos algarismos do número
          \[
            2022^{2022} + 2022^{2021} - 2 \times 2022^{2020} \times 2022^{2019}
          \]
          é igual a
        \item Na figura, $AD$ é altura e $O$ é o centro da circunferência circunscrita ao triângulo $ABC$. Sabendo que $\angle BAC = 72^\circ$, qual é a medida do ângulo $\angle AEB$?
          
          \begin{figure}[h]
            \centering
            \includegraphics[width=0.23\textwidth]{second.png}
          \end{figure}

          \item Um número inteiro positivo $n$ é tal que:
            \begin{itemize}[nosep]
              \item $n$ possui exatamente $15$ divisores positivos;
              \item $n$ não é múltiplo de $5$.
            \end{itemize}
            Qual é a diferença entre o quarto e o terceiro menores valores possíveis de $n$?
          \item No triângulo $ABC$, retângulo em $A$, temos $AB=6$ e $AC=8$. Seja $D$ o pé da altura relativa ao lado $BC$ e $M$ o ponto médio
            de $BC$. O valor de $DM$ é
          \item As raízes da equação $x^{2}-nx+m=0$ são os números reais não nulos $a$ e $b$. As raízes da equação $x^{2}-2nx+2m=0$ são 
            $a^{2}$ e $b^{2}$. O valor de $m+n$ é
            \item De quantas maneiras podemos cobrir algumas casas de um tabuleiro $10\times 10$ com dominós $2\times 1$ ou $1\times 2$ de tal forma que, dadas quaisquer quatro casas do tabuleiro que compartilham um vértice, exatamente duas delas estão cobertas por dominós? A seguir exibimos parte de um tabuleiro válido, em que os retângulos cinzas representam os dominós.
              \begin{figure}[h]
                \centering
                \includegraphics[width=0.2\textwidth]{third.png}
              \end{figure}
      \end{enumerate}
    \subsection{Respostas Numéricas}
    \begin{enumerate}[label=\textbf{\arabic*.}]
              \item Maria observou que os ponteiros de um relógio analógico formavam um ângulo de $33^\circ$ (sim, Maria mediu), como mostra a figura
        a seguir. Após algum tempo, menor do que uma hora, o ponteiro dos minutos (ponteiro maior) passou pelo ponteiro das horas (ponteiro
        menor) e formou novamente um ângulo de $33^\circ$. Quantos minutos se passaram? \\
        \begin{figure}[h]
          \centering
          \includegraphics[width=0.2\textwidth]{fourth.png}
        \end{figure}

      \item A calculadora do professor Piraldo tem só um botão. Ao apertá-lo, ele subtrai do número $N$ no visor o seu maior divisor diferente
        de $N$. Por exemplo, se no visor aparece $75$, o botão faz aparecer o número $75-25=50$. Suponha que a calculadora exiba o número
        $5^{2022}$ (a calculadora é bem grande!). Quantas vezes o botão deve ser apertado até que apareça $1$ no visor pela primeira vez?

      \item As 9 casinhas de um tabuleiro $3 \times 3$ devem ser preenchidas com os números de $1$ a $9$, com um número em cada casinha. Um preenchimento é \emph{sinuoso} quando todos os pares de números consecutivos ocupam casas vizinhas (com um lado em comum). O tabuleiro a seguir, por exemplo, é sinuoso:
        \[
          \begin{array}{|c|c|c|}\hline
            1 & 2 & 9 \\ \hline
            4 & 3 & 8 \\ \hline
            5 & 6 & 7 \\ \hline
          \end{array}
        \]
        Quantos são os tabuleiros sinuosos?

      \item Na figura, $M$ é ponto médio do lado de um quadrado de área $36$. Qual é a área do triângulo retângulo cinza? \\
        \begin{figure}[h]
          \centering
          \includegraphics[width=0.2\textwidth]{fifth.png}
        \end{figure}

      \item Para cada $x$ real denotamos $\kb{x}$ o maior inteiro menor que ou igual a $x$. Por exemplo, $\kb{3{,}15}=3$. Sabendo que
        \[
          \kb{\sqrt{1}}+\kb{\sqrt{2}}+\kb{\sqrt{3}}+\cdots+\kb{\sqrt{n}}=217,
        \]
        qual é o valor de $n$? \\

      \end{enumerate}

  \clearpage

  \section{\textsf{Soluções}}
    \subsection{Testes}
    \subsection{Respostas Numéricas}

  \clearpage

  \section{\textsf{Referências}}
\end{document}
