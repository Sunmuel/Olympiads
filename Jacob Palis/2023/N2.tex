\documentclass[12pt]{article}

% Specify how big is going to be the paper margins.
\usepackage[a4paper, margin=1in]{geometry}

% amsmath: Add useful commans like aligh and gather.
% amsfonts: Add useful fonts like \mathbb{R}.
% amssymb: Add useful symbles like \therefore (needs amsfonts to work).
\usepackage{amsmath, amsfonts, amssymb}

% Makes the use of colors possible.
\usepackage{xcolor}

\definecolor{color1}{HTML}{084a8c}
\definecolor{color2}{HTML}{487fb7}
\definecolor{color3}{HTML}{95b7da}
\definecolor{color4}{HTML}{c9daea}

% Add Latin Modern Fonts like Sans-serif and Roman.
\usepackage{lmodern}

% Makes header and footer configurable.
\usepackage{fancyhdr}

% Makes the use of colored and configured tables possible
\usepackage{tcolorbox}

% Add commands to specify theorems like \newtheorem{x}{y}.
\usepackage{amsthm}

% Enables enumeration of items.
\usepackage{enumitem}

% Enables adding images.
\usepackage{graphicx}

% Enables cool hyper references.
\usepackage[colorlinks=true, linkcolor=color2, urlcolor=color2, citecolor=color2]{hyperref}

\title{\sffamily\bfseries{Soluções Jacob Palis 2023 N2}}
\author{Samuel de Araújo Brandão}
\date{4 de Setembro de 2025}

\pagestyle{fancy}
\fancyhf{}

\fancyhead[L]{\sffamily\bfseries{Soluções Jacob Palis 2023 N2}}
\fancyhead[R]{\textcolor{color2}{Samuel Brandão}, 4 de Setembro de 2025}
\fancyfoot[C]{\thepage}
\setlength{\headheight}{14.5pt}

\tcbset{
  statementbox/.style = {
    enhanced,
    width=\textwidth,

    title={Enunciado},
    titlefilled,
    fonttitle=\sffamily\bfseries,
    coltitle=color4
    colbacktitle=color1,

    colback=white,
    colframe=color1,
    boxrule=1pt,
    arc=2mm,
    boxsep=2pt,
  }
}

\tcbset{
  theorembox/.style = {
    enhanced,
    width=\textwidth,

    colback=white,
    colframe=color1,
    boxrule=1pt,
    arc=2mm,
    boxsep=2pt
  }
}

\tcbset{
  lemmabox/.style = {
    enhanced,
    width=\textwidth,

    colback=white,
    colframe=color2,
    boxrule=1pt,
    arc=2mm,
    boxsep=2pt
  }
}

\renewcommand*\contentsname{\textsf{Conteúdos}}
\newcommand{\kb}[1]{\left\lfloor #1 \right\rfloor}

\begin{document}
  \maketitle
  Uma coleção de soluções para a \textbf{Jacob Palis 2023 Nível 2}, inspirada no estilo de Evan Chen.
  Pode-se encontrar todos os problemas e respostas oficiais 
  \textbf{\href{https://www.obm.org.br/content/uploads/2025/04/prova_jacob_palis_2023.pdf}{aqui}}.

  Todas as soluções foram inteiramente escritas por mim, enquanto me preparava para a
  International Mathematical Olympiad (IMO).

  Caso encontre algum erro ou tiver sugestões ou comentários, sinta-se a vontade 
  para entrar em contato!

  \tableofcontents

  \clearpage

  \section{\textsf{Problemas}}
    \subsection{Testes}
      \begin{enumerate}[label=\textbf{\arabic*.}]
        \item Ana foi à feira com 20 reais, comprou 3 bananas e 2 peras e recebeu certo valor de troco. Mais tarde, seu irmão João foi ao
          mesmo local com 29 reais, comprou 5 bananas e 3 peras e também recebeu troco. Depois Maria, mãe de João e Ana, comprou mais uma
          banana e uma pera. Sabendo que Ana, João e Maria receberam a mesma quantia de troco, quantos reais Maria levou para a feira?
        \item Regis vai comprar uma capinha personalizada de celular na internet. A capinha custa 100 reais, o frete custa 20 reais e a
          personalização custa 30 reais. Regis possui dois cupons de desconto, mas só pode usar um deles. O primeiro dá frete grátis e o
          segundo dá desconto de 20\% no total da compra (capinha, frete e personalização). Se Regis usar o cupom no qual paga o menor valor
          possível, quanto Regis vai pagar?
        \item José preencheu um tabuleiro $3\times3$ com os números de 1 a 9 e notou que a soma dos números em \(k\) filas (linhas ou colunas)
          era ímpar. Quantos são os possíveis valores para \(k\)?
        \item Qual é o número mínimo de cores necessárias para colorir as bolinhas da figura abaixo de modo que bolinhas ligadas por um
          segmento tenham cores distintas?
          \begin{figure}[h]
            \centering
            \includegraphics[width=0.3\textwidth]{first.png}
          \end{figure}
        \item José escreveu no quadro a igualdade
          \[
            2^n + 2^n + \cdots + 2^n = 15360.
          \]
          Maria percebeu que havia \(2m+1\) parcelas iguais a \(2^n\) no lado esquerdo, sendo \(m\) um número inteiro. Quanto vale \(m+n\)?
        \item O número de seis algarismos \(N = (2aaaa6)\) é divisível por 24. A soma dos algarismos de \(N\) é quanto?
        \item Sendo \(x\) e \(y\) reais tais que
          \[
            \frac{x+1}{2^y+1} = \frac{x+2}{2^y+2} = k,
          \]
          quanto vale \(k\)?
        \item De quantas maneiras podemos pintar as letras da palavra JACOB se as vogais devem ser coloridas de azul ou vermelho e as
          consoantes devem ser coloridas de azul ou verde e, além disso, não podemos ter letras adjacentes com a mesma cor?

        \item As letras \(O, B, M, J, P\) representam algarismos distintos. Sabendo que
          \[
            OBM + OBM = JP \cdot JP,
          \]
          qual é o valor de \(O+B+M+J+P\)?
        \item Na figura a seguir, \(ABCD\) é um paralelogramo. Os pontos \(M\) e \(N\) são pontos médios de \(DP\) e \(BP\), respectivamente.
          Se a área do paralelogramo \(ABCD\) é 24, qual é a área da região sombreada?
          \begin{figure}[h]
            \centering
            \includegraphics[width=0.3\textwidth]{second.png}
          \end{figure}
        \item Seja \(X\subset\{1,2,\dots,2023\}\) tal que, se \(a,b\in X\), então \(a+b\) não é múltiplo de 3. Qual é o maior valor possível
          da quantidade de elementos de \(X\)?
        \item O número
          \[
            \sqrt{2022^2 + 2023^2 + (2022\cdot2023)^2} + \sqrt{2023^2 + 2024^2 + (2023\cdot2024)^2}
          \]
          é qual tipo de número (irracional, inteiro e múltiplo de 3, inteiro e múltiplo de 5, inteiro e múltiplo de 8, primo)?
        \item No triângulo acutângulo \(ABC\), \(AH\) é altura, com \(H\) sobre \(BC\). Sejam \(P\) e \(Q\) as projeções de \(H\) em \(AB\)
          e \(AC\), respectivamente. Sabendo que \(\angle ABC - \angle ACB = 20^\circ\). Qual é o ângulo agudo determinado pelas retas \(PQ\)
          e \(AH\)?
        \item Sejam \(a\) e \(b\) números reais. As raízes da equação \(x^2 - a x + b = 0\) são \(r\) e \(s\), e as raízes da equação
          \(x^2 - (b+3)x + (a+3) = 0\) são \(1/r\) e \(1/s\). Então \((b+1)^3\) é igual a quê?
        \item Considere que \(n\) times de futebol jogam exatamente uma vez contra cada um dos outros \(n-1\) times. Em cada partida, o time 
          vencedor ganha 3 pontos e o perdedor 0; em caso de empate, cada time ganha 1 ponto. Ao fim do campeonato, ordenamos os times por 
          pontos em ordem decrescente. Para \(n=3\), há sete possibilidades de pontuações dos três times listadas. Para \(n=4\), há quantas
          possibilidades de pontuação dos quatro times?
      \end{enumerate}
    \subsection{Respostas Numéricas}
      \begin{enumerate}[label=\textbf{\arabic*.}, start=16]
      \item Se \(a\) e \(b\) são inteiros positivos tais que \(\gcd(a,b) = 6\) e \(\mathrm{lcm}(a,b) = 36a^2\), quanto vale \(a+b\)?
      \item De quantos modos podemos colorir um tabuleiro \(2\times8\) de modo que cada quadrado unitário seja verde ou amarelo e cada 
        quadrado \(2\times2\) possua três quadrados unitários de uma cor e o outro da cor oposta?
      \item No trapézio retângulo \(ABCD\), com \(\angle ABC = \angle BCD = 90^\circ\), a base \(AB\) mede 104 e a base \(CD\) mede 234.
        Sabendo que a bissetriz do ângulo \(D\) intersects \(BC\) no seu ponto médio, determine a medida do lado \(BC\).
        \begin{figure}[h]
          \centering
          \includegraphics[width=0.25\textwidth]{third.png}
        \end{figure}
      \item Um número é chamado de perfeitoso quando nenhum dos seus algarismos é zero e a soma dos seus algarismos é um quadrado perfeito.
        Por exemplo, 97, 112 e 1111 são números perfeitosos com 2, 3 e 4 algarismos, respectivamente. Luiz escreveu todos os números 
        perfeitosos de 2023 algarismos. Quantos valores possíveis para a soma dos algarismos dos números da lista?
      \item A sequência de Fibonacci começa com 1, 1, e cada termo a partir do terceiro é a soma dos dois anteriores. Entre os 100 primeiros 
        termos, quantos são múltiplos de 3 ou de 4?
      \end{enumerate}

  \clearpage

  \section{\textsf{Soluções}}
    \subsection{Testes}
    \subsection{Respostas Numéricas}

  \clearpage

  \section{\textsf{Referências}}
\end{document}
