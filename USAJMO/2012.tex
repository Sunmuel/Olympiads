\documentclass[12pt]{article}
\usepackage{amsmath, amssymb} % Recomendado para melhor formatação matemática
\usepackage{lmodern}
\usepackage{fancyhdr}
\usepackage{xcolor}
\usepackage{enumitem}
\usepackage[most]{tcolorbox}

\definecolor{color1}{HTML}{941b0c}
\definecolor{color2}{HTML}{bc3908}
\definecolor{color3}{HTML}{f6aa1c}

\usepackage[font=small, labelfont=bf, justification=centering]{caption}
\usepackage[colorlinks=true, linkcolor=black, urlcolor=color2, citecolor=black]{hyperref}
\usepackage[a4paper, margin=1in]{geometry}

\renewcommand*\contentsname{\textsf{Contents}}

\title{\bfseries\sffamily{USAJMO 2012 Solutions}}
\author{Samuel de Araújo Brandão}
\date{\today}


\pagestyle{fancy}
\fancyhf{}

\fancyhead[L]{\textbf{USAJMO 2012 Solution}}
\fancyhead[R]{\textcolor{color2}{Samuel Brandão}, \today}
\fancyfoot[C]{\thepage}
\setlength{\headheight}{14.5pt}

\tcbset{
  problembox/.style={
    enhanced,
    colback=white,
    colframe=color2,
    boxrule=1pt,
    arc=2mm,
    top=1pt,
    bottom=1pt,
    left=1pt,
    right=1pt,
    fonttitle=\sffamily\bfseries\color{white},
    colbacktitle=color2,
    title={#1}
  }
}

\begin{document}
  \maketitle
  A collection of USAJMO 2012 solutions, inspired by Evan Chen’s style.

  All solutions were written by me while preparing for the International Mathematical Olympiad (IMO).

  If you spot any errors or have suggestions or comments, feel free to reach out!
  \tableofcontents
  \clearpage

  \section{\textsf{Problems}}
  \begin{enumerate}[label=\textbf{\arabic*.}]
    \item Given a triangle $ABC$, let $P$ and $Q$ be points on segments $\overline{AB}$ and $\overline{AC}$,
    respectively, such that $AP = AQ$. Let $S$ and $R$ be distinct points on segment $\overline{BC}$
    such that $S$ lies between $B$ and $R$, $\angle BPS = \angle PRS$, and $\angle CQR = \angle QSR$.
    Prove that $P$, $Q$, $R$, $S$ are concyclic.

  \end{enumerate}

  \clearpage

  \section{\textsf{Solutions: Day 1}}
    \subsection{Problem 1.}

      \begin{tcolorbox}[problembox=Problem statement]
        Given a triangle $ABC$, let $P$ and $Q$ be points on segments $\overline{AB}$ and $\overline{AC}$,
        respectively, such that $AP = AQ$. Let $S$ and $R$ be distinct points on segment $\overline{BC}$
        such that $S$ lies between $B$ and $R$, $\angle BPS = \angle PRS$, and $\angle CQR = \angle QSR$.
        Prove that $P$, $Q$, $R$, $S$ are concyclic.
      \end{tcolorbox}
      By the $\textit{Alternate Segment Theorem}, \overline{AC}$ is tangent to $(QRS)$ and $\overline{AB}$
      is tangent to $(PRS)$. Assume for the sake of contradiction that $(QRS)$ and $(PRS)$ are distinct. In that case,
      $A \in \overline{BC}$ since $\overline{BC}$ is the radical axis and $\text{Pow}_{(QRS)}(A) =
      \text{Pow}_{(PRS)}(A)$. This leads to a contradiction, as $A \notin \overline{BC}$. Therefore,
      $P$, $Q$, $R$ and $S$ are concyclic.

    \clearpage
    \subsection{Problem 2.}
    \clearpage
    \subsection{Problem 3.}
    \clearpage

  \section{\textsf{Solutions: Day 2}}
    \subsection{Problem 4.}
    \clearpage
    \subsection{Problem 5.}
    \clearpage
    \subsection{Problem 6.}
    \clearpage

    \section{\textsf{References}}
    This document was made possible thanks to the help and inspiration of the following resource:
    \renewcommand{\refname}{\vspace{-2em}}
    \begin{thebibliography}{9}
      \bibitem{noic}
      NOIC - Núcleo Olímpico de Iniciação Científica
      \textit{Potência de Ponto}, 2023.
      Available in: \href{https://noic.com.br/wp-content/uploads/2023/10/potencia.pdf}
      {https://noic.com.br/wp-content/uploads/2023/10/potencia.pdf}
      \bibitem{obm}
      OBM - Olimpíada Brasileira de Matemática.
      \textit{Potência de Ponto, Eixo Radical, Centro Radical e Aplicações}, 2017.
      Available in: \href{https://www.obm.org.br/content/uploads/2017/01/eixos-2.pdf}
      {https://www.obm.org.br/content/uploads/2017/01/eixos-2.pdf}
      \bibitem{evan}
      Evan Chen.
      \textit{JMO 2012 Solution Notes}, 2025.
      Available in: \href{https://web.evanchen.cc/exams/JMO-2012-notes.pdf}
      {https://web.evanchen.cc/exams/JMO-2012-notes.pdf}
    \end{thebibliography}
\end{document}

