\documentclass[12pt]{article}
\usepackage{amsmath, amssymb} % Recomendado para melhor formatação matemática
\usepackage[font=small, labelfont=bf, justification=centering]{caption}
\usepackage[colorlinks=true, linkcolor=black, urlcolor=red, citecolor=black]{hyperref}
\usepackage[a4paper, margin=1in]{geometry}
\usepackage{lmodern}
\usepackage{fancyhdr}
\usepackage{xcolor}
\usepackage{enumitem}
\renewcommand*\contentsname{\textsf{Contents}}

\title{\bfseries\sffamily{USAJMO 2012 Solutions}}
\author{Samuel de Araújo Brandão}
\date{\today}

\definecolor{red1}{HTML}{800e13}
\definecolor{red2}{HTML}{3a5a40}
\definecolor{red3}{HTML}{800e13}

\pagestyle{fancy}
\fancyhf{}

\fancyhead[L]{\textbf{USAJMO 2012 Solution}}
\fancyhead[R]{\textcolor{red1}{Samuel de Araújo Brandão}, \today}

\begin{document}
  \maketitle

  \tableofcontents
  \clearpage

  \section{\textsf{Problems}}
  \begin{enumerate}[label=\textbf{\arabic*.}]
    \item Given a triangle $ABC$, let $P$ and $Q$ be points on segments $\overline{AB}$ and $\overline{AC}$,
    respectively, such that $AP = AQ$. Let $S$ and $R$ be distinct points on segment $\overline{BC}$
    such that $S$ lies between $B$ and $R$, $\angle BPS = \angle PRS$, and $\angle CQR = \angle QSR$.
    Prove that $P$, $Q$, $R$, $S$ are concyclic.

  \end{enumerate}

  \clearpage

  \section{\textsf{Solutions: day 1}}
    \subsection{Problem 1.}
    \clearpage
    \subsection{Problem 2.}
    \clearpage
    \subsection{Problem 3.}
    \clearpage

  \section{\textsf{Solutions: day 2}}
   \subsection{Problem 4.}
   \clearpage
   \subsection{Problem 5.}
   \clearpage
   \subsection{Problem 6.}
   \clearpage

  {\centering \textbf{Problem 1} \par}

  Given a triangle $ABC$, let $P$ and $Q$ be points on segments $\overline{AB}$ and $\overline{AC}$,
  respectively, such that $AP = AQ$. Let $S$ and $R$ be distinct points on segment $\overline{BC}$
  such that $S$ lies between $B$ and $R$, $\angle BPS = \angle PRS$, and $\angle CQR = \angle QSR$.
  Prove that $P$, $Q$, $R$, $S$ are concyclic.

  {\centering \textbf{Solution} \par}

  By the $\textit{Alternate Segment Theorem}, \overline{AC}$ is tangent to $(QRS)$ and $\overline{AB}$
  is tangent to $(PRS)$. Assume for the sake of contradiction that $(QRS)$ and $(PRS)$ are distinct. In that case,
  $A \in \overline{BC}$ since $\overline{BC}$ is the radical axis and $\text{Pow}_{(QRS)}(A) =
  \text{Pow}_{(PRS)}(A)$. This leads to a contradiction, as $A \notin \overline{BC}$. Therefore,
  $P$, $Q$, $R$ and $S$ are concyclic.
\end{document}
