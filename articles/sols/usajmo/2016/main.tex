\documentclass[12pt]{article}
\usepackage{../../../sty/sols}

\olympiad{\textsf{\textbf{USAJMO 2016}}}

\title{\olympiad}
\author{Samuel de Araújo Brandão}
\date{\today}

\begin{document}
\maketitle
\preface

\tableofcontents

\clearpage

\section{\textsf{Problems}}
\begin{enumerate}[label={\textbf{\arabic*.}}]
	\item Find, with proof, the least integer $N$ such that if any $2016$ elements are
	      removed from the set ${1, 2,...,N}$, one can still find $2016$ distinct numbers
	      among the remaining elements with sum $N$.
\end{enumerate}

\clearpage

\section{\textsf{Solutions}}

\begin{problem}
Find, with proof, the least integer $N$ such that if any $2016$ elements are
removed from the set ${1, 2,...,N}$, one can still find $2016$ distinct numbers
among the remaining elements with sum $N$.
\end{problem}

\begin{claim}
	badfals
\end{claim}

\begin{observation}
	adfadf
\end{observation}

The least integer $N$ that satisfies the statement is
\[
	\sum^{4032}_{i=2017} i = 6049 \cdot 1008 = 6097392.
\]

Notice that if we form pairs of numbers from the set with equal sum, there will be
at least $3024$ such pairs. Even if at most $2016$ pairs are destroyed, there will
still remain at least $1008$ pairs with equal sum, i.e., $2016$ numbers. Each pair
consists of the $x$th number from the left and the $x$th number from the right.
This way, each pair has sum $6048+1$ as in $(1,6048), (2,6047), \dots, (3024,3025)$.
$6049 \cdot 1008 = 6097392$.

$N$ can't be less than $6,097,392$ because the least possible sum is
\[
	\sum^{2016}_{i=1} i=2033136.
\]
However, the first $2016$ numbers of the set can be removed, changing the least
possible sum to
\[
	\sum^{4032}_{i=2017} i = 6097392.
\]

\clearpage

\section{\textsf{References}}\label{sec:references}
\end{document}
