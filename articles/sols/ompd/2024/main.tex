\documentclass[12pt]{article}
\usepackage{../../../sty/articlept}

\olympiad{OMpD 2024 N2 Fase 2}

\title{\olympiad}
\author{Samuel de Araújo Brandão}
\date{\today}

\begin{document}
\maketitle
\preface

\tableofcontents

\clearpage

\section{Problemas}

\begin{enumerate}[label={\textbf{\arabic*.}}]
	\item Sejam $O, M, P, D$ algarismos distintos entre si, e distintos de zero,
	      tais que $O < M < P < D$ e a seguinte equação é verdadeira:
	      \[
		      OMP D \times (OM - D) = MDDMP - OM.
	      \]

	      \begin{itemize}
		      \item[(a)] Usando estimativas, explique por que é impossível que o valor de $O$ seja maior do que ou igual a $3$.
		      \item[(b)] Explique por que $O$ não pode ser igual a $1$.
		      \item[(c)] É possível termos $M$ maior do que ou igual a $5$? Justifique.
		      \item[(d)] Determine os valores de $M, P$ e $D$.
	      \end{itemize}

	      Observação: $X_1X_2 \ldots X_n$ é o número obtido da justaposição
	      dos algarismos \\ $X_1, X_2, \ldots, X_n$, nessa ordem. Por exemplo,
	      se $n = 4, X_1 = 2, X_2 = 0, X_3 = 2$ e $X_4 = 2$, então $X_1X_2
		      \ldots X_n = 2024$.
	\item Sejam $ABCD$ um quadrilátero convexo, $M, N$ e $P$ os pontos médios das
	      diagonais $AC$ e $BD$ e do lado $AD$, respectivamente. Suponha também
	      que $\angle ABC + \angle DCB = 90^\circ$ e que $AB = 6, CD = 8$. Calcule
	      o perímetro do triângulo $MNP$.
	\item Uma barata tonta está inicialmente no vértice $A$ de um cubão
	      $ABCDEFGH$ de aresta medindo $1$ metro, conforme a figura ao lado.
	      A cada segundo, a barata anda $1$ metro, sempre escolhendo ir para um dos
	      três vértices adjacentes ao vértice que ela está. Por exemplo, após
	      $1$ segundo a barata pode parar no vértice $B, D$ ou $E$.

	      \begin{figure}[h]
		      \centering
		      \includegraphics[width=0.25\textwidth]{../../../img/ompd/6.png}
	      \end{figure}

	\item Seja $a_0, a_1, a_2, \ldots$ uma sequência infinita de inteiros
	      positivos com as seguintes propriedades:

	      \begin{itemize}
		      \item $a_0$ é um inteiro positivo dado;
		      \item Para cada $n \geq 1$ inteiro, $a_n$ é o menor inteiro maior
		            do que $a_{n-1}$ de tal modo que $a_n + a_{n-1}$ é um quadrado
		            perfeito.
	      \end{itemize}

	      Por exemplo, se $a_0 = 3$, então $a_1 = 6, a_2 = 10, a_3 = 15$ e assim
	      por diante.

	      \begin{itemize}
		      \item[(a)] Seja $T$ o conjunto dos números da forma $a_k - a_\ell$, com
		            $k \geq \ell \geq 0$ inteiros. Demonstre que, independentemente do
		            valor de $a_0$, o número de inteiros positivos que não estão em
		            $T$ é finito.
		      \item[(b)] Calcule, em função de $a_0$, a quantidade de inteiros
		            positivos que não estão em $T$.
	      \end{itemize}

\end{enumerate}

\clearpage

\section{Soluções}

\subsection{Problema 1.}
\begin{problem}
Sejam $O, M, P, D$ algarismos distintos entre si, e distintos de zero,
tais que $O < M < P < D$ e a seguinte equação é verdadeira:
\[
	OMP D \times (OM - D) = MDDMP - OM.
\]

\begin{itemize}
	\item[(a)] Usando estimativas, explique por que é impossível que o valor de $O$ seja maior do que ou igual a $3$.
	\item[(b)] Explique por que $O$ não pode ser igual a $1$.
	\item[(c)] É possível termos $M$ maior do que ou igual a $5$? Justifique.
	\item[(d)] Determine os valores de $M, P$ e $D$.
\end{itemize}

Observação: $X_1X_2 \ldots X_n$ é o número obtido da justaposição
dos algarismos \\ $X_1, X_2, \ldots, X_n$, nessa ordem. Por exemplo,
se $n = 4, X_1 = 2, X_2 = 0, X_3 = 2$ e $X_4 = 2$, então $X_1X_2
	\ldots X_n = 2024$.
\end{problem}

\clearpage

\subsection{Problema 2.}
\begin{problem}
Sejam $ABCD$ um quadrilátero convexo, $M, N$ e $P$ os pontos médios das
diagonais $AC$ e $BD$ e do lado $AD$, respectivamente. Suponha também
que $\angle ABC + \angle DCB = 90^\circ$ e que $AB = 6, CD = 8$. Calcule
o perímetro do triângulo $MNP$.
\end{problem}

\clearpage

\subsection{Problema 3.}
\begin{problem}
Uma barata tonta está inicialmente no vértice $A$ de um cubão
$ABCDEFGH$ de aresta medindo $1$ metro, conforme a figura ao lado.
A cada segundo, a barata anda $1$ metro, sempre escolhendo ir para um dos
três vértices adjacentes ao vértice que ela está. Por exemplo, após
$1$ segundo a barata pode parar no vértice $B, D$ ou $E$.

\begin{center}
	\includegraphics[width=0.25\textwidth]{../../../img/ompd/6.png}

\end{center}

\end{problem}

\clearpage

\subsection{Problema 4.}
\begin{problem}
Seja $a_0, a_1, a_2, \ldots$ uma sequência infinita de inteiros
positivos com as seguintes propriedades:

\begin{itemize}
	\item $a_0$ é um inteiro positivo dado;
	\item Para cada $n \geq 1$ inteiro, $a_n$ é o menor inteiro maior
	      do que $a_{n-1}$ de tal modo que $a_n + a_{n-1}$ é um quadrado
	      perfeito.
\end{itemize}

Por exemplo, se $a_0 = 3$, então $a_1 = 6, a_2 = 10, a_3 = 15$ e assim
por diante.

\begin{itemize}
	\item[(a)] Seja $T$ o conjunto dos números da forma $a_k - a_\ell$, com
	      $k \geq \ell \geq 0$ inteiros. Demonstre que, independentemente do
	      valor de $a_0$, o número de inteiros positivos que não estão em
	      $T$ é finito.
	\item[(b)] Calcule, em função de $a_0$, a quantidade de inteiros
	      positivos que não estão em $T$.
\end{itemize}

\end{problem}

\clearpage

\section{Referências}\label{sec:references}
\end{document}
