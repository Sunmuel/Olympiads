\documentclass[12pt]{article}
\usepackage{../../../sty/sols}

\olympiad{\textsf{\textbf{Solutions IMO Shortlist 2010}}}

\title{\olympiad}
\author{Samuel de Araújo Brandão}
\date{\today}

\begin{document}
\maketitle
\preface

\tableofcontents

\clearpage

\section{\textsf{Problems}}
\begin{enumerate}[label={\textbf{\arabic*.}}]
	\item Let $ABC$ be an acute triangle with $D, E, F$ the feet of the altitudes
	      lying on $BC, CA, AB$ respectively. One of the intersection points of the
	      line $EF$ and the circumcircle is $P.$ The lines $BP$ and $DF$ meet at point
	      $Q.$ Prove that $AP = AQ.$
\end{enumerate}

\clearpage

\section{\textsf{Solutions}}

\begin{problem}
Let $ABC$ be an acute triangle with $D, E, F$ the feet of the altitudes
lying on $BC, CA, AB$ respectively. One of the intersection points of the
line $EF$ and the circumcircle is $P.$ The lines $BP$ and $DF$ meet at point
$Q.$ Prove that $AP = AQ$.
\end{problem}

\begin{figure}[h]
	\centering
	\definecolor{ffwwqq}{rgb}{1,0.4,0}
\definecolor{wwqqcc}{rgb}{0.4,0,0.8}
\definecolor{wwccqq}{rgb}{0.4,0.8,0}
\definecolor{qqwuqq}{rgb}{0,0.39215686274509803,0}
\definecolor{ffqqtt}{rgb}{1,0,0.2}
\definecolor{qqzzff}{rgb}{0,0.6,1}
\begin{tikzpicture}[scale=0.7,line cap=round,line join=round,>=triangle 45,x=1cm,y=1cm]
	\clip(-14,-5) rectangle (3.85066872014997,15.100995344430798);
	% \clip(-24.751826875499052,-5.393061030219024) rectangle (16.85066872014997,15.100995344430798);
	\fill[line width=1.2pt,color=qqzzff,fill=qqzzff,fill opacity=0.1] (-6,10) -- (-10,0) -- (3,0) -- cycle;
	\draw [line width=1.2pt,color=qqzzff] (-6,10)-- (-10,0);
	\draw [line width=1.2pt,color=qqzzff] (-10,0)-- (3,0);
	\draw [line width=1.2pt,color=qqzzff] (3,0)-- (-6,10);
	\draw [line width=1.2pt,color=ffqqtt] (-3.5,3.2) circle (7.244998274671981cm);
	\draw [line width=1.2pt,color=qqwuqq] (1.7846032699052694,8.156104143347527)-- (-10.736403612827731,3.5527927894015696);
	\draw [line width=1.2pt,color=wwccqq] (-8.206896551724139,4.482758620689655)-- (-6,0);
	\draw [line width=1.2pt,color=wwqqcc] (-10,0)-- (1.7846032699052694,8.156104143347527);
	\draw [line width=1.2pt,color=ffwwqq] (-10,0)-- (-10.736403612827731,3.5527927894015696);
	\draw [line width=1.2pt,color=ffwwqq] (-10,0)-- (-12.908778820726198,14.03345697960008);
	\draw [line width=1.2pt,color=qqwuqq] (-6,0)-- (-12.908778820726198,14.03345697960008);
	\draw [line width=1.2pt,color=ffwwqq] (-12.908778820726198,14.03345697960008)-- (-6,10);
	\draw [line width=1.2pt,color=ffwwqq] (-9.436973501856762,12.133600585973628) -- (-9.547602375952922,11.944107943264914);
	\draw [line width=1.2pt,color=ffwwqq] (-9.361176444773278,12.089349036335163) -- (-9.471805318869437,11.899856393626452);
	\draw [line width=1.2pt,color=ffwwqq] (-10.736403612827731,3.5527927894015696)-- (-6,10);
	\draw [line width=1.2pt,color=ffwwqq] (-8.48260001110793,6.805984424305418) -- (-8.305767264811019,6.676075266577386);
	\draw [line width=1.2pt,color=ffwwqq] (-8.430636348016716,6.876717522824183) -- (-8.253803601719802,6.746808365096151);
	\draw [line width=1.2pt,color=wwqqcc] (-7.016540287675389,2.064847459340631)-- (-6,10);
	\draw [line width=1.2pt,color=wwqqcc] (-6.617092054644521,6.046364464388411) -- (-6.399448233030872,6.018482994952221);
	\draw [line width=1.2pt,color=wwqqcc] (1.7846032699052694,8.156104143347527)-- (-6,10);
	\draw [line width=1.2pt,color=wwqqcc] (-2.13298537363624,8.971294778935729) -- (-2.0824113564584885,9.184809364411802);
	\begin{scriptsize}
		\draw [fill=black] (-10,0) circle (2.5pt);
		\draw[color=black] (-9.831100607017337,0.4764893768675147) node {$B$};
		\draw [fill=black] (3,0) circle (2.5pt);
		\draw[color=black] (3.1806503888792195,0.4764893768675147) node {$C$};
		\draw [fill=black] (-6,10) circle (2.5pt);
		\draw[color=black] (-5.815669861234757,10.482152874555222) node {$A$};
		\draw [fill=black] (-6,0) circle (2pt);
		\draw[color=black] (-5.815669861234757,0.4326048878425687) node {$D$};
		\draw [fill=black] (-2.81767955801105,6.464088397790055) circle (2pt);
		\draw[color=black] (-2.6340444069261553,6.883624774509643) node {$E$};
		\draw [fill=black] (-8.206896551724139,4.482758620689655) circle (2pt);
		\draw[color=black] (-8.031836556994541,4.908822768387069) node {$F$};
		\draw [fill=black] (1.7846032699052694,8.156104143347527) circle (2pt);
		\draw[color=black] (2.0177114297181444,8.639004335507487) node {$P_1$};
		\draw [fill=black] (-10.736403612827731,3.5527927894015696) circle (2pt);
		\draw[color=black] (-10.48936794239153,4.05307523240062) node {$P_2$};
		\draw [fill=black] (-7.016540287675389,2.064847459340631) circle (2pt);
		\draw[color=black] (-6.781128619783574,2.5610026055524537) node {$Q_1$};
		\draw [fill=black] (-12.908778820726198,14.03345697960008) circle (2pt);
		\draw[color=black] (-12.661650149126368,14.519525864850262) node {$Q_2$};
	\end{scriptsize}
\end{tikzpicture}

\end{figure}

Let $\measuredangle$ denote directed angles mod $180^{\circ}$. The line EF meets
the circumcircle at two points. Directed angles allow us to treat both
at once, so we fix one of them.

Our goal is to show $\measuredangle PQA = \measuredangle APQ$, i.e., that $\triangle
	APQ$ is isosceles. We will first prove that $A$, $F$, $P$, $Q$ are concyclic.

Since $\triangle DEF$ is an orthic triangle, $\measuredangle CFA = \measuredangle
	ADC = 90^{\circ}$. Therefore, $FACD$ is cyclic, culminating in $\measuredangle
	ACD = \measuredangle AFD = \measuredangle AFQ$. However, $APCB$ is
cyclic, so $\measuredangle APB = \measuredangle ACB =
	\measuredangle ACD$. Thus, $\measuredangle APQ = \measuredangle AFQ$. Which
means that $AFPQ$ is a cyclic quadrilateral.

$FECB$ is a cyclic quadrilateral either, by the same $FACD$'s reason.
Therefore, putting together everything we've seen so far:

\[
	\measuredangle PQA = \measuredangle PFA = \measuredangle EFB = \measuredangle ECB
	= \measuredangle ACB = \measuredangle APB.
\]
Hence, $AP=AQ$.

\clearpage

\section{\textsf{References}}\label{sec:references}
\end{document}
