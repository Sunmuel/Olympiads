\documentclass[12pt]{article}
\usepackage{../sty/otis}

\begin{document}

\clearpage

\section{\textsf{Ineq Basic}}
\begin{itemize}
	\item [\texttt{\textcolor{color7}{11AMO1}}] \href{https://artofproblemsolving.com/community/c5h404347p2254758}{\textbf{(USAMO 2011 P1)}} Let $a, b, c$ be positive real numbers such that $a^2+b^2+c^2+(a+b+c)^2\leq4$. Prove that
	      \[\frac{ab+1}{(a+b)^2}+\frac{bc+1}{(b+c)^2}+\frac{ca+1}{(c+a)^2}\geq 3.\]
	\item [\texttt{\textcolor{color7}{01IMO2}}] \href{https://artofproblemsolving.com/community/c6h17451p119168}{\textbf{(IMO 2001 P2)}} Prove that for all positive real numbers $a,b,c$,\[ \frac{a}{\sqrt{a^2 + 8bc}} + \frac{b}{\sqrt{b^2 + 8ca}} + \frac{c}{\sqrt{c^2 + 8ab}} \geq 1.  \]
	\item [\texttt{\textcolor{color7}{05IMO3}}] \href{https://artofproblemsolving.com/community/c6h44480p281573}{\textbf{(IMO 2005 P3)}} Let $x,y,z$ be three positive reals such that $xyz\geq 1$. Prove that
	      \[ \frac { x^5-x^2 }{x^5+y^2+z^2} + \frac {y^5-y^2}{x^2+y^5+z^2} + \frac {z^5-z^2}{x^2+y^2+z^5} \geq 0 . \]
	\item [\texttt{\textcolor{color6}{12IMO2}}] \href{https://artofproblemsolving.com/community/p2736375}{\textbf{(IMO 2012 P2)}} Let $n\ge 3$ be an integer, and let $a_2,a_3,\ldots ,a_n$ be positive real numbers such that $a_{2}a_{3}\cdots a_{n}=1$. Prove that
	      \[(1 + a_2)^2 (1 + a_3)^3 \dotsm (1 + a_n)^n > n^n.\]
	\item [\texttt{\textcolor{color6}{03ELMO4}}] \href{https://artofproblemsolving.com/community/c6h514107p2887981}{\textbf{(ELMO 2003 P4)}} Let $x,y,z \ge 1$ be real numbers such that\[ \frac{1}{x^2-1} + \frac{1}{y^2-1} + \frac{1}{z^2-1} = 1. \]Prove that\[ \frac{1}{x+1} + \frac{1}{y+1} + \frac{1}{z+1} \le 1. \]
	\item [\texttt{\textcolor{color6}{04IMO4}}] \href{https://artofproblemsolving.com/community/p99756}{\textbf{(IMO 2004 P4)}} Let $n \geq 3$ be an integer. Let $t_1$, $t_2$, ..., $t_n$ be positive real numbers such that\[n^2 + 1 > \left( t_1 + t_2 + \cdots + t_n \right) \left( \frac{1}{t_1} + \frac{1}{t_2} + \cdots + \frac{1}{t_n} \right).\]Show that $t_i$, $t_j$, $t_k$ are side lengths of a triangle for all $i$, $j$, $k$ with $1 \leq i < j < k \leq n$.
	\item [\texttt{\textcolor{color6}{11MOPR42}}] \href{https://artofproblemsolving.com/community/c6h3204552p29271422}{\textbf{(MOP 2011 R4.2)}} For positive real numbers $a, b, c$ with $a + b + c = 3$ prove that
	      \[\sum_{\text{cyc}}\sqrt{\frac{a^3+b^3}{a+b}}+9\sqrt[3]{abc}\le12.\]
	\item [\texttt{\textcolor{color3}{04SLA5}}] \href{https://artofproblemsolving.com/community/c6h30865p191113}{\textbf{(Shortlist 2004 A5)}} If $a$, $b$ ,$c$ are three positive real numbers such that $ab+bc+ca = 1$, prove that\[ \sqrt[3]{ \frac{1}{a} + 6b} + \sqrt[3]{\frac{1}{b} + 6c} + \sqrt[3]{\frac{1}{c} + 6a } \leq \frac{1}{abc}.  \]
	\item [\texttt{\textcolor{color3}{98IRN}}] \href{https://artofproblemsolving.com/community/c6h1100451p4954922}{\textbf{(Iran 1998 P5)}} When $x(\ge1),$ $y(\ge1),$ $z(\ge1)$ satisfy $\frac{1}{x}+\frac{1}{y}+\frac{1}{z}=2,$ prove in equality.
	      $$\sqrt{x+y+z}\ge\sqrt{x-1}+\sqrt{y-1}+\sqrt{z-1}$$
	\item [\texttt{\textcolor{color3}{95IMO2}}] \href{https://artofproblemsolving.com/community/c6h60434p365178}{\textbf{(IMO 1995 P2)}} Let $ a$, $ b$, $ c$ be positive real numbers such that $ abc = 1$. Prove that
	      \[ \frac {1}{a^{3}\left(b + c\right)} + \frac {1}{b^{3}\left(c + a\right)} + \frac {1}{c^{3}\left(a + b\right)}\geq \frac {3}{2}.
	      \]
	\item [\texttt{\textcolor{color3}{12JMO3}}] \href{https://artofproblemsolving.com/community/p2669114}{\textbf{(USAJMO 2012 P3)}} Let $a,b,c$ be positive real numbers. Prove that
	      \[
		      \frac{a^3+3b^3}{5a+b}+\frac{b^3+3c^3}{5b+c}+\frac{c^3+3a^3}{5c+a} \geq \frac{2}{3}(a^2+b^2+c^2)
	      \]
	\item [\texttt{\textcolor{color3}{98SLA3}}] \href{https://artofproblemsolving.com/community/c6h18488p124421}{\textbf{(Shortlist 1998 A3)}} Let $x,y$ and $z$ be positive real numbers such that $xyz=1$. Prove that
	      \[
		      \frac{x^{3}}{(1 + y)(1 + z)}+\frac{y^{3}}{(1 + z)(1 + x)}+\frac{z^{3}}{(1 + x)(1 + y)}
		      \geq \frac{3}{4}.
	      \]
\end{itemize}

\section{\textsf{Elem Geo}}

\begin{itemize}
	\item [\texttt{\textcolor{color7}{21USEMO4}}] \href{https://artofproblemsolving.com/community/c6h2707599p23524100}{\textbf{(USEMO 2021 P4)}} \quad Let $ABC$ be a triangle with circumcircle $\omega$, and let $X$ be the reflection of $A$ in $B$. Line $CX$ meets $\omega$ again at $D$. Lines $BD$ and $AC$ meet at $E$, and lines $AD$ and $BC$ meet at $F$. Let $M$ and $N$ denote the midpoints of $AB$ and $ AC$.
	      Can line $EF$ share a point with the circumcircle of triangle $AMN?$
	\item [\texttt{\textcolor{color7}{18JMO3}}] \href{https://artofproblemsolving.com/community/c5h1629606p10226149}{\textbf{(USAJMO 2018 P3)}} \quad Let $ABCD$ be a quadrilateral inscribed in circle $\omega$ with $\overline{AC} \perp \overline{BD}$. Let $E$ and $F$ be the reflections of $D$ over lines $BA$ and $BC$, respectively, and let $P$ be the intersection of lines $BD$ and $EF$. Suppose that the circumcircle of $\triangle EPD$ meets $\omega$ at $D$ and $Q$, and the circumcircle of $\triangle FPD$ meets $\omega$ at $D$ and $R$. Show that $EQ = FR$.
	\item [\texttt{\textcolor{color7}{17TSTST5}}] \href{https://artofproblemsolving.com/community/c6h1470025p8526136}{\textbf{(USA TSTST 2017 P5)}} \quad Let $ABC$ be a triangle with incenter $I$. Let $D$ be a point on side $BC$ and let $\omega_B$ and $\omega_C$ be the incircles of $\triangle ABD$ and $\triangle ACD$, respectively. Suppose that $\omega_B$ and $\omega_C$ are tangent to segment $BC$ at points $E$ and $F$, respectively. Let $P$ be the intersection of segment $AD$ with the line joining the centers of $\omega_B$ and $\omega_C$. Let $X$ be the intersection point of lines $BI$ and $CP$ and let $Y$ be the intersection point of lines $CI$ and $BP$. Prove that lines $EX$ and $FY$ meet on the incircle of $\triangle ABC$.
	\item [\texttt{\textcolor{color7}{16TSTST2}}] \href{https://artofproblemsolving.com/community/c6h1264174p6575204}{\textbf{(USA TSTST 2016 P2)}} \quad Let $ABC$ be a scalene triangle with orthocenter $H$ and circumcenter $O$. Denote by $M$, $N$ the midpoints of $\overline{AH}$, $\overline{BC}$. Suppose the circle $\gamma$ with diameter $\overline{AH}$ meets the circumcircle of $ABC$ at $G \neq A$, and meets line $AN$ at a point $Q \neq A$. The tangent to $\gamma$ at $G$ meets line $OM$ at $P$. Show that the circumcircles of $\triangle GNQ$ and $\triangle MBC$ intersect at a point $T$ on $\overline{PN}$.
	\item [\texttt{\textcolor{color6}{22USEMO4}}] \href{https://artofproblemsolving.com/community/c6h2946605p26379724}{\textbf{(USEMO 2022 P4)}} \quad Let $ABCD$ be a cyclic quadrilateral whose opposite sides are not parallel. Suppose points $P, Q, R, S$ lie in the interiors of segments $AB, BC, CD, DA,$ respectively, such that$$\angle PDA = \angle PCB, \text{ } \angle QAB = \angle QDC, \text{ } \angle RBC = \angle RAD, \text{ and } \angle SCD = \angle SBA.$$Let $AQ$ intersect $BS$ at $X$, and $DQ$ intersect $CS$ at $Y$. Prove that lines $PR$ and $XY$ are either parallel or coincide.
	\item [\texttt{\textcolor{color6}{23TSTST1}}] \href{https://artofproblemsolving.com/community/c6h2946605p26379724}{\textbf{(USA TSTST 2023 P1)}} \quad Let $ABC$ be a triangle with centroid $G$. Points $R$ and $S$ are chosen on rays $GB$ and $GC$, respectively, such that
	      \[ \angle ABS=\angle ACR=180^\circ-\angle BGC.\]Prove that $\angle RAS+\angle BAC=\angle BGC$.
	\item [\texttt{\textcolor{color6}{16EGMO4}}] \href{https://artofproblemsolving.com/community/c6h1227238p6177803}{\textbf{(EGMO 2016 P4)}} \quad Two circles $\omega_1$ and $\omega_2$, of equal radius intersect at different points $X_1$ and $X_2$. Consider a circle $\omega$ externally tangent to $\omega_1$ at $T_1$ and internally tangent to $\omega_2$ at point $T_2$. Prove that lines $X_1T_1$ and $X_2T_2$ intersect at a point lying on $\omega$.
	\item [\texttt{\textcolor{color6}{11IRNTST1}}] \href{https://artofproblemsolving.com/community/c6h405937p2266382}{\textbf{(Iran TST 2011 P1)}} \quad In acute triangle $ABC$ angle $B$ is greater than$C$. Let $M$ is midpoint of $BC$. $D$ and $E$ are the feet of the altitude from $C$ and $B$ respectively. $K$ and $L$ are midpoint of $ME$ and $MD$ respectively. If $KL$ intersect the line through $A$ parallel to $BC$ in $T$, prove that $TA=TM$.
	\item [\texttt{\textcolor{color6}{06SLG2}}] \href{https://artofproblemsolving.com/community/p10632285}{\textbf{(Shortlist 2006 G2)}} \quad Let $ ABCD$ be a trapezoid with parallel sides $ AB > CD$. Points $ K$ and $ L$ lie on the line segments $ AB$ and $ CD$, respectively, so that $AK/KB=DL/LC$. Suppose that there are points $ P$ and $ Q$ on the line segment $ KL$ satisfying\[\angle{APB} = \angle{BCD}\qquad\text{and}\qquad \angle{CQD} = \angle{ABC}.\]Prove that the points $ P$, $ Q$, $ B$ and $ C$ are concyclic.
	\item [\texttt{\textcolor{color3}{17SLG3}}] \href{https://artofproblemsolving.com/community/c6h1671271p10632285}{\textbf{(Shortlist 2017 G3)}} \quad Let $O$ be the circumcenter of an acute triangle $ABC$. Line $OA$ intersects the altitudes of $ABC$ through $B$ and $C$ at $P$ and $Q$, respectively. The altitudes meet at $H$. Prove that the circumcenter of triangle $PQH$ lies on a median of triangle $ABC$.
	\item [\texttt{\textcolor{color3}{20ELMO4}}] \href{https://artofproblemsolving.com/community/c6h2210480p16724057}{\textbf{(ELMO 2020 P4)}} \quad Let acute scalene triangle $ABC$ have orthocenter $H$ and altitude $AD$ with $D$ on side $BC$. Let $M$ be the midpoint of side $BC$, and let $D'$ be the reflection of $D$ over $M$. Let $P$ be a point on line $D'H$ such that lines $AP$ and $BC$ are parallel, and let the circumcircles of $\triangle AHP$ and $\triangle BHC$ meet again at $G \neq H$. Prove that $\angle MHG = 90^\circ$.
	\item [\texttt{\textcolor{color3}{04IMO1}}] \href{https://artofproblemsolving.com/community/p99445}{\textbf{(IMO 2004 P1)}} \quad Let $ABC$ be an acute-angled triangle with $AB\neq AC$. The circle with diameter $BC$ intersects the sides $AB$ and $AC$ at $M$ and $N$ respectively. Denote by $O$ the midpoint of the side $BC$. The bisectors of the angles $\angle BAC$ and $\angle MON$ intersect at $R$. Prove that the circumcircles of the triangles $BMR$ and $CNR$ have a common point lying on the side $BC$.
	\item [\texttt{\textcolor{color3}{23AMO1}}] \href{https://artofproblemsolving.com/community/c5h3038296p27349297}{\textbf{(USAMO 2023 P1)}} \quad In an acute triangle $ABC$, let $M$ be the midpoint of $\overline{BC}$. Let $P$ be the foot of the perpendicular from $C$ to $AM$. Suppose that the circumcircle of triangle $ABP$ intersects line $BC$ at two distinct points $B$ and $Q$. Let $N$ be the midpoint of $\overline{AQ}$. Prove that $NB=NC$.
	\item [\texttt{\textcolor{color3}{97IMO2}}] \href{https://artofproblemsolving.com/community/p356701}{\textbf{(IMO 1997 P2)}} \quad It is known that $ \angle BAC$ is the smallest angle in the triangle $ ABC$. The points $ B$ and $ C$ divide the circumcircle of the triangle into two arcs. Let $ U$ be an interior point of the arc between $ B$ and $ C$ which does not contain $ A$. The perpendicular bisectors of $ AB$ and $ AC$ meet the line $ AU$ at $ V$ and $ W$, respectively. The lines $ BV$ and $ CW$ meet at $ T$.

	      Show that $ AU = TB + TC$.

\end{itemize}

\section{\textsf{Entry Combo}}

\begin{itemize}
	\item [\texttt{\textcolor{color7}{19JMO1}}] \href{https://artofproblemsolving.com/community/c5h1823554p12189456}{\textbf{(USAJMO 2019 P1)}} There are $a+b$ bowls arranged in a row, numbered $1$ through $a+b$, where $a$ and $b$ are given positive integers. Initially, each of the first $a$ bowls contains an apple, and each of the last $b$ bowls contains a pear.
	      A legal move consists of moving an apple from bowl $i$ to bowl $i+1$ and a pear from bowl $j$ to bowl $j-1$, provided that the difference $i-j$ is even. We permit multiple fruits in the same bowl at the same time. The goal is to end up with the first $b$ bowls each containing a pear and the last $a$ bowls each containing an apple. Show that this is possible if and only if the product $ab$ is even.
	\item [\texttt{\textcolor{color7}{12SLC1}}] \href{https://artofproblemsolving.com/community/c6h546168p3160559}{\textbf{(Shortlist 2012 C1)}} Several positive integers are written in a row. Iteratively, Alice chooses two adjacent numbers $x$ and $y$ such that $x>y$ and $x$ is to the left of $y$, and replaces the pair $(x,y)$ by either $(y+1,x)$ or $(x-1,x)$. Prove that she can perform only finitely many such iterations.
	\item [\texttt{\textcolor{color7}{12AMO4}}] \href{https://artofproblemsolving.com/community/c5h1083477p4774079}{\textbf{(USAMO 2015 P4)}} Steve is piling $m\geq 1$ indistinguishable stones on the squares of an $n\times n$ grid. Each square can have an arbitrarily high pile of stones. After he finished piling his stones in some manner, he can then perform stone moves, defined as follows. Consider any four grid squares, which are corners of a rectangle, i.e. in positions $(i, k), (i, l), (j, k), (j, l)$ for some $1\leq i, j, k, l\leq n$, such that $i<j$ and $k<l$. A stone move consists of either removing one stone from each of $(i, k)$ and $(j, l)$ and moving them to $(i, l)$ and $(j, k)$ respectively, or removing one stone from each of $(i, l)$ and $(j, k)$ and moving them to $(i, k)$ and $(j, l)$ respectively.

	      Two ways of piling the stones are equivalent if they can be obtained from one another by a sequence of stone moves.

	      How many different non-equivalent ways can Steve pile the stones on the grid?
	\item [\texttt{\textcolor{color6}{11AMO2}}] \href{https://artofproblemsolving.com/community/c5h404349p2254765}{\textbf{(USAMO 2011 P2)}} \quad An integer is assigned to each vertex of a regular pentagon so that the sum of the five integers is 2011. A turn of a solitaire game consists of subtracting an integer $m$ from each of the integers at two neighboring vertices and adding $2m$ to the opposite vertex, which is not adjacent to either of the first two vertices. (The amount $m$ and the vertices chosen can vary from turn to turn.) The game is won at a certain vertex if, after some number of turns, that vertex has the number 2011 and the other four vertices have the number 0. Prove that for any choice of the initial integers, there is exactly one vertex at which the game can be won.
	\item [\texttt{\textcolor{color6}{19CAN3}}] \href{https://artofproblemsolving.com/community/c6h1811674p12071600}{\textbf{(Canadian MO 2019 P3)}} \quad You have a $2m$ by $2n$ grid of squares coloured in the same way as a standard checkerboard. Find the total number of ways to place $mn$ counters on white squares so that each square contains at most one counter and no two counters are in diagonally adjacent white squares.
	\item [\texttt{\textcolor{color6}{21JMO4}}] \href{https://artofproblemsolving.com/community/c5h2529924p21498566}{\textbf{(USAJMO 2021 P4)}} \quad Carina has three pins, labeled $A, B$, and $C$, respectively, located at the origin of the coordinate plane. In a move, Carina may move a pin to an adjacent lattice point at distance $1$ away. What is the least number of moves that Carina can make in order for triangle $ABC$ to have area 2021?
	      (A lattice point is a point $(x, y)$ in the coordinate plane where $x$ and $y$ are both integers, not necessarily positive.)
	\item [\texttt{\textcolor{color6}{08AMO4}}] \href{https://artofproblemsolving.com/community/c5h202905p1116177}{\textbf{(USAMO 2008 P4)}} \quad Let $ \mathcal{P}$ be a convex polygon with $ n$ sides, $ n\ge3$. Any set of $ n - 3$ diagonals of $ \mathcal{P}$ that do not intersect in the interior of the polygon determine a triangulation of $ \mathcal{P}$ into $ n - 2$ triangles. If $ \mathcal{P}$ is regular and there is a triangulation of $ \mathcal{P}$ consisting of only isosceles triangles, find all the possible values of $ n$.
	\item [\texttt{\textcolor{color3}{21PAGMO5}}] \href{https://artofproblemsolving.com/community/c6h2687954p23322931}{\textbf{(PAGMO 2021 P5)}} \quad Celeste has an unlimited amount of each type of $n$ types of candy, numerated type 1, type 2, ... type n. Initially she takes $m>0$ candy pieces and places them in a row on a table. Then, she chooses one of the following operations (if available) and executes it:

	      $1.$ She eats a candy of type $k$, and in its position in the row she places one candy type $k-1$ followed by one candy type $k+1$ (we consider type $n+1$ to be type 1, and type 0 to be type $n$).

	      $2.$ She chooses two consecutive candies which are the same type, and eats them.

	      Find all positive integers $n$ for which Celeste can leave the table empty for any value of $m$ and any configuration of candies on the table.
	\item [\texttt{\textcolor{color3}{25XOOK1}}] \href{https://artofproblemsolving.com/community/c6h3497045p33958997}{\textbf{(XOOK 2025 P1)}} \quad For the Exeter disco party, Oron wants to tile a $2024 \times 2024$ square in the middle of the dance floor with $1 \times 1$ tiles which have arrows pointing either up, left, right, or down.

	      After placing down all the tiles in some configuration, he realizes that he wants to rotate some of the tiles so that no matter which tile a dancer starts on, after following the arrows they end up leaving the grid through some fixed cell. What's the minimum number of tiles he needs to rotate so that this is always possible?
	\item [\texttt{\textcolor{color3}{18SLN1}}] \href{https://artofproblemsolving.com/community/p12752840}{\textbf{(Shortlist 2018 N1)}} \quad Determine all pairs $(n, k)$ of distinct positive integers such that there exists a positive integer $s$ for which the number of divisors of $sn$ and of $sk$ are equal.
	\item [\texttt{\textcolor{color3}{74IMO4}}] \href{https://artofproblemsolving.com/community/c6h58591p357975}{\textbf{(IMO 1974 P4)}} \quad Consider decompositions of an $8\times 8$ chessboard into $p$ non-overlapping rectangles subject to the following conditions:
	      (i) Each rectangle has as many white squares as black squares.
	      (ii) If $a_i$ is the number of white squares in the $i$-th rectangle, then $a_1<a_2<\ldots <a_p$.
	      Find the maximum value of $p$ for which such a decomposition is possible. For this value of $p$, determine all possible sequences $a_1,a_2,\ldots ,a_p$.
	\item [\texttt{\textcolor{color3}{18EGMO4}}] \href{https://artofproblemsolving.com/community/c6h1625929p10191585}{\textbf{(EGMO 2018 P4)}} \quad A domino is a $ 1 \times 2 $ or $ 2 \times 1 $ tile.
	      Let $n \ge 3 $ be an integer. Dominoes are placed on an $n \times n$ board in such a way that each domino covers exactly two cells of the board, and dominoes do not overlap. The value of a row or column is the number of dominoes that cover at least one cell of this row or column. The configuration is called balanced if there exists some $k \ge 1 $ such that each row and each column has a value of $k$. Prove that a balanced configuration exists for every $n \ge 3 $, and find the minimum number of dominoes needed in such a configuration.
\end{itemize}

\section{\textsf{Mod Artith 1: Proof}}

\begin{itemize}
	\item [\texttt{\textcolor{color6}{PENH61}}] \href{https://artofproblemsolving.com/community/p409056}{\textbf{(PEN H61)}} \quad Find all positive integer solutions to $2^x - 5 = 11^y$.
	\item [\texttt{\textcolor{color6}{12JMO5}}] \href{https://artofproblemsolving.com/community/c5h476849p2669967}{\textbf{(USAJMO 2012 P5)}} \quad For distinct positive integers $a, b<2012$, define $f(a, b)$ to be the number of integers $k$ with $1\le k<2012$ such that the remainder when $ak$ divided by $2012$ is greater than that of $bk$ divided by $2012$. Let $S$ be the minimum value of $f(a, b)$, where $a$ and $b$ range over all pairs of distinct positive integers less than $2012$. Determine $S$.
	\item [\texttt{\textcolor{color3}{25AMO1}}] \href{https://artofproblemsolving.com/community/p34326777}{\textbf{(USAMO 2025 P1)}} \quad Let $k$ and $d$ be positive integers. Prove that there exists a positive integer $N$ such that for every odd integer $n>N$, the digits in the base-$2n$ representation of $n^k$ are all greater than $d$.
	\item [\texttt{\textcolor{color3}{17TSTST4}}] \href{https://artofproblemsolving.com/community/p8526131}{\textbf{(USA TSTST 2017 P4)}} \quad Find all nonnegative integer solutions to $2^a + 3^b + 5^c = n!$.
	\item [\texttt{\textcolor{color3}{06IMO4}}] \href{https://artofproblemsolving.com/community/p572815}{\textbf{(IMO 2006 P4)}} \quad Determine all pairs $(x, y)$ of integers such that\[1+2^{x}+2^{2x+1}= y^{2}.\]
	\item [\texttt{\textcolor{color3}{20IBERO2}}] \href{https://artofproblemsolving.com/community/c5h2808845p24774800}{\textbf{(Iberoamerican 2020 P2)}} \quad Let $T_n$ denotes the least natural such that
	      $$n\mid 1+2+3+\cdots +T_n=\sum_{i=1}^{T_n} i$$Find all naturals $m$ such that $m\ge T_m$.
	\item [\texttt{\textcolor{color3}{04SLN6}}] \href{https://artofproblemsolving.com/community/c6h41072p258123}{\textbf{(Shortlist 2004 N6)}} \quad Given an integer ${n>1}$, denote by $P_{n}$ the product of all positive integers $x$ less than $n$ and such that $n$ divides ${x^2-1}$. For each ${n>1}$, find the remainder of $P_{n}$ on division by $n$.
	\item [\texttt{\textcolor{color3}{25JMO6}}] \href{https://artofproblemsolving.com/community/c5h3532099p34335894}{\textbf{(USAJMO 2025 P6)}} \quad Let $S$ be a set of integers with the following properties:
	      \begin{itemize}
		      \item $\{ 1, 2, \dots, 2025 \} \subseteq S$.
		      \item If $a, b \in S$ and $\gcd(a, b) = 1$, then $ab \in S$.
		      \item If for some $s \in S$, $s + 1$ is composite, then all positive divisors of $s + 1$ are in $S$.
	      \end{itemize}
\end{itemize}

\end{document}
