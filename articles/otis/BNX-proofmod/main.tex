\documentclass[12pt]{article}
\usepackage{../../sty/otis}

\unit{\textsf{\textbf{Mod Arith 1: Proofs}}}

\title{\unit}
\author{Samuel de Araújo Brandão}
\date{\today}

\begin{document}
\maketitle
\preface

\tableofcontents

\clearpage

\section{\textsf{Practice Problems}}
\begin{itemize}
	\item [\texttt{\textcolor{color6}{PENH61}}] \href{https://artofproblemsolving.com/community/p409056}{\textbf{(PEN H61)}} \quad Find all positive integer solutions to $2^x - 5 = 11^y$.
	\item [\texttt{\textcolor{color6}{12JMO5}}] \href{https://artofproblemsolving.com/community/c5h476849p2669967}{\textbf{(USAJMO 2012 P5)}} \quad For distinct positive integers $a, b<2012$, define $f(a, b)$ to be the number of integers $k$ with $1\le k<2012$ such that the remainder when $ak$ divided by $2012$ is greater than that of $bk$ divided by $2012$. Let $S$ be the minimum value of $f(a, b)$, where $a$ and $b$ range over all pairs of distinct positive integers less than $2012$. Determine $S$.
	\item [\texttt{\textcolor{color3}{25AMO1}}] \href{https://artofproblemsolving.com/community/p34326777}{\textbf{(USAMO 2025 P1)}} \quad Let $k$ and $d$ be positive integers. Prove that there exists a positive integer $N$ such that for every odd integer $n>N$, the digits in the base-$2n$ representation of $n^k$ are all greater than $d$.
	\item [\texttt{\textcolor{color3}{17TSTST4}}] \href{https://artofproblemsolving.com/community/p8526131}{\textbf{(USA TSTST 2017 P4)}} \quad Find all nonnegative integer solutions to $2^a + 3^b + 5^c = n!$.
	\item [\texttt{\textcolor{color3}{06IMO4}}] \href{https://artofproblemsolving.com/community/p572815}{\textbf{(IMO 2006 P4)}} \quad Determine all pairs $(x, y)$ of integers such that\[1+2^{x}+2^{2x+1}= y^{2}.\]
	\item [\texttt{\textcolor{color3}{20IBERO2}}] \href{https://artofproblemsolving.com/community/c5h2808845p24774800}{\textbf{(Iberoamerican 2020 P2)}} \quad Let $T_n$ denotes the least natural such that
	      $$n\mid 1+2+3+\cdots +T_n=\sum_{i=1}^{T_n} i$$Find all naturals $m$ such that $m\ge T_m$.
	\item [\texttt{\textcolor{color3}{04SLN6}}] \href{https://artofproblemsolving.com/community/c6h41072p258123}{\textbf{(Shortlist 2004 N6)}} \quad Given an integer ${n>1}$, denote by $P_{n}$ the product of all positive integers $x$ less than $n$ and such that $n$ divides ${x^2-1}$. For each ${n>1}$, find the remainder of $P_{n}$ on division by $n$.
	\item [\texttt{\textcolor{color3}{25JMO6}}] \href{https://artofproblemsolving.com/community/c5h3532099p34335894}{\textbf{(USAJMO 2025 P6)}} \quad Let $S$ be a set of integers with the following properties:
	      \begin{itemize}
		      \item $\{ 1, 2, \dots, 2025 \} \subseteq S$.
		      \item If $a, b \in S$ and $\gcd(a, b) = 1$, then $ab \in S$.
		      \item If for some $s \in S$, $s + 1$ is composite, then all positive divisors of $s + 1$ are in $S$.
	      \end{itemize}
\end{itemize}

\clearpage

\section{\textsf{Solutions}}

\subsection{Lecture Notes}

\clearpage

\subsection{Mandatory}

\subsubsection{PEN H61}
\begin{problem}
Find all positive integer solutions to $2^x - 5 = 11^y$.
\end{problem}

\subsubsection{USAJMO 2012 P5}
\begin{problem}
For distinct positive integers $a, b<2012$, define $f(a, b)$ to be the number of integers $k$ with $1\le k<2012$ such that the remainder when $ak$ divided by $2012$ is greater than that of $bk$ divided by $2012$. Let $S$ be the minimum value of $f(a, b)$, where $a$ and $b$ range over all pairs of distinct positive integers less than $2012$. Determine $S$.
\end{problem}

\clearpage
\subsection{Not mandatory}

\subsubsection{USAMO 2025 P1}
\begin{problem}
Let $k$ and $d$ be positive integers. Prove that there exists a positive integer $N$ such that for every odd integer $n>N$, the digits in the base-$2n$ representation of $n^k$ are all greater than $d$.
\end{problem}

\clearpage

\subsubsection{USA TSTST 2017 P4}
\begin{problem}
Find all nonnegative integer solutions to $2^a + 3^b + 5^c = n!$.
\end{problem}

\clearpage

\subsubsection{IMO 2006 P4}
\begin{problem}
Find all nonnegative integer solutions to $2^a + 3^b + 5^c = n!$.
\end{problem}

\clearpage

\subsubsection{Iberoamerican 2020 P2}
\begin{problem}
Let $T_n$ denotes the least natural such that
$$n\mid 1+2+3+\cdots +T_n=\sum_{i=1}^{T_n} i$$Find all naturals $m$ such that $m\ge T_m$.
\end{problem}

\clearpage


\subsubsection{Shortlist 2004 N6}
\begin{problem}
Given an integer ${n>1}$, denote by $P_{n}$ the product of all positive integers $x$ less than $n$ and such that $n$ divides ${x^2-1}$. For each ${n>1}$, find the remainder of $P_{n}$ on division by $n$.
\end{problem}

\clearpage

\subsubsection{USAJMO 2025 P6}
\begin{problem}
Let $S$ be a set of integers with the following properties:
\begin{itemize}
	\item $\{ 1, 2, \dots, 2025 \} \subseteq S$.
	\item If $a, b \in S$ and $\gcd(a, b) = 1$, then $ab \in S$.
	\item If for some $s \in S$, $s + 1$ is composite, then all positive divisors of $s + 1$ are in $S$.
\end{itemize}
Prove that $S$ contains all positive integers.

\end{problem}

\clearpage


\section{\textsf{References}}\label{sec:references}
\end{document}
