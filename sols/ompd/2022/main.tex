\documentclass[12pt]{article}
\usepackage{../sty/articlept}

\olympiad{OMpD 2022 N2 Fase 2}

\title{\olympiad}
\author{Samuel de Araújo Brandão}
\date{\today}

\begin{document}
\maketitle
\preface

\tableofcontents

\clearpage

\section{Problemas}
\begin{enumerate}[label={\textbf{\arabic*.}}]
	\item Considere um tabuleiro quadriculado $6 \times 6$, formado de $36$ casinhas unitárias. Desejamos colocar
	      $6$ torres neste tabuleiro, uma torre em cada casinha, de modo que não haja duas torres numa mesma linha,
	      nem duas torres numa mesma coluna. Note que, uma vez colocadas as torres dessa maneira, temos que, para toda
	      casinha onde não foi colocada uma torre, há uma torre na mesma linha que ela e uma torre na mesma coluna que ela.
	      Diremos que tais torres são alinhadas com essa casinha.

	      Para cada uma dessas $30$ casas sem torres, pinte-a de verde se as duas torres alinhadas com essa mesma casinha
	      estão a uma mesma distância dela, e pinte-a de amarelo caso contrário. Por exemplo, ao colocarmos as $6$ torres
	      (T) como abaixo, temos:

	      \begin{figure}[h]
		      \centering
		      \includegraphics[width=0.3\textwidth]{img/1.png}
	      \end{figure}

	      \begin{enumerate}[label=({\alph*})]
		      \item É possível colocarmos as torres de modo que haja $30$ casas verdes?
		      \item É possível colocarmos as torres de modo que haja $30$ casas amarelas?
		      \item É possível colocarmos as torres de modo que haja $15$ casas verdes e $15$ amarelas?
	      \end{enumerate}
	\item Seja $ABCD$ um retângulo. O ponto $E$ está sobre o lado $\overline{AB}$ e o ponto $F$ está sobre o lado
	      $\overline{AD}$, de modo que $\angle FEC = \angle CEB$ e $\angle DFC = \angle CFE$.

	      Determine a medida do ângulo $\angle FCE$ e da razão $AD/AB$.
	\item Seja $N$ um inteiro positivo. Inicialmente, um número inteiro positivo $A$ está escrito no quadro.
	      A cada passo, podemos realizar uma das duas operações seguintes com o número escrito no quadro:

	      \begin{enumerate}[label=(\roman*)]
		      \item Somar $N$ ao número escrito no quadro e trocar tal número pela soma obtida;
		      \item Se o número no quadro for maior do que $1$ e tiver pelo menos um algarismo $1$, então
		            podemos remover o algarismo $1$ desse número, e trocar o número escrito inicialmente por este
		            (com remoção de possíveis zeros à esquerda).
	      \end{enumerate}

	      Por exemplo, se $N = 63$ e $A = 25$, podemos fazer a seguinte sequência de operações:
	      \[
		      25 \to 88 \to 151 \to 51 \to 5
	      \]

	      E se $N = 143$ e $A = 2$, podemos fazer a seguinte sequência de operações:
	      \[
		      2 \to 145 \to 288 \to 431 \to 574 \to 717 \to 860 \to 1003 \to 3
	      \]

	      Para quais valores de $N$ sempre é possível, não importando o valor inicial de $A$ na lousa, obter
	      o número $1$ na lousa, mediante um número finito de operações?

	\item Dizemos que uma sêxtupla de números reais positivos $(a_1, a_2, a_3, b_1, b_2, b_3)$ é \textit{phika}
	      se $a_1 + a_2 + a_3 = b_1 + b_2 + b_3 = 1$.
	      \begin{enumerate}[label=({\alph*})]
		      \item Prove que existe uma sêxtupla phika $(a_1, a_2, a_3, b_1, b_2, b_3)$ tal que:
		            \[
			            a_1(\sqrt{b_1} + a_2) + a_2(\sqrt{b_2} + a_3) + a_3(\sqrt{b_3} + a_1) > 1 - \frac{1}{2022^{2022}}.
		            \]

		      \item Prove que para toda sêxtupla phika $(a_1, a_2, a_3, b_1, b_2, b_3)$, temos:
		            \[
			            a_1(\sqrt{b_1} + a_2) + a_2(\sqrt{b_2} + a_3) + a_3(\sqrt{b_3} + a_1) < 1.
		            \]
	      \end{enumerate}
\end{enumerate}

\clearpage

\section{Soluções}
\subsection{Problema 1.}

\begin{problem}{}{}
Considere um tabuleiro quadriculado $6 \times 6$, formado de $36$ casinhas unitárias. Desejamos colocar
$6$ torres neste tabuleiro, uma torre em cada casinha, de modo que não haja duas torres numa mesma linha,
nem duas torres numa mesma coluna. Note que, uma vez colocadas as torres dessa maneira, temos que, para toda
casinha onde não foi colocada uma torre, há uma torre na mesma linha que ela e uma torre na mesma coluna que ela.
Diremos que tais torres são alinhadas com essa casinha.

Para cada uma dessas $30$ casas sem torres, pinte-a de verde se as duas torres alinhadas com essa mesma casinha
estão a uma mesma distância dela, e pinte-a de amarelo caso contrário. Por exemplo, ao colocarmos as $6$ torres
(T) como abaixo, temos:

\begin{center}
	\includegraphics[width=0.3\textwidth]{img/1.png}
\end{center}

\begin{enumerate}[label=({\alph*})]
	\item É possível colocarmos as torres de modo que haja $30$ casas verdes?
	\item É possível colocarmos as torres de modo que haja $30$ casas amarelas?
	\item É possível colocarmos as torres de modo que haja $15$ casas verdes e $15$ amarelas?
\end{enumerate}
\end{problem}

\begin{enumerate}[label=\textbf{({\alph*})}]
	\item Sim, como pode-se ver a seguinte configuração
	      \begin{center}
		      \includegraphics[width=0.3\textwidth]{img/2.png}
	      \end{center}
	\item Sim, como pode-se ver a seguinte configuração
	      \begin{center}
		      \includegraphics[width=0.3\textwidth]{img/3.png}
	      \end{center}
	\item Não. Já que é impossível encontrar um número ímpar de casas verdes.

	      Pode-se provar tal fato ao perceber que existe apenas uma maneira de formar casas verdes:
	      alinhar duas torres diagonalmente, o que resulta em um par de casas verdes únicos, que não
	      podem ser conquistados por meio de nenhuma outra alinhação de torres na diagonal. Ou seja,
	      haverá sempre um número par de casas verdes no tabuleiro, impossibilitando haver 15 casas
	      verdes em tal.
\end{enumerate}

\clearpage

\subsection{Problema 2.}

\begin{problem}{}{}
Seja $ABCD$ um retângulo. O ponto $E$ está sobre o lado $\overline{AB}$ e o ponto $F$ está sobre o lado
$\overline{AD}$, de modo que $\angle FEC = \angle CEB$ e $\angle DFC = \angle CFE$.

Determine a medida do ângulo $\angle FCE$ e da razão $AD/AB$.
\end{problem}

$\angle FCE = 45^{\circ}$ e $\frac{AB}{AD} = 1$.

É possível encontrar a medida de $\angle FCE$ notando que $\overline{CF}$ é a bissetriz de $\angle EFD$ e $\overline{CE}$
é a bissetriz de $\angle BEF$. Já que o ângulo interno total de EFDCB é $540^{\circ}$ e $\angle FDC + \angle DCB + \angle CBE = 270^{\circ}$,
pode-se afirmar que $\angle BEF + \angle EFD = 270^{\circ}$, logo, $\angle CEF + \angle EFC = 135^{\circ} \iff \angle FCE = 45^{\circ}$.

\begin{figure}[h]
	\centering
	\includegraphics[width=0.3\textwidth]{img/4.png}
\end{figure}

Visando encontrar o resultado de $\frac{AD}{AB}$, digamos que $H$ é o pé da altura $\overline{CH}$ relativa a $\overline{EF}$. Podemos afirmar que
$\triangle CBE \cong \triangle CEH$ pelo caso $ALA$. Assim como podemos afirmar, pelo mesmo caso, que $\triangle CHF \cong \triangle CFD$. Essas
similaridades resultam em $\overline{BC} = \overline{CD}$, ou seja, $\overline{AB} = \overline{AD} \iff \frac{AB}{AD}=1$.
\clearpage

\subsection{Problema 3.}

\begin{problem}{}{}
Seja $N$ um inteiro positivo. Inicialmente, um número inteiro positivo $A$ está escrito no quadro.
A cada passo, podemos realizar uma das duas operações seguintes com o número escrito no quadro:

\begin{enumerate}[label=(\roman*)]
	\item Somar $N$ ao número escrito no quadro e trocar tal número pela soma obtida;
	\item Se o número no quadro for maior do que $1$ e tiver pelo menos um algarismo $1$, então
	      podemos remover o algarismo $1$ desse número, e trocar o número escrito inicialmente por este
	      (com remoção de possíveis zeros à esquerda).
\end{enumerate}

Por exemplo, se $N = 63$ e $A = 25$, podemos fazer a seguinte sequência de operações:
\[
	25 \to 88 \to 151 \to 51 \to 5
\]

E se $N = 143$ e $A = 2$, podemos fazer a seguinte sequência de operações:
\[
	2 \to 145 \to 288 \to 431 \to 574 \to 717 \to 860 \to 1003 \to 3
\]

Para quais valores de $N$ sempre é possível, não importando o valor inicial de $A$ na lousa, obter
o número $1$ na lousa, mediante um número finito de operações?
\end{problem}

Para todos os números $N$ que satisfazem o enunciado e $\gcd(10,N)=1$, é sempre possível
obter o número $1$ na lousa.

\begin{claim}
	$N$ não é múltiplo nem de $2$, nem de $5$.
\end{claim}

\begin{proof}
	Caso $A$ seja um número par, $N$ certamente não pode ser múltiplo de $2$, já que, caso tal fosse
	possível, nem (i) nem (ii) seriam capazes de chegar à um número ímpar, pois toda soma de
	números pares resulta em números pares. Sabendo que $1$ é ímpar, é impossível que $N$ seja múltiplo
	de 2.

	O mesmo serve para $N$ múltiplo de 5. Tal é impossível, porque se tanto $N$ quanto $A$ fossem
	múltiplos de $5$, o número escrito no lousa jamais seria $1$, pois continuará para
	sempre sendo múltiplo de $5$, ja que 1 não é múltiplo de $5$ e toda soma de números
	múltiplos de $5$ resultam em múltiplos de $5$. Com isso, podemos afirmar que $\gcd(10,N)=1$.
\end{proof}

\begin{claim}
	Para algum $k \ge 1$, é certo que $10^k \equiv 1 \pmod N$
\end{claim}

\begin{proof}
	Imagine o conjunto \{$10^1$, $10^2$, \ldots, $10^{N + 1}$\}. É certo que, entre
	os números desse conjunto, pelo princípio da casa dos pombos, existem
	dois números de resto equivalente na divisão por $N$, digamos $10^x$ e $10^y$.
	SPDG, $x > y$. Perceba que $10^x - 10^y \equiv 0 \pmod N$. Sabe-se que
	$10^x - 10^y = 10^y(10^{x-y} - 1)$, mas já que $N$ não é múltiplo de
	$10$, pode-se afirmar que $N$ é múltiplo de $10^{x-y} - 1$. \qedhere
\end{proof}

Agora, deve-se perceber que, ao somar o número $N$ repetidas vezes, encontraremos
o número $10^k-1 + A$, que é composto por $1$, seguido de vários números
$0$, e no final, um número que nomearemos de $B$. Já que $B \equiv A-1 \pmod N$,
depois de apagar o primeiro número $1$ da esquerda para direita, teremos escrito
no lousa o número $B$. Após realizarmos esse processo sucessivamente, encontraremos
o número $0$.

\begin{claim}
	$N$ é múltiplo de $111 \ldots 1$.
\end{claim}

\begin{proof}
	Como visto na alegação anterior, $10^k \equiv 1 \pmod N$, portanto, $10^k-1$
	é múltiplo de $N$, assim como $\frac{10^k-1}{9} = 111\ldots1$.
\end{proof}

Portanto, agora basta somarmos o número $N$ repetidas vezes até encontrarmos o número
$111\ldots1$, e proceder apagando todos os números $1$, com exceção de apenas um deles.
\clearpage

\subsection{Problema 4.}

\begin{problem}{}{}
Dizemos que uma sêxtupla de números reais positivos $(a_1, a_2, a_3, b_1, b_2, b_3)$ é \textit{phika}
se $a_1 + a_2 + a_3 = b_1 + b_2 + b_3 = 1$.
\begin{enumerate}[label=({\alph*})]
	\item Prove que existe uma sêxtupla phika $(a_1, a_2, a_3, b_1, b_2, b_3)$ tal que:
	      \[
		      a_1(\sqrt{b_1} + a_2) + a_2(\sqrt{b_2} + a_3) + a_3(\sqrt{b_3} + a_1) > 1 - \frac{1}{2022^{2022}}.
	      \]

	\item Prove que para toda sêxtupla phika $(a_1, a_2, a_3, b_1, b_2, b_3)$, temos:
	      \[
		      a_1(\sqrt{b_1} + a_2) + a_2(\sqrt{b_2} + a_3) + a_3(\sqrt{b_3} + a_1) < 1.
	      \]
\end{enumerate}
\end{problem}

\begin{enumerate}[label=\textbf{({\alph*})}]
	\item Imagine que $(a_1, a_2, a_3, b_1, b_2, b_3)$ são números naturais. Nesse caso,
	      poderiamos afirmar que $a_1=0$,$a_2=0$,$a_3=1$, assim como $b_1=0$,$b_2=0$,$b_3=0$.
	      Nesse caso,
	      \[
		      a_1(\sqrt{b_1} + a_2) + a_2(\sqrt{b_2} + a_3) + a_3(\sqrt{b_3} + a_1)=1
	      \]
	      Já que $(a_1, a_2, a_3, b_1, b_2, b_3)$ são, na verdade, números naturais,
	      basta imaginar números $a_1$,$a_2$,$b_3$,$b_4$ ridiculamente maiores
	      que $0$, ou seja, muito perto de $0$. Podemos dizer também que
	      $a_3$,$b_3$ são muito, muito próximos de $1$. Desse modo, por mais que
	      a soma não seja equivalente a $1$, podemos dizer que é muito próxima de $1$.

	      \begin{observation}{}{}
		      O fato de 1 ser dividido por $2022^{2022}$ não é importante, já que simplesmente
		      ilustra o quão próximo de $1$ o número $1 - \frac{1}{2022^{2022}}$ é.
	      \end{observation}
	\item
	      Pela desigualdade AM-GM, sabe-se que
	      \[
		      \frac{a^2_i + b_i}{2} \ge a_i\sqrt{b_i}.
	      \]

	      Portanto,
	      \begin{align*}
		      a_1\sqrt{b_1} + a_2\sqrt{b_2} + a_3\sqrt{b_3} + a_1a_2 + a_2a_3 + a_3a_1 & \le
		      \frac{a^2_1 + b_1}{2} + \frac{a^2_2 + b_2}{2} + \frac{a^2_3 + b_3}{2} +                                                          \\
		                                                                               & \quad \ a_1a_2 + a_2a_3 + a_3a_1                      \\
		                                                                               & = \frac{a^2_1 + b_1 + a^2_2 + b_2 + a^2_3 + b_3}{2} + \\
		                                                                               & \quad \ \frac{2a_1a_2 + 2a_2a_3 + 2a_3a_1}{2}         \\
		                                                                               & = \frac{(a_1 + a_2 + a_3)^2 + (b_1 + b_2 + b_3)}{2}   \\
		                                                                               & = 1.
	      \end{align*}
	      Contudo, pode-se ver facilmente que não existe igualdade, já que, se ouvesse, $1 = a_1 + a_2 + a_3 = a^2_1 + a^2_2 + a^3_1$,
	      um absurdo, pois $0 < a_1,a_2,a_3 < 1 \implies a_1 > a^2_1, a_2 > a^2_2, a_3 > a^2_3$.
\end{enumerate}

\clearpage

\section{Referências}\label{sec:references}
\end{document}
