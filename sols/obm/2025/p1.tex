\documentclass[12pt]{article}
\usepackage{../../../sty/solspt}

\usepackage{csquotes}

\olympiad{\textsf{\textbf{OBM 2025 Problema 1}}}

\title{\olympiad}
\author{Samuel de Araújo Brandão}
\date{\today}

\begin{document}
\maketitle

Eu acredito que minha solução para o Problema 1 mereça ser avaliada em \textbf{10
	pontos}. Abaixo encontra-se o motivo.

Para solucionar o primeiro problema da OBM 2025, utilizei da terceira solução,
apresentada no critério de solução. É evidente que os critérios (a), (b), (c) e
(d) foram satisfeitos. Porém, imagino que o corretor(a) tenha considerado que
o critério (e) não fora satisfeito com excelência. Contudo, argumento que no
penúltimo parágrafo de minha solução, tal critério tenha sido satisfeito sim.

Compreendo que minha estratégia difere levemente daquela utilizada na solução
oficial, já que não separei explicitamente a solução em $z > 0$ e $z = 0$. Contudo,
acredito que isso não interfere na qualidade de minha solução, pois deixei claro
o que ocorre quando $z=0$.

Sobre $z>1$, eu já havia deixado claro anteriormente que uma potência de $2$ deve
ser obviamente par, portanto, a casa das unidades não pode equivaler a $1$. Abaixo
encontram-se os dois trechos que mencionei tal fato.

\begin{displayquote}
	Portanto, ao dividir essa potência de $2$ por $2$, encontraremos $PM=5$ e $UN=1$
	($UN=6$ é inviável, já que culminaria em outro dígito $1$ quando $\alpha=3$,
	pois $6 \cdot 2=12$). Contradição! $UN$ é obrigatoriamente par para formar uma
	potência de $2$.
\end{displayquote}

O trecho acima pode ser encotrado no quinto parágrafo, onde provo que $\alpha=3$
é impossível. O seguinte trecho, pode ser encontrado no sexto parágrafo, onde
demonstro que $\alpha=4$ também é impossível

\begin{displayquote}
	$PM=UN=2$ é visívelmente impossível pois, ao dividir por $2$, $UN=1$ pelo mesmo
	motivo já visto.
\end{displayquote}

Portanto, pode-se ver que a impossível existência de $UN=1$ para uma potência de
$2$ já era um fato conhecido por mim. Logo, acredito que minha solução possa ser
avaliada em \textbf{10 pontos}, como citei anteriormente.

\begin{figure}[h]
	\centering
	\includegraphics[width=1\textwidth]{../../../img/sols/obm/7.png}
\end{figure}


\end{document}
