\documentclass[12pt]{article}
\usepackage{../../../sty/solspt}

\olympiad{OBM 2017 N2}

\title{\olympiad}
\author{Samuel de Araújo Brandão}
\date{\today}

\begin{document}
\maketitle
\preface

\tableofcontents

\clearpage

\section{Problemas}

\begin{enumerate}[label={\textbf{\arabic*.}}]
	\item Os pontos $X$, $Y$ e $Z$ estão marcados nos lados $AB$, $BC$ e $AC$ do triângulo $ABC$, respectivamente. Os pontos $A'$, $B'$ e $C'$ estão nos lados $XZ$, $XY$ e $YZ$ do triângulo $XYZ$, respectivamente, de modo que
	      \[
		      \frac{AB}{A'B'}=\frac{AC}{A'C'}=\frac{BC}{B'C'}=2
	      \]
	      e $ABB'A'$, $BCC'B'$ e $ACC'A'$ são trapézios em que os lados do triângulo $ABC$ são bases.
	      \begin{itemize}
		      \item[(a)] Determine a razão entre a área do trapézio $ABB'A'$ e a área do \\ triângulo $A'B'X$.
		      \item[(b)] Determine a razão entre a área do triângulo $XYZ$ e a área do triângulo $ABC$.
	      \end{itemize}
	\item Sabemos que o número real $C$ e números reais não-nulos $x$, $y$ e $z$, dois a dois distintos, satisfazem:
	      \[
		      x+\frac{y}{z}+\frac{z}{y}=y+\frac{z}{x}+\frac{x}{z}=z+\frac{x}{y}+\frac{y}{x}=C.
	      \]
	      \begin{itemize}
		      \item[(a)] Mostre que $C=-1$;
		      \item[(b)] Exiba pelo menos uma solução $(x,y,z)$ para a equação dada.
	      \end{itemize}
	\item Seja $n>1$ um inteiro e considere um tabuleiro $n\times n$, em que algumas das $n^2$ casas foram pintadas de preto, e as restantes foram pintadas de branco. Prove que é possível escolhermos uma das $n^2$ casas do tabuleiro, de modo que, ao removermos completamente a linha e a coluna que a contém, haja um número diferente de casas pretas e de casas brancas, dentre as $(n-1)^2$ casas restantes.
	\item Na Terra dos Impas, somente os algarismos ímpares são utilizados para contar e escrever números. Assim, em vez dos números $1,2,3,4,5,6,7,8,9,10,11,12,\ldots$ os Impas têm os números correspondentes $1,3,5,7,9,11,13,15,17,19,31,33,\ldots$ (note que os números dos Impas têm somente algarismos ímpares). Por exemplo, se uma criança tem $11$ anos, os Impas diriam que ela tem $31$ anos.
	      \begin{itemize}
		      \item[(a)] Como os Impas escrevem o nosso número $20$?
		      \item[(b)] Numa escola desse lugar, a professora escreveu no quadro-negro a continha de multiplicar abaixo. Se você fosse um aluno Impa, o que escreveria como resultado?
		            \[
			            13\times 5
		            \]
		      \item[(c)] Escreva, na linguagem dos Impas, o número que na nossa representação decimal é escrito como $2017$.
	      \end{itemize}
	\item No triângulo $ABC$, com $AB\ne AC$, seja $I$ seu incentro. Os pontos $P$ e $Q$ são definidos como os pontos onde o circuncírculo do triângulo $BCI$ intersecta novamente as retas $AB$ e $AC$, respectivamente. Seja $D$ o ponto de interseção de $AI$ e $BC$.
	      \begin{itemize}
		      \item[(a)] Prove que $P$, $Q$ e $D$ são colineares;
		      \item[(b)] Sendo $T$, diferente de $P$, o ponto de encontro dos circuncírculos dos triângulos $PDB$ e $QDC$, prove que $T$ está no circuncírculo do triângulo $ABC$.
	      \end{itemize}
	      Observação: O Incentro de um triângulo é o ponto de interseção de suas bissetrizes internas e o Circuncírculo de um triângulo é a circunferência que passa pelos seus três vértices.
	\item Demonstre que, para todo $n$ inteiro positivo, existem inteiros positivos $a$ e $b$, sem fatores primos em comum, de modo que $a^2+2017b^2$ possui mais de $n$ fatores primos distintos.
\end{enumerate}

\clearpage

\section{Soluções}

\subsection{Problema 1.}
\begin{problem}
Os pontos $X$, $Y$ e $Z$ estão marcados nos lados $AB$, $BC$ e $AC$ do triângulo $ABC$, respectivamente. Os pontos $A'$, $B'$ e $C'$ estão nos lados $XZ$, $XY$ e $YZ$ do triângulo $XYZ$, respectivamente, de modo que
\[
	\frac{AB}{A'B'}=\frac{AC}{A'C'}=\frac{BC}{B'C'}=2
\]
e $ABB'A'$, $BCC'B'$ e $ACC'A'$ são trapézios em que os lados do triângulo $ABC$ são bases.
\begin{itemize}
	\item[(a)] Determine a razão entre a área do trapézio $ABB'A'$ e a área do \\ triângulo $A'B'X$.
	\item[(b)] Determine a razão entre a área do triângulo $XYZ$ e a área do triângulo $ABC$.
\end{itemize}
\end{problem}

\clearpage

\subsection{Problema 2.}
\begin{problem}
Sabemos que o número real $C$ e números reais não-nulos $x$, $y$ e $z$, dois a dois distintos, satisfazem:
\[
	x+\frac{y}{z}+\frac{z}{y}=y+\frac{z}{x}+\frac{x}{z}=z+\frac{x}{y}+\frac{y}{x}=C.
\]
\begin{itemize}
	\item[(a)] Mostre que $C=-1$;
	\item[(b)] Exiba pelo menos uma solução $(x,y,z)$ para a equação dada.
\end{itemize}
\end{problem}

\clearpage

\subsection{Problema 3.}
\begin{problem}
Seja $n>1$ um inteiro e considere um tabuleiro $n\times n$, em que algumas
das $n^2$ casas foram pintadas de preto, e as restantes foram pintadas
de branco. Prove que é possível escolhermos uma das $n^2$ casas do
tabuleiro, de modo que, ao removermos completamente a linha e a coluna
que a contém, haja um número diferente de casas pretas e de casas brancas,
dentre as $(n-1)^2$ casas restantes.
\end{problem}

\clearpage

\subsection{Problema 4.}
\begin{problem}
Na Terra dos Impas, somente os algarismos ímpares são utilizados para contar e escrever números. Assim, em vez dos números $1,2,3,4,5,6,7,8,9,10,11,12,\ldots$ os Impas têm os números correspondentes $1,3,5,7,9,11,13,15,17,19,31,33,\ldots$ (note que os números dos Impas têm somente algarismos ímpares). Por exemplo, se uma criança tem $11$ anos, os Impas diriam que ela tem $31$ anos.
\begin{itemize}
	\item[(a)] Como os Impas escrevem o nosso número $20$?
	\item[(b)] Numa escola desse lugar, a professora escreveu no quadro-negro a continha de multiplicar abaixo. Se você fosse um aluno Impa, o que escreveria como resultado?
	      \[
		      13\times 5
	      \]
	\item[(c)] Escreva, na linguagem dos Impas, o número que na nossa representação decimal é escrito como $2017$.
\end{itemize}
\end{problem}

\clearpage

\subsection{Problema 5.}
\begin{problem}
No triângulo $ABC$, com $AB\ne AC$, seja $I$ seu incentro. Os pontos $P$ e $Q$ são definidos como os pontos onde o circuncírculo do triângulo $BCI$ intersecta novamente as retas $AB$ e $AC$, respectivamente. Seja $D$ o ponto de interseção de $AI$ e $BC$.
\begin{itemize}
	\item[(a)] Prove que $P$, $Q$ e $D$ são colineares;
	\item[(b)] Sendo $T$, diferente de $P$, o ponto de encontro dos circuncírculos dos triângulos $PDB$ e $QDC$, prove que $T$ está no circuncírculo do triângulo $ABC$.
\end{itemize}
Observação: O Incentro de um triângulo é o ponto de interseção de suas bissetrizes internas e o Circuncírculo de um triângulo é a circunferência que passa pelos seus três vértices.
\end{problem}

\clearpage

\subsection{Problema 6.}
\begin{problem}
Demonstre que, para todo $n$ inteiro positivo, existem inteiros positivos
$a$ e $b$, sem fatores primos em comum, de modo que $a^2+2017b^2$ possui
mais de $n$ fatores primos distintos.
\end{problem}

\clearpage

\section{Referências}\label{sec:references}
\end{document}
