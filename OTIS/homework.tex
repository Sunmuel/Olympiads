\documentclass[12pt]{article}
\usepackage[a4paper, margin=1in]{geometry}
\usepackage{amsfonts, amssymb, amsmath, amsthm}
\usepackage{xcolor}
\usepackage{lmodern}
\usepackage{fancyhdr}
\usepackage{tcolorbox}
\usepackage{enumitem}
\usepackage[font=small, labelfont=bf, justification=centering]{caption}

\definecolor{color1}{HTML}{941b0c}
\definecolor{color2}{HTML}{bc3908}
\definecolor{color3}{HTML}{f6aa1c}

\usepackage[colorlinks=true, linkcolor=black, urlcolor=color2, citecolor=black]{hyperref}

\title{\sffamily\bfseries{OTIS Homework Problems}}
\author{Samuel de Araújo Brandão}
\date{10 de Agosto de 2025}

\renewcommand*\contentsname{\textsf{Contents}}

\pagestyle{fancy}
\fancyhf{}

\fancyhead[L]{\sffamily\bfseries{OTIS Homework Problems}}
\fancyhead[R]{\textcolor{color2}{Samuel Brandão}, August 10 2025}
\fancyfoot[C]{\thepage}
\setlength{\headheight}{14.5pt}

\tcbset{
  problembox/.style = {
    colback=white,
    colframe=color2,
    boxrule=1pt,
    arc=2mm,
    top=1pt,
    bottom=1pt,
    left=1pt,
    right=1pt,
    fonttitle=\sffamily\bfseries\color{white},
    colbacktitle=color2,
    title={Enunciado do problema}
  }
}

\tcbset{
  claimbox/.style={
    colback=white,
    colframe=color1,
    boxrule=1pt,
  }
}

\begin{document}
  \maketitle

  A collection of solutions for Geometry, Inequalities and others, meant to be sent to Evan Chen, in the OTIS Application Homework.

  \tableofcontents
  
  \clearpage
  
  \section{\textsf{Problems}}
    \subsection{Geometry}
      \begin{enumerate}[label=\textbf{\arabic*.}]
        \item \textbf{(\#2.28, JMO 2012)} Given a triangle $ABC$, let $P$ and $Q$ be points on segments $AB$ and $AC$, respectively, such
          that $AP = AQ$. Let $S$ and $R$ be distinct points on segment $BC$ such that $S$ lies between $B$ and $R$, $\angle BPS = \angle PRS$,
          and $\angle CQR = \angle QSR$. Prove that $P$, $Q$, $R$, $S$ are concyclic.

        \item \textbf{(\#2.35, IMO 2009)} Let $ABC$ be a triangle with circumcenter $O$. The points $P$ and $Q$ are interior points of the sides $CA$
          and $AB$ respectively. Let $K$, $L$, $M$ be the midpoints of $BP$, $CQ$, $PQ$. Suppose that $PQ$ is tangent to the circumcircle 
          of $\triangle KLM$. Prove that $OP = OQ$.

        \item \textbf{(\#3.25, USAMO 1993)} Let $ABCD$ be a quadrilateral whose diagonals are perpendicular and meet at $E$. Prove that the reflections
          of $E$ across the sides of $ABCD$ are concyclic.
      \end{enumerate}

    \subsection{Inequalities}
      \begin{enumerate}[label=\textbf{\arabic*.}]
        \item Suppose that $a^2 + b^2 + c^2 = 1$ for positive real numbers $a, b, c$. Find the minimum possible value of
          \[
            \frac{ab}{c} + \frac{bc}{a} + \frac{ca}{b}.
          \]
        \item Let $a, b, c$ be positive real numbers such that $a^2 + b^2 + c^2 + (a+b+c)^2 \le 4$. Prove that
          \[
            \frac{ab + 1}{(a+b)^2} + \frac{bc + 1}{(b+c)^2} + \frac{ca + 1}{(c+a)^2} \ge 3.
          \]
        \item Let $a, b, c, d$ be positive reals with $(a+c)(b+d) = 1$. Prove that
          \[
            \frac{a^3}{b+c+d} + \frac{b^3}{c+d+a} + \frac{c^3}{d+a+b} + \frac{d^3}{a+b+c} \ge \frac13.
          \]
      \end{enumerate}
    \subsection{Additional}
      \begin{enumerate}[label=\textbf{\arabic*.}]
        \item Write a computer program to find the number of ordered pairs of prime numbers $(p,q)$ such that when
          \[
            N = p^2 + q^3
          \]
          is written in decimal (without leading zeros), each digit from $0$ to $9$ appears exactly once. For example, $(109,1163)$ is one
          such pair because $109^2 + 1163^3 = 1573049628$.
        \item Find all functions $f:\mathbb{R}\to\mathbb{R}$ for which
          \[
            f\big(xf(x)+f(y)\big)=f(x)^2+y
          \]
          holds for all real numbers $x$ and $y$.
        \item Let $a,b,c,d$ be real numbers such that $b-d\ge 5$ and all zeros $x_1,x_2,x_3,x_4$ of the polynomial $P(x)=x^4+ax^3+bx^2+cx+d$ are 
          real. Find the smallest value the product $(x_1^2+1)(x_2^2+1)(x_3^2+1)(x_4^2+1)$ can take.
        \item Ana and Banana are playing a game. First Ana picks a word, which is defined to be a nonempty sequence of capital English
          letters. Then Banana picks a nonnegative integer $k$ and challenges Ana to supply a word with exactly $k$ subsequences which are 
          equal to Ana's word. Ana wins if she is able to supply such a word, otherwise she loses. For example, if Ana picks the word “TST”,
          and Banana chooses $k=4$, then Ana can supply the word “TSTST” which has $4$ subsequences equal to Ana's word. Which words can Ana 
          pick so that she can win no matter what value of $k$ Banana chooses?
      \end{enumerate}

    \clearpage

  \section{\textsf{Geometry Solutions}}
    \
    \subsection{Problem 1.}
      \begin{tcolorbox}[problembox=Problem statement]
        Given a triangle $ABC$, let $P$ and $Q$ be points on segments $\overline{AB}$ and $\overline{AC}$,
        respectively, such that $AP = AQ$. Let $S$ and $R$ be distinct points on segment $\overline{BC}$
        such that $S$ lies between $B$ and $R$, $\angle BPS = \angle PRS$, and $\angle CQR = \angle QSR$.
        Prove that $P$, $Q$, $R$, $S$ are concyclic.
      \end{tcolorbox}
      By the $\textit{Alternate Segment Theorem}, \overline{AC}$ is tangent to $(QRS)$ and $\overline{AB}$
      is tangent to $(PRS)$. Assume for the sake of contradiction that $(QRS)$ and $(PRS)$ are distinct. In that case,
      $A \in \overline{BC}$ since $\overline{BC}$ is the radical axis and $\text{Pow}_{(QRS)}(A) =
      \text{Pow}_{(PRS)}(A)$. This leads to a contradiction, as $A \notin \overline{BC}$. Therefore,
      $P$, $Q$, $R$ and $S$ are concyclic.

    \clearpage

    \subsection{Problem 2.}
      \begin{tcolorbox}[problembox]
      \end{tcolorbox}

    \clearpage

    \subsection{Problem 3.}
      \begin{tcolorbox}[problembox]
      \end{tcolorbox}

  \clearpage

  \section{\textsf{Inequalities Solutions}}
    \subsection{Problem 1.}

    \clearpage

    \subsection{Problem 2.}

    \clearpage

    \subsection{Problem 3.}

  \clearpage

  \section{\textsf{Additional Solutions}}
    \subsection{Problem 1.}

    \clearpage

    \subsection{Problem 2.}

    \clearpage

    \subsection{Problem 3.}

    \clearpage

    \subsection{Problem 4.}

  \clearpage

  \section{\textsf{References}}
    This document was made possible thanks to the help and inspiration of the following resource:
    \renewcommand{\refname}{\vspace{-2em}}
    \begin{thebibliography}{9}
      \bibitem{noic}
      NOIC - Núcleo Olímpico de Iniciação Científica
      \textit{Potência de Ponto}, 2023.
      Available in: \url{https://noic.com.br/wp-content/uploads/2023/10/potencia.pdf}
      \bibitem{obm}
      OBM - Olimpíada Brasileira de Matemática.
      \textit{Potência de Ponto, Eixo Radical, Centro Radical e Aplicações}, 2017.
      Available in: \url{https://www.obm.org.br/content/uploads/2017/01/eixos-2.pdf}
      \bibitem{evan}
      Evan Chen.
      \textit{JMO 2012 Solution Notes}, 2025.
      Available in: \url{https://web.evanchen.cc/exams/JMO-2012-notes.pdf}
    \end{thebibliography}

\end{document}
