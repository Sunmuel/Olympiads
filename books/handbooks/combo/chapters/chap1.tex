\chapter{\textsf{Counting Basics}}

\section{Permutations}

A permutation can be defined as the arrangement of objects where the order matters.

\begin{comboexa}{}{}
  How many ways can one arrange 4 math books, 3 chemistry books, 2 physics
  books, and 1 biology book on a bookshelf so that all the math books are together, all the
  chemistry books are together, and all the physics books are together?
\end{comboexa}

\textcolor{c1}{\textbf{Solution.}} Consider the math books first. There are $4$ possibilities
for the first book, $3$ possibilities for the second, $2$ possibilities for the third and $1$ 
for the fourth.

By the \textit{Multiplication Principle}, there are $4 \times 3 \times 2 \times 1 = 4! = 24$ ways to arrange 
the math books. The symbol ``$!$'' denotes factorial, for integers $n \ge 1$, $n! = n \cdot (n-1) \cdot \ldots \cdot 1$

Similarly, there are $3!$ ways to arrange the chemistry books, $2!$ ways to arrange the physics books and
only $1!$ way to arrange the biology book. Besides that, there are $4!$ ways to arrange the four subject
blocks. The order matters, since it is a permutation.

Again, by the Multiplication Principle, one can say that there are
\[
  4! \cdot (4! \cdot 3! \cdot 2! \cdot 1!) = 6912
\]
ways to arrange the books on the bookshelf.

\begin{comboobs}{}{}
  Do not worry, later problems and solutions are going to be harder than this one, it is only an introduction.
\end{comboobs}

\begin{comboobs}{}{}
  $0! = 1$. This might be a surprise for you, but makes sense, since there is exactly one way to arrange $0$ elements:
  the empty arrangement.
\end{comboobs}

\begin{comboexa}{}{}
  From the five letters a,b,c,d,e, how many different 3-letter arrangements without repetition can be formed?
\end{comboexa}
This time, we can't use $5!$ or $3!$ directly, but the idea is the same. For the first letter, there are $5$ possibilities,
$4$ possibilities for the second and $3$ possibilities for the third. In total, there are  $5 \cdot 4 \cdot 3 = 60$.
This count is the number of $k$-permutations, as you can see in \cref{combodef:permutations}.

\begin{combodef}[label={combodef:permutations}]{Number of Permutations}{}
  The number of different groups of $k$ objects chosen from a total of $n$ objects with regard to order is equal to

  \[
    P(n,k) = \frac{n!}{(n-k)!}
  \]

  When $n,k \in \NN$ and $0 \le k \le n$.
\end{combodef}

\begin{comboobs}{}{}
  You might see the \textit{number of permutations} denoted by ${}_nP_k$ instead of $P(n,k)$ either.
\end{comboobs}

\section{Combinations}
A combination can be defined as the arrangement of objects where the order does not matter.

\subsection{Binomial Coefficient}
\begin{combodef}[label={combodef:binom}]{Binomial Coefficient}{}
  The number of different groups of $k$ objects chosen from a total of $n$ objects without regard to order is equal to

  \[
    \binom{n}{k} = \frac{n!}{k! \cdot (n-k)!}
  \]

  when $n \in \NN, k \in \ZZ$ and $0 \le k \le n$. $\binom{n}{k} = 0 \iff k < 0$ or $k > n$ and $\binom{n}{0} = \binom{n}{n} = 1$.
\end{combodef}

We derive it from $P(n,k)$ by dividing by $k!$, since order is ignored. We have to count $abc$ and $bca$, for example,
a repetition and divide $P(n,k)$ by $k!$, as we can see in the following.

\begin{comboexa}{}{}
  How many ways are there to choose $2$ letters from the set $\{a,b,c\}$?
\end{comboexa}

\textcolor{c1}{\textbf{Solution.}} Let's first demonstrate it using the \cref{combodef:permutations}. $P(3,2) \\ = 6$
gives $ab$, $ac$, $ba$, $bc$, $ca$, $cb$. Each subset is counted $2!$ times. Divide
by $2!$. Hence $\binom{3}{2} = 3$. That is exactly what goes behind \cref{combodef:binom}!

\begin{comboexa}[label={comboexa:multinom}]{}{}
  How many ways are there to divide $9$ people into one committee of $3$, one committee of $4$ and one committee of $2$?
\end{comboexa}

\textcolor{c1}{\textbf{Solution.}} For the first committee, we decide between $9$ people, so there are $\binom{9}{3} = 84$
possibilities. Once it is done, there are $6$ people left for the second committee, so there are $\binom{6}{4} = 15$ 
possibilities. After that, the remainder must go in the third committee. In total, there are $\frac{9!}{3! 4! 2!} =
84 \cdot 15 = 1260$ ways to divide the $9$ people.

This example illustrates what is called the \textit{multinomial coefficient}:

\subsection{Multinomial Coefficient}

\begin{combodef}[label={combodef:multin}]{Multinomial Coefficient}{}
  For nonnegative integers $k_1,k_2, \ldots, k_m$ with $k_1+k_2+\cdots+k_m=n$, the multinomial coefficient is
  defined as
  \[
    \binom{n}{k_1,k_2,\ldots,k_m}=\frac{n!}{k_1!k_2!\cdots k_m!}.
  \]

\end{combodef}
As you can see, the multinomial coefficient is a generalization of the binomial coefficient, where $k_m = n-k$
and $m=2$.

It would have been easier to solve the \cref{comboexa:multinom} using the multinomial coefficient, as you can see below.

\[
  \binom{n}{k_1,k_2,\ldots,k_m} = \binom{9}{3, 4, 2} = \frac{9!}{3!4!2!} = 1260.
\]

Let's finish this section with a really well known example.
\begin{comboexa}{}{}
  In how many ways can the letters of the word \texttt{MISSISSIPPI} be arranged?
\end{comboexa}

Let's use the multinomial coefficient to solve this one. There are 
\begin{enumerate}[label=\textbullet]
  \item $1$ letter \texttt{M},
  \item $4$ letters \texttt{I},
  \item $4$ letters \texttt{S},
  \item $2$ letters \texttt{P}.
\end{enumerate}
Therefore,
\[
  \binom{n}{k_1,k_2,\ldots,k_m} = \binom{11}{1,4,4,2} = \frac{11!}{1!4!4!2!} = 34650
\]

\subsection{Symmetry}

\begin{combodef}[label={combodef:binom}]{Binomial Coefficient}{}
  Choosing $k$ objects is the same as rejecting $n-k$ objects:
  \[
    \binom{n}{k} = \binom{n}{n - k}.
  \]
\end{combodef}

\begin{comboexa}{}{}
  In tossing a fair coin $10$ times, how many sequences contain exactly $8$ heads?
\end{comboexa}
\textcolor{c1}{\textbf{Solution.}} To solve this one, you could either count the times that exactly 
$8$ heads appear or count the times that exaclty $2$ tails appear, as you can see:
\[
  \binom{10}{8} = \binom{10}{2} = 45.
\]

\section{Pascal's Identity}

\begin{combothm}[label={combothm:pascalthm}]{Pascal's Identity}{}
  For all $n \ge 1$ and $1 \le k \le n-1$,
  \[
    \binom{n}{k} = \binom{n-1}{k-1} + \binom{n-1}{k}.
  \]
\end{combothm}

\begin{proof}
  It is well known that
  \[
    \binom{n-1}{k-1} = \frac{(n-1)!}{(k-1)!(n-k)!} \quad \text{and} \quad \binom{n-1}{k} = \frac{(n-1)!}{k!(n-1-k)}.
  \]
  Therefore,

  \begin{align*}
    \frac{(n-1)!}{(k-1)!(n-k)!} + \frac{(n-1)!}{k!(n-1-k)} & = \frac{(n-1)!k + (n-1)!(n-k)}{k!(n-k)!} \\
                                                           & = \frac{(n-1)!(k + (n-k))}{k!(n-k)!} \\
                                                           & = \frac{n!}{k!(n-k)!}. \qedhere
  \end{align*}
\end{proof}

\begin{comboexa}{}{}
  A club has $n$ members. In how many ways can you choose a committee of size $k$ if you separate the cases:
  \begin{enumerate}[label={\textbullet}]
    \item one special member (say Alice) is included,
    \item or Alice is not included?
  \end{enumerate}
\end{comboexa}

\textcolor{c1}{\textbf{Solution.}} 
\begin{enumerate}[label={\textbullet}]
  \item If Alice is included: choose the other $k-1$ from $n-1$ members, resulting in $\binom{n-1}{k-1}$.
  \item If Alice is not included: choose the $k$ from $n-1$ members, resulting in $\binom{n-1}{k}$.
\end{enumerate}

So total:
\[
  \binom{n-1}{k-1} + \binom{n-1}{k} = \binom{n}{k}.
\]
That is exactly Pascal's Identity in story form.


\section{The Binomial Theorem}

\begin{combothm}[label={combothm:binomthm}]{Binomial Theorem}{}
  For any nonnegative integer $n$ and $x$,$y \in \RR $ or $\CC$
  \[
    (x+y)^n = \sum_{k=0}^{n} \binom{n}{k} x^{n-k}y^k.
  \]
\end{combothm}

\section{Sum of Row}
\section{Alternating Sum}
\section{Vandermonde's Identity}

\begin{combocor}[label={combocor:vandermonde}]{Vandermonde's Identity}{}
\end{combocor}

\section{Hockey-Stick Identity}

\begin{combocor}[label={combocor:hockey}]{Hockey-Stick Identity}{}
\end{combocor}

\section{Grid Paths and Recursion}

\begin{combocor}[label={combocor:gridpaths}]{Grid Paths Recursion}{}
\end{combocor}

\section{Pigeonhole Principle}

\begin{combothm}{Pigeonhole Principle}{}
  If $n+1$ objects are placed into $n$ boxes, then some box contains at least two objects.
\end{combothm}
\begin{proof}
  Supose, for the sake of contradiction, that no box has more than one object. Then each of the $n$ boxes
  contain at most one object. Hence, the total number of objects would be at most $n$.

  But we assumed there were $n+1$ objects. Contradiction. Therefore, at least one box contains two or more 
  objects. \qedhere
\end{proof}

This is indeed a really simple concept, but at the same time, a tremendously important one.

\begin{comboexa}{}{}
  Choose $7$ different numbers from $\{1,2,\dots,12\}$. Prove that some pair sums to $13$.
\end{comboexa}
\textcolor{c1}{\textbf{Solution.}} There are exactly $6$ sums resulting in $13$ using each number of the set
$\{1,2,\dots,12\}$ exactly once:
\[
  1 + 12, \quad 2 + 11, \quad 3 + 10, \quad 4 + 9, \quad 5 + 8, \quad 6 + 7.
\]
Since we have to choose $7$ numbers from the set, but there are $6$ sums resulting in $13$, by the Pigeonhole Principle,
we'll need to choose two numbers from a single sum, and those two sum to $13$.

\begin{comboexa}{}{}
  Place $5$ points in the unit square. Show two are at distance less than or equal $\frac{\sqrt{2}}{2}$
\end{comboexa}
\textcolor{c1}{\textbf{Solution.}} Partition the unit square in four $\frac{1}{2} \cdot \frac{1}{2}$ subsquares.
Since there are $5$ points and $4$ squares, by the Pigeonhole Principle, a square must contain $2$ points.
The greatest distance between two points in a subsquare is $\sqrt{\frac{1}{2}^2 + \frac{1}{2}^2} = \frac{\sqrt{2}}{2}$.


\begin{comboexa}{}{}
  In any set of $n+1$ integers, chosen \\ from $\{1,2,\dots,2n\}$, prove that one number divides another.
\end{comboexa}
\textcolor{c1}{\textbf{Solution.}} 
\begin{combolem}{}{}
  Every positive integer $a$ can be expressed as $2^bc$, where $b = \{x \ge 0 \mid x \in \ZZ^+\}$ and
  $c = 2y+1$ for some positive integer $y$.
\end{combolem}
\begin{proof}
  Let's split this proof.
  \begin{itemize}
    \item $a$ is even: keep dividing $a$ by $2$ as long as the result is even. Eventually, you will stop,
      because dividing an integer by $2$ cannot go on forever. When you stop, the number you reach must
      be odd. Call that number $c$.
    \item $a$ is odd: $b=0$, so $2^b = 1$ and $c=a$. \qedhere
  \end{itemize}
\end{proof}

Since there are $n$ odd numbers inside $\{1,2,\dots,2n\}$ and we chose $n+1$ numbers, by the Pigeonhole 
Principle there must be two numbers that have the same $c$ in $2^bc$. Those two numbers are 
certainly divisible by each other.

\begin{proof}{}{}
  Let's say that, WLOG, $b_2<b_1 \implies \frac{2^{b_1}c}{2^{b_2}c} = 2^{b_1-b_2}$, an integer number.
\end{proof}

\begin{comboexa}{}{}
  Among $6$ people, prove there exist $3$ mutual acquaintances or $3$ mutual strangers.
\end{comboexa}
\textcolor{c1}{\textbf{Solution.}} Let's fix one person, call them $A$. There are $5$ people left. $A$ has
$5$ possible relations: each of the $5$ others is friend or stranger. By the Pigeonhole Principle, at least
$3$ of them must fall into the same category (since $5=2\cdot2+1$).
\begin{itemize}
  \item $A$ has $3$ friends $B$, $C$, $D$: if any two of $B$, $C$ and $D$ are friends, say $B$ and $C$, then we have a 
    triangle of friends: $A$, $B$, $C$. If none of them are friends with each other, then $B$, $C$, $D$ are pairwise
    strangers, so we have a triangle of strangers.
  \item $A$ has $3$ strangers $E$, $D$, $F$: if any two of $E$, $F$, and $G$ are strangers, then with $A$ we form a 
    triangle of strangers. If they are all mutual friends, then $E$, $D$ and $F$ form a triangle of friends.
\end{itemize}

\subsection{Generalized form}
\begin{combothm}{Generalized Pigeonhole Principle}{}
  If $n$ objects are distributed among $m$ boxes, then some box contains at least
  \[
    \left\lceil \frac{n}{m} \right\rceil
  \]
  objects.
\end{combothm}

\begin{proof}
  Suppose, for the sake of contradiction, that no box has more than $\left\lfloor \frac{n}{m} \right\rfloor$
  objects. Then, each of the $m$ boxes contain at most $\left\lceil \frac{n}{m} \right\rceil - 1 = \left\lfloor 
  \frac{n}{m} \right\rfloor$. Hence, the total number of objects would be at most $m \cdot \left\lfloor 
  \frac{n}{m} \right\rfloor$.

  But we assumed there were $n$ objects. Contradiction, since $\left\lfloor \frac{n}{m} \right\rfloor < \frac{n}{m}$.
  Therefore, at least one box contains at least $\left\lceil \frac{n}{m} \right\rceil$ objects. \qedhere
\end{proof}

\section{Problems}
